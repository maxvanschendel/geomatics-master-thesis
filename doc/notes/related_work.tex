\section{Related work}
Previous research on mapping and map merging has considered various map representations \citep{tomatis_hybrid_2003,huang_topological_2005,bonanni_3-d_2017,gholamishahbandi_2d_2019}. According to \citet{andersone_heterogeneous_2019} and \citet{yu_review_2020} these representations can be subdivided into three types: metric-, feature-, and topological maps. Hybrid maps that are combinations of two or more map types also exist, such as topometric maps, which are a combination of metric and topological maps \citep{yu_review_2020}. Map types that are not one of the three main types or a hybrid are rarely used but do exist \citep{yu_review_2020}. In this section we will discuss the work that has been done on extracting and merging metric and topometric maps. Figure \ref{fig:euler} shows a diagram of the fields of research that are relevant for this thesis.

\begin{figure}[h]
    \centering
    \includegraphics*[width=.8\textwidth]{./fig/euler_diagram.drawio.pdf}
    \caption{Euler diagram showing the overlapping fields of research that are relevant for this thesis.}
    \label{fig:euler}
\end{figure}


\subsection{Metric Maps}
\label{section:metric_map_merging}
In this section we give an overview of the existing research into metric map extraction and map merging.

\subsubsection{Metric map extraction}
Metric maps are a map representation that represent the geometry of an environment. Two common metric map types that are relevant for our research are point clouds and voxel grids. Point clouds are usually the direct output of 3D mapping sensors and algorithms so it is not necessary to extract them from another map representation \citep{rusu_3d_2011}. \citet{volodine_point_2007} gives an overview of point clouds and how to process them. \citet{elfes_occupancy_1990} gives a description of voxel grids and how to extract them from point cloud maps.

\subsubsection{Metric map merging}
Metric map matching is the problem of recognizing overlapping areas between partial maps based purely on their geometry. Metric map matching is a mature area of research that has applications for 3D mapping and place recognition.  Some approaches use \gls{localdescriptor}s that describe the properties of each point in the point cloud to identify corresponding points between point clouds. Descriptors may include corners, lines, planes or other points of interest, e.g. SIFT, SURF, FPFH or Harris points \citep{andersone_heterogeneous_2019,rusu_fast_2009}. Some descriptors are better suited for different environments and input data. For example, the descriptor approach by \citet{li_general_2010} is scale-independent and the approach of \citet{yang_fast_2016} is able to deal with differences in resolution. Recent research into using deep learning for local descriptor extraction has also shown great potential, examples include PointNet and DGCNN \citep{qi_pointnet_2017,phan_dgcnn_2018}.

Another approach to metric map matching uses \gls{globaldescriptor}s that describe the properties of segments of the point cloud, overlapping segments can then be identified based on the similarity of their global descriptor \citep{dube_segmatch_2017}. A large number of global descriptors have been proposed which describe the point cloud based on various properties, such as its volume, planarity or roughness \citep{han_comprehensive_2018}. Another approach to global descriptor extraction aggregrates the local descriptor of each point into a global descriptor, examples include DBoW and VLAD \citep{shan_robust_2021,arandjelovic_all_2013}. Global descriptors can also be based on the spectral characteristics of graphs derived from the point cloud, including approaches such as ShapeDNA and heat kernel signatures \citep{reuter_laplacebeltrami_2006,bronstein_scale-invariant_2010}. An advantage of these approaches is that they do not require manual selection of relevant descriptors. Finally, recent research has shown that deep learning can be used to extract global descriptors, with approaches such as PointNetVLAD, LPDNet and MinkLoc3D \citep{uy_pointnetvlad_2018-1,liu_lpd-net_2019,komorowski_minkloc3d_2021}. Although these approaches claim to have better resilience against differences between partial maps than other approaches they often require large amounts of training data and may not be able to deal with environments that are not similar to the training data. Note that the above deep learning approaches often use a trainable variation of the local descriptor aggregration described above \citep{arandjelovic_netvlad_2016}. 

Metric map fusion is a mature area of research and various approaches have been proposed. The problem of metric map fusion comes down to finding a transformation between partial maps that brings the geometry of their overlapping areas into alignment, this is called registration. Most approaches to registration include a variation of the iterative closest point (ICP) algorithm. This algorithm finds the transformation between two point clouds by iteratively applying rigid transformations that minimize the distance between the points in one point clouds and their closest points in the other \citep{rusinkiewicz_efficient_2001}. Since its first introduction a number of variations on the ICP algorithm have been proposed that improve its accuracy and performance. An example of this is the normal iterative closest point algorithm, which improves data-association by taking into account the normal vector of a point's neighbourhood and the gravity-aligned ICP algorithm, which constrains the transformation to use with maps that have a consistent up direction \citep{serafin_nicp_2015,kubelka_gravity-constrained_2022}. 

The ICP algorithm and its variations are not guarantueed to find a globally optimal transformation and they are sensitive to the initial transformation between point clouds. To mitigate this \citet{yang_fast_2016} proposes a two step algorithm that first performs a rough, global registration followed by a precise, local registration (see figure \ref{fig:global_local}). The function of the global registration step is to find a good initialization that will allow the local registration step to find the global optimum. Various approaches to global registration have been proposed, with most depending on the detection of correspondences between point clouds using one of the descriptor approaches described above. A global transformation can then be estimated using RANSAC \citep{koguciuk_parallel_2017}.

\begin{figure}[h]
    \centering
    \includegraphics*[width=\textwidth]{./fig/feature_matching.png}
    \caption{Illustration of two-step approach to registration \citep{yang_fast_2016}.}
    \label{fig:global_local}
\end{figure}

More recently, deep learning approaches for point cloud registration have been proposed, such as PointNetLK and DCP \citep{aoki_pointnetlk_2019,wang_deep_2019}. However, these methods require training and their results can not be constrained without retraining the model.

\pagebreak

\subsection{Topometric Maps}
In this section we discuss the existing research into the extraction and merging of topometric maps.

\subsubsection{Topometric map extraction}
Various approaches for extracting topological maps from raw sensor data or metric maps have been proposed. \citet{bormann_room_2016} and \citet{pintore_state---art_2020} both give a review of different methods for extracting structured maps of indoor spaces in which different room segmentation methods are discussed and compared.

\citet{kuipers_robot_1991} first proposes identifying distinctive places, nodes of the topological map, directly from sensor data by finding local maxima of a distinctiveness measure within a neighbourhood. Edges are identified by having the robot try to move between vertices, if this is possible an edge is created. Local metric information in the form of an occupancy grid is associated with the nearest vertex in the graph, resulting in a topometric map. Note that this approach is dependent on the mapping agent's exploration strategy. 

% visibility clustering
\citet{thrun_learning_1998} extracts a 2D topological map from a 2D metric map by identifying narrow passages. They then partition the metric map into areas divided by passages which are respectively the vertices and the edges of the graph. This approach is not able to deal with 3D environments with multiple storeys and it assumes that rooms are always separated by narrow passages.

\citet{mura_automatic_2014} proposes an approach to room segmentation that divides the ground plane of the environment into polygons which are then iteratively clustered to form rooms. \citet{mura_piecewise-planar_2016} avoids this clustering step by directly clustering scanner positions using a graph clustering approach. This approach is then further improved by \citet{ambrus_automatic_2017} by generating synthetic scanning positions along the medial axis of the map. Figure \ref{fig:vis_cluster} illustrates the concept behind visibility clustering-based room segmentation.

\begin{figure}[h]
    \centering
    \includegraphics*[width=.7\textwidth]{./fig/visibility_clustering.png}
    \caption{Illustration of visibility clustering-based room segmentation \citep{pintore_state---art_2020}.}
    \label{fig:vis_cluster}
\end{figure}

\citet{ochmann_towards_2014} describes extracting hierarchical topometric maps directly from point clouds. The hierachy is divided into four encapsulating layers, building - storey - room - object. Entities within the graph are represented as vertices, with edges representing the topological and spatial relationships between entities. Each vertex is linked to a local metric map of the entity's geometry. 

\citet{gorte_navigation_2019} provides a novel approach for extracting the walkable floor space from a voxel grid across multiple storeys (see figure \ref{fig:gorte}). They do so by first applying a 3D convolution filter using a stick-shaped kernel to extract the parts of the floor without obstructions. They then apply an upwards dilation to connect steps of stairways into a connected volume. Because the topology of the environment depends on the traversability between spaces the extraction of navigable floor space is essential for the extraction of topological maps.

\begin{figure}[h]
    \centering
    \includegraphics*[width=.7\textwidth]{./fig/gorte.png}
    \caption{Illustration of walkable floor space extraction \citep{gorte_navigation_2019}.}
    \label{fig:gorte}
\end{figure}

\citet{he_hierarchical_2021} describes an approach for extracting a hierarchical topological-metric map with three layers: storey - region - volume, from a voxel grid map. To extract the map they use a novel approach to room segmentation using raycasting. A downside of their methodology is that it depends on the presence of ceilings in the metric map, which are often not captured in practice when using handheld scanners. 

% deep learning
\citet{ma_semantic_2020} and \citet{tang_bim_2022} both propose a deep learning approach to semantic segmentation of indoor spaces. Despite the fact that the primary goal of these methods is not room segmentation but instance segmentation (splitting the map into walls, floor, furniture, etc.) their findings can also be applied to room segmentation. In contrast to previous approaches these approaches require training a model on a labelled dataset.

\subsubsection{Topometric map merging}
In comparison to topometric map extraction relatively little work has been done on the subject of topometric map merging. \citet{dudek_topological_1998} first proposes an approach to topological map merging which depends on a robot meeting strategy to merge partial maps created by each robot. When new distinctive places are recognized at the frontier of the global map the other robots will travel towards it and synchronize their maps. As with other early approaches to map extraction and map merging this approach depends on a coordinated exploration strategy.

The work of \citet{huang_topological_2005} is a significant milestone in topological map merging, demonstrating that topological maps can be merged using both map structure and map geometry. They first identify vertex matches by comparing the similarity of their attributes, such as their degree, and the spatial relationships of incident edges. Vertex matches are then expanded using a region growing approach, where every added edge and vertex is compared for similarity and rejected if too dissimilar. The results are multiple hypotheses for overlapping areas between partial maps. They then estimate a rigid transformation between the partial maps for each hypothesis. Afterwards, hypotheses that result in similar transformations are grouped into hypothesis clusters. They then select the most appropriate hypothesis cluster by using a heuristic that includes the number of vertices in the cluster, the error between matched vertices after transformation and the number of hypotheses in the cluster.

\citet{bonanni_3-d_2017} provides a unique approach to topometric map merging using the pose graph of mapping agents as the topological component and the point cloud captured at each node as the metric component. Matches between nodes are identified by computing the similarity of their associated point cloud. In comparison to most other map merging approaches they fuse the maps using a non-rigid transformation, meaning the partial maps are deformed to improve their alignment. 

\citet{garcia-fidalgo_hierarchical_2017} proposes a hierarchical approach for place recognition in topological maps in which images of the environment are grouped by similarity and described by both a local descriptor of their properties and a global descriptor of their grouping's properties. This approach reduces the search space when recognizing places. Note that this approach does not use a 3D metric component but an image-based one.

\citet{rincon_map_2019} proposes an approach to topological map merging that is based on both the similarity of nodes and their context within the graph based on a model of human object recognition. This approach depends on the previous alignment of partial maps.

\citet{rozemberczki_multi-scale_2021} propose an approach to feature embedding in graphs that combines each node's associated attribute with the distribution of its neighbourhood's attributes over multiple scales. While they do not use this for 3D mapping their approach can be applied to topometric maps.