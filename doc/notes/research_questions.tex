\section{Research Questions}

\subsection{Main question}
To what extent can topometric representations of indoor environments be applied to heterogeneous map merging?

\subsection{Subquestions}
\begin{enumerate}
    \item In what way can partial topological-metric maps be extracted from heterogeneous partial point cloud maps?
    \item What approach is best suited for identifying matches between partial hierarchical topological-metric maps?
    \item How can the identified matches be used to fuse two or more partial hierarchical topological-metric maps into a global map?
\end{enumerate}

\subsection{Scope}
To better delineate the scope of the thesis we provide several aspects that will \textbf{not} be researched or discussed. 

\begin{enumerate}
    \item Map merging using known relative poses between agents or meeting strategies. Agent behaviour is assumed to be independent and agents are not able to sense eachother.
    \item Map merging using observations unrelated to the environment's geometrical and topological characteristics. E.g. the environment's colour or actively transmitted beacon signals.
    \item Map merging assisted by a priori knowledge of the environment. E.g. building information models (BIM) or floor plans.
    \item Map merging using the pose graphs of agents. Agent poses are assumed to be unknown.
    \item Achieving (near) real-time performance.
\end{enumerate}