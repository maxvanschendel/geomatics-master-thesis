\section{Discussion}

\subsection{Map Extraction}
In the previous section we showed the results of our map extraction approach. In this section we will discuss these results. For the majority of tested environments the resultant room segmentation closely matches the ground truth room segmentation. This is especially the case in environments where rooms have clear delineations; environments where rooms have walls between them and are only connected by small openings. The results of our room segmentation approach match less closely in environments where this is not the case. Our approach often splits single rooms that are large in one or multiple dimensions, such as hallways or auditoriums, into multiple rooms. Additionally, rooms where there are obstructions to the view from inside the room are also split into multiple parts. However, these effects do not necessarily indicate a failure of our approach. The segmentation in the ground truth data is based on human intuition about what separates a room from its neighbours. Although room segmentation based on visibility clustering often closely matches this intuition it is inherently different as it does not take into account the intended use of rooms. Where humans might recognize that a long hallway or a large hall serves a single purpose, and should therefore be considered as the same room, visibility clustering fails to take this subjective interpretation of purpose into account. Nevertheless, the objective visibility clustering approach comes remarkably close to the subjective human approach. The opposite also occurs, where two or more rooms that are separate in the ground truth data are not separate in the room segmentation. This mostly occurs when there are no obstructions between two adjacent rooms. This can often be resolved by changing the clustering parameters. However, when the only separation between two rooms is based on purpose and not on visibility then our approach cannot separate them.

A common failure mode of our approach, which causes room segmentation to fail completely, is when the input point cloud data is of insufficient density to construct a connected navigation graph. In this case, the voxels belonging to the navigation graph are identified correctly but there are gaps between voxels. This can be solved by increasing the size of the kernel used for constructing the neighbourhood graph of the navigable voxels. However, this has the side effect that voxels that are not actually navigable are added to the navigation graph. The result is that low elevated surfaces with sloping sides, beds for example, are added to the navigation graph. While this usually does not have a large effect on map extraction in extreme cases it can also cause the ceiling to become part of the navigation graph. This will usually cause significant errors in room segmentation, as the view from above the ceiling towards the rest of the map is often completely unobstructed.

Another way that room segmentation may fail is when stairs have very shallow treads and steep rises (respectively the horizontal and vertical part of its steps). This causes the stick kernel approach to fail to label the stairs' voxels as navigable, which means it will not be included in the navigation graph. This is because the wide part of the stick kernel placed on one step may intersect with the next step. If there are no other connections between two storeys then this will cause one or multiple storeys to become disconnected from the navigation graph, excluding it from the extracted topometric map. This problem can be resolved by changing the dimensions of the stick kernel. However, this may in turn cause other problems. Increasing the height of the thin part of the stick kernel causes low elevated surfaces with sloping sides to be included in the navigation graph, as described in the previous paragraph. Decreasing the radius of the stick kernel's wide part will include parts of the map that are not actually navigable in the navigation graph. The problem can also be solved by increasing the size of the kernel used to construct the navigable voxels' neighbourhood graph to force the stairs'  voxels to become connected even though some are missing but this causes the same issues as described in the previous paragraph.

Differences between the ground truth topological graph and the extracted topological graph are caused by differences in room segmentation. One such case is when a hallway connected to a room is split into multiple rooms around the opening towards the connected room. This will result in a triangular subgraph between the two parts of the hallway and the connected room, which in reality should just be a single edge between the hallway and the room. 

\subsection{Map Matching}
\subsection{Map Fusion}
\subsection{Future Works}