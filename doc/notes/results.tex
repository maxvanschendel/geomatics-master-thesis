\section{Results}
We evaluated the algorithm described in the methodology section on several datasets. In this section we describe these datasets and show the results we achieved with them. The shown results are divided into three sections based on the major components of our methodology: map extraction, map matching and map fusion. 

\subsection{Datasets}
In this section we describe the datasets that we used to test our algorithm. We also describe how we prepared the ground truth datasets so as to be able to compare our results to them and objectively measure their performance. 

\subsubsection{Simulated Scan}
To objectively evaluate the performance of our methodology it is necessary to compare the results to a ground truth. This ground truth must contain the following elements: room labels and their topological relationships to evaluate map extraction, room matches between partial maps to evaluate map matching, and the transformations between partial maps to evaluate map fusion. When capturing real-world data, the second and third elements are especially difficult to determine. Furthermore, little pre-existing data is available that contains exactly these elements. To solve this problem we simulate partial maps from annotated global maps from various sources. The annotations are integer labels for every point representing the ground truth room segmentation and a graph representing the ground truth topological map. 

We create the partials maps by manually defining a trajectory for each desired partial map and simulating an agent moving along them, scanning the global map at a set interval. We do this by using the DDA algorithm described in the methodology section. This allows us to objectively evaluate our approach for a large number of different scenarios at the cost of missing some of the subtleties inherent in non-simulated partial maps, such as changes in the environment and measurement error. To mitigate this, we apply pre-processing steps to the global map. These pre-processing steps are: random removal of points, adding noise to points and random rotation and translation of the point cloud. The latter's purpose is to simulate the unknown transformation between real-world partial maps. By comparing the results of our approach to the annotations in the ground truth global map we can objectively measure the performance of map extraction, matching and fusion. 

\subsubsection{Stanford 3D Indoor Scene Dataset}
The Stanford 3D Indoor Scene Dataset (S3DIS) consists of a collection of 3D scans of six different indoor environments (Area 1 through 6) and the raw measurements used to create them. The environments all span a single floor. Each environment scan in its original state consists of a number of point clouds, one for each room in the building. We merge these separate point clouds into a single cloud but retain the room labels as point attributes, as we use these as our ground truth room labels for the map extraction step. We manually created the topological graph of each area. The S3DIS was captured using high-end 3D scanners and thus has a high point density.

\subsubsection{Collaborative SLAM Dataset}
The Collaborative SLAM Dataset (CSLAMD) is a dataset meant specifically for collaborative SLAM. It consists of three environments (House, flat and lab), each consisting of multiple partial maps and their ground truth transformations. Two of the environments consist of multiple storeys. The partial maps were captured using a low-end 3D scanner and thus has a low point density. We merge the partial maps of each environment into a single global map to create the simulated partial maps described above. We then manually annotate each environment with our interpretation of an appropriate room segmentation and manually create the topological graph.

\subsubsection{Various Sources}
Finally, we also include datasets from various sources. These include a two-storey house with basement and an office conference room, both captured using high-end 3D scanners. We annotated these datasets by hand according to our interpretation of the spaces and their segmentation into rooms and manually create the topological graph. 

\subsubsection{Overview}

\begin{tabular}{ c c c c }
    Scan & Dataset & Number of rooms & Number of storeys \\ 
    Area 1 & S3DIS & 44 & 1 \\  
    Area 2 & S3DIS & 40 & 1 \\
    Area 3 & S3DIS & 23 & 1 \\
    Area 4 & S3DIS & 49 & 1 \\
    Area 5 & S3DIS & 68 & 1 \\
    Area 6 & S3DIS & 48 & 1 \\

    House & CSLAMD & cell9 & 2 \\
    Flat & CSLAMD & cell9 & 1 \\
    Lab & CSLAMD & cell9 & 4 \\

    House2 & Various & 23 & 2 \\

\end{tabular}

\subsection{Map Extraction}
In this section we show the results of our map extraction approach. To evaluate the results of our map extraction approach we compare the extracted topometric partial maps to the ground truth global map. For each node's geometry in a topometric map we find its Jaccard index with every node in the global map's geometry. This gives us a weighted bipartite graph, one part being the nodes in the partial map and the other being the nodes in the global map, with the weight representing the Jaccard index between nodes' geometry. We then assign each node in the partial map to a single node in the global map such that the sum of weights is maximized using linear assignment as described in the map matching section. This gives us the correspondence between nodes in the partial map and nodes in the global map. Afterwards, we compute the mean Jaccard index of the correspondences. This metric is called Mean Intersection over Union (MIoU). 

\begin{tabular}{ c c c c }
    Scan & Dataset & MIoU \\ 
    Area 1 & S3DIS & 0.732 \\  
    Area 2 & S3DIS & 0.828 \\
    Area 3 & S3DIS &  \\
    Area 4 & S3DIS &  \\
    Area 5 & S3DIS &  \\
    Area 6 & S3DIS &  \\

    House & CSLAMD &  &  \\
    Flat & CSLAMD &  &  \\
    Lab & CSLAMD &  &  \\

    House2 & Various &  &  \\

\end{tabular}

\subsection{Map Matching}
In this section we show the results of our map matching approach. Using to partial map to global map node correspondences described in the previous section we can distinguish incorrect matches from correct matches; a match between two partial maps is correct if both nodes in the match correspond to the same node in the global map. With this information we evaluate our map matching approach based on four metrics: precision, accuracy, recall and F1. 

\begin{tabular}{ c c c c c c}
    Scan & Dataset & Precision & Accuracy & Recall & F1 \\ 
    Area 1 & S3DIS &  &  & & \\  
    Area 2 & S3DIS &  &  & &\\
    Area 3 & S3DIS &  &  & &\\
    Area 4 & S3DIS &  &  & &\\
    Area 5 & S3DIS &  &  & &\\
    Area 6 & S3DIS &  &  & &\\

    House & CSLAMD &  &  & &\\
    Flat & CSLAMD &  &  & &\\
    Lab & CSLAMD &  &  & &\\

    House2 & Various &  &  \\

\end{tabular}


\subsection{Map Fusion}
In this section we show the results of our map merging approach. We evaluate the performance of our map merging approach by comparing the computed transformation to the transformations applied to the ground truth when simulating the partial maps. We also evaluate the extracted topological graph by computing its edit distance to the ground truth global map's topological graph.

\begin{tabular}{ c c c c }
    Scan & Dataset & Registration Error & Edit Distance \\ 
    Area 1 & S3DIS &  &  \\  
    Area 2 & S3DIS &  &  \\
    Area 3 & S3DIS &  &  \\
    Area 4 & S3DIS &  &  \\
    Area 5 & S3DIS &  &  \\
    Area 6 & S3DIS &  &  \\

    House & CSLAMD &  &  \\
    Flat & CSLAMD &  &  \\
    Lab & CSLAMD &  &  \\

    Elspeet & Various &  &  \\

\end{tabular}
