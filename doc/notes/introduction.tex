%TODO: MODIFY INTRO TO WORK WITH CHANGE OF SCOPE AND CONTENTS

\section{Introduction}

% why is it important
3D \gls{map}s of indoor environments are used for a wide range of applications, ranging from domestic to industrial, and may be used to provide comprehension of the environment to humans through cartography or for automated scene understanding that enables complex robot behaviour \citep{chen_indoor_2020,wang_safe_2019,hermann_design_2016}. In many cases it is advantageous for multiple mapping agents to collaboratively create a map, such as during rescue operations where the location of the subject(s) is unknown \citep{queralta_collaborative_2020}. By working together larger areas can be mapped in a shorter amount of time \citep{lajoie_towards_2022}. Furthermore, each agent may perceive the environment differently, resulting in a more complete whole \citep{schuster_arches_2020}.

\begin{figure}[h]
    \centering
    \includegraphics*[width=.7\textwidth]{./fig/overview_diagrams-Page-4.drawio.pdf}
    \caption{The map merging problem. How do you combine partial maps if relative positions and orientations are unknown?}
    \label{fig:map_merging}
\end{figure}

% what is it
The problem of creating a single \gls{globalmap} from multiple \gls{partialmap}s is called \gls{mapmerging}. Map merging is challenging because the relative positions and orientations (and sometimes scale) of agents within the environment are unknown. In indoor environments where external positioning signals like GNSS are highly attenuated map merging can only rely on the properties of the partial maps themselves. Figure \ref{fig:map_merging} illustrates the map merging problem, showing two agents with unknown relative positions and orientations within the same environment. Map merging can be subdivided into two subproblems. The first is \gls{mapmatching}, the identification of overlapping areas between partial maps. The second is \gls{mapfusion}, the alignment and combination of the partial maps based on the overlapping areas. 

\begin{figure}[h]
    \centering
    \includegraphics*[width=.8\textwidth]{./fig/overview_diagrams-Page-2.drawio.pdf}
    \caption{Extraction of topometric maps from raw data.}
    \label{fig:extract_intro}
\end{figure}

Map matching is essentially a problem of place recognition, or identifying the same place in different maps despite differences in appearance. Human place recognition uses a combination of the visible properties of that place and the structure, or topology, of its environment \citep{kuipers_modeling_1978}. To illustrate, when asked to describe a room, someone may answer that it has an L-shape but also that the room has two neighbouring rooms. In this thesis we propose an approach that uses both the geometric and the topological properties of indoor environments to solve the map matching problem. For this purpose we also propose an approach for extracting \gls{topometricmap}s, which represent both the environment's geometry and its room-level topological structure, from purely geometric \gls{pointcloud} maps. We further propose an approach to fuse the geometry and topology of the partial topometric maps into a global topometric map based on the identified matches. Figures \ref{fig:extract_intro}, \ref{fig:match_intro} and \ref{fig:fuse_intro} respectively illustrate the problems of map extraction, matching and fusion.

\begin{figure}[h]
    \centering
    \includegraphics*[width=.7\textwidth]{./fig/overview_diagrams-Page-3.drawio.pdf}
    \caption{Map matching using both geometric and topological properties.}
    \label{fig:match_intro}

\end{figure}

We hypothesize that the topological structure of rooms within the environment are consistently identifiable between partial maps. We further hypothesize that using the metric characteristics of the environment in conjunction with its topological characteristics will improve identification of overlapping areas over a purely geometric approach. Finally, we hypothesize that the identified matches can be used to fuse both the geometry and topology of the partial topometric maps into a global topometric map.

\begin{figure}[h]
    \centering
    \includegraphics*[width=.9\textwidth]{./fig/overview_diagrams-Page-1.drawio.pdf}
    \caption{Fusion of partial topometric maps into a global topometric map.}
    \label{fig:fuse_intro}
\end{figure}

% main contributions
The most important contributions of our work are:

\begin{enumerate}
    \item \Gls{mapextraction} of 3D topometric maps from point clouds.
    \item Map matching using both the geometry and topology of topometric maps.
    \item Map fusion of topometric maps.
\end{enumerate}
