%TODO: MODIFY INTRO TO WORK WITH CHANGE OF SCOPE AND CONTENTS

\section{Introduction}
Collaborative mapping allows multiple agents to work together to create a single, global map of their environment. By working together large areas can be mapped in a short amount of time. Most existing research has focused on when mapping agents are homogeneous, meaning they sense their environment and behave similarly \citep{andersone_heterogeneous_2019}. However, there are situations where mapping with heterogeneous agents can be advantageous \citep{hermann_design_2016}. For example, a human carrying lightweight, low-end sensors collaborating with a robot carrying heavy, high-end sensors to map an environment where some areas are not accessible by humans. By compensating for the weaknesses of one agent with the strengths of another more environments and situations can be handled. 

Existing collaborative mapping approaches are not well suited for mapping with heterogeneous agents as they often rely on agents being able to communicate with eachother. This is especially the case in indoor environments where external positioning signals are highly attenuated. In this case, a global map can only be created by merging the partial maps of the environment created by each agent individually based on their overlapping areas. This is called map merging (see figure \ref{fig:map_merging}). When partial maps are created by heterogeneous agents they might represent different aspects of the environment at different a different scale, resolution or accuracy, which further complicates map merging as overlapping areas may not appear the same between partial maps.

\begin{figure}[h]
    \centering
    \includegraphics*[width=\textwidth]{./figures/raster/map_merging.png}
    \caption{Partial maps (1) captured by three different agents and resultant global map (2) after map merging.}
    \label{fig:map_merging}
\end{figure}

% the method of solving the problem, often stated as a claim or a working thesis;
In this thesis we propose to use both the hierarchical topological relationships of indoor environments, meaning the connectivity between distinctive places and their nesting relationships, and their metric characteristics, the geometry of the environment, to solve the heterogeneous map merging problem.  We do this by extracting 3D hierarchical topological-metric maps from heterogeneous partial maps. We then use both the topological and metric characteristics of the partial maps to robustly identify overlapping areas that might appear differently due to being captured by heterogeneous agents.  Our hypothesis is that the connectivity and hierarchy of places within the environment are identifiable between heterogeneous partial maps. We further hypothesize that using the metric characteristics of the environment in conjunction with its topological characteristics will improve identification of overlapping areas over a purely topological approach, despite geometrical differences between heterogeneous partial maps. The most important contributions of our work are: 1) applying 3D hierarchical topological-metric map merging to indoor environments. 2) extracting 3D hierarchical topological-metric maps from heterogeneous partial maps. In the rest of this section we will give more detailed definitions for the concepts mentioned above and others that are commonly used in this report.

