%TODO: MODIFY INTRO TO WORK WITH CHANGE OF SCOPE AND CONTENTS

\section{Introduction}
Collaborative mapping allows multiple agents to work together to create a single, global map of their environment. By working together large areas can be mapped in a short amount of time. Collaborative mapping of indoor environments is especially challenging, because external positioning signals are highly attenuated. In this case, a global map can only be created by merging the partial maps of the environment created by each agent individually based on their overlapping areas. This is called map merging (see figure \ref{fig:map_merging}). Partial maps might represent the environment at different a different scale, resolution or accuracy, which further complicates map merging as overlapping areas may not appear the same between partial maps.

\begin{figure}[h]
    \centering
    \includegraphics*[width=\textwidth]{./fig/map_merging.png}
    \caption{Partial maps (1) captured by three different agents and resultant global map (2) after map merging.}
    \label{fig:map_merging}
\end{figure}

% the method of solving the problem, often stated as a claim or a working thesis;
In this thesis we propose to use both the topological relationships of indoor environments, meaning the connectivity between distinctive places, and their metric characteristics, the geometry of the environment, to solve the map merging problem.  We do this by extracting hybrid 3D topometric maps from heterogeneous partial maps. We then use both the topological and metric characteristics of the partial maps to robustly identify overlapping areas that might appear differently.  Our hypothesis is that the connectivity of places within the environment are identifiable between heterogeneous partial maps. We further hypothesize that using the metric characteristics of the environment in conjunction with its topological characteristics will improve identification of overlapping areas over a purely topological approach, despite geometrical differences between heterogeneous partial maps. The most important contributions of our work are: 1) applying 3D topometric map merging to indoor environments. 2) extracting 3D topometric maps from heterogeneous partial maps.