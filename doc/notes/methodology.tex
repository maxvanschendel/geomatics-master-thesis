\section{Methodology}
In this section we will describe our methodology for solving the map merging problem. We divide our methodology into three major components: map extraction, map matching and map fusion, which correspond with the three research subquestions. This section follows this division, with an added subsection describing the different map representations that we use. Figure \ref{fig:flowchart_complete} shows the steps of our methodology. Refer to the specific subsections for each step for a further description of the algorithms and notation. We will now give a short summary of each of the steps of our methodology.

\paragraph{Map extraction}
For each input partial map, a point cloud, we create a voxel grid with a given cell size. Within these voxel grids we detect which voxels could feasibly be used to navigate (walk) through the environment. Using this information we segment the voxel grids into submaps which closely match a human interpretation of how an indoor environment can be divided into rooms. We do so by finding areas with many common viewpoints. By combining the room submaps with the navigable voxels we can extract the environment's topological graph. We then create a topometric map for each input point cloud by merging the topological graph with the segmented voxel grid into a single map.

\paragraph{Map matching}
In the second part of our methodology, map matching, we identify matches between the rooms of the partial topometric maps with the purpose of detecting overlapping areas. We do this by first generating a global descriptor for each room that captures its geometric features and those of its context, the rooms that lie within a number of steps in the topological graph. We then find a matching by growing multiple matching hypotheses along the topological graphs in a constrained manner and selecting the one that contains the largest number of matches.

\paragraph{Map fusion}
In the final part of our methodology, map fusion, we find the transformation that aligns the partial maps' geometry and use it to create a single, global topometric map. We do so by finding the optimal transformation between each pair of matched rooms. We then cluster the transformations based on similarity and select the cluster whose mean transformation best aligns the partial maps as the most likely correct transformation. After using the mean transformation to align the geometry of the partial maps into a global voxel grid map we extract a topometric map from it to create the global topometric map.

\afterpage{
    \begin{figure}[h]
        \centering
        \includegraphics*[width=.8\textwidth]{./fig/flowchart_complete-All.pdf}
        \caption{Diagram showing overview of methodology.}
        \label{fig:flowchart_complete}
    
    \end{figure}
    \clearpage
}


\pagebreak
\subsection{Map Representations}
In this section we will give a description of the different kinds of map representation that are used in this research, their mathematical notation, and the operations that we perform on them.

\subsubsection{Point Cloud}
An unordered collection of points representing the geometry of an object or environment in 3D euclidean space \citep{volodine_point_2007}.

\begin{equation}
\mathcal{P}=\{p_i\}_{i=1}^n, p_i \in \mathbb{R}^3
\end{equation}

Where \(\mathcal{P}\) denotes the point cloud and \(n\) the number of points that it contains.

\subsubsection{Voxel Grid}
A voxel is the 3D equivalent of a pixel. A voxel represents a single cell in a bounded 3D volume divided into a regular voxel grid. A voxel may contain information about whether it is occupied, what color it is, or any other property. A voxel can be represented by a three-dimensional vector containing its coordinates along the x, y and z axes of the voxel grid, as shown in equation \ref{eq:voxel_coords}. 

\begin{equation}
    \label{eq:voxel_coords}
    \boldsymbol{v} = (x, y, z) \in \mathbb{Z}^{3+}
\end{equation}

We define a \gls{voxelgrid} as a set of voxels with an associated edge length \(e_{l}\), as shown in equation \ref{eq:voxelgrid}. Figure \ref{fig:vg_basic} shows an example voxel grid and its components.

\begin{equation}
    \label{eq:voxelgrid}
    \mathcal{V}=\{\voxel_{i}\}_{i=1}^{n},\ n \in [1, \prod{\boldsymbol{v_{V_{max}}}}]
\end{equation}

To generate a voxel grid we divide a 3D axis-aligned volume \(V\), defined by minimum and maximum bounds \(V_{min}, V_{max} \in \mathbb{R}^{3}\) into a grid of cubic cells with edges of length \(e_{l}\). A voxel represents a subvolume of \(V\) bounded by a single cell. A voxel coordinate only consists of integer values that represent multiples of the edge length along each dimension. Voxel \(\boldsymbol{v_{V_{min}}} = (0,0,0)\) represents the first cell along each of the voxel grid's axes and the minimum of the volume's bounds, voxel represents \((0,1,1)\) the first cell along the x and the second along the y and z axes, etc. We also restrict voxel coordinates to only be positive as negative coordinates would fall outside of the bounds of the volume. For the same reason a voxel's coordinates can not be larger than that of the voxel representing the volume's maximum bounds \(\boldsymbol{v_{V_{max}}} = (V_{max} - V_{min})\ //\ e_l\), where \(//\) denotes integer division. 


\begin{equation}
    \label{eq:vmin}
\boldsymbol{v}_{min} = V_{min} + \boldsymbol{v}*e_{l}
\end{equation}
\begin{equation}
    \label{eq:vmax}
\boldsymbol{v}_{max} = \boldsymbol{v}_{min} + e_{l}
\end{equation}

The minimum and maximum bounds of this subvolume are given by equation \ref{eq:vmin} and \ref{eq:vmax}. The voxel's centroid is given by equation \ref{eq:vc}. Given a point \(\boldsymbol{p}\) within the bounds of \(V\), the corresponding voxel is given by equation \ref{eq:vp} . 
\begin{equation}
    \label{eq:vc}
    \boldsymbol{v_c} = (\boldsymbol{v_{min}} + \boldsymbol{v_{max}})*0.5
\end{equation}
\begin{equation}
    \label{eq:vp}
    \boldsymbol{v_p} = (\boldsymbol{p} - V_{min})\ //\ e_l
\end{equation}


\begin{figure}[h]
    \centering
    \includegraphics*[width=\textwidth]{./fig/voxel_basics.pdf}
    \caption{A voxel grid and its components.}
    \label{fig:vg_basic}
\end{figure}

A property of a voxel is given by equation \ref{eq:vproperty}. We define a property as an \(m\) by \(n\) real matrix for the purposes of this research. However, in reality a property could be any value. For example, we denote the property \textit{occupied}, which can be either 1 or 0, of a voxel \(\voxel\) as \(V_{occupied}(\voxel) = \voxel_{occupied} = \begin{pmatrix} 1 \end{pmatrix} | \begin{pmatrix} 0 \end{pmatrix}\)

\begin{equation}
    \label{eq:vproperty}
    \mathcal{V}_{property}: \mathbb{Z}^{3+} \mapsto \mathbb{R}^{m*n},\ \mathcal{V}_{property}(\boldsymbol{v}) = \boldsymbol{v}_{property}
\end{equation}


\paragraph{Sparse Voxel Octree}
Several operations on voxel grids benefit from using a spatial index, including radius searching and level of detail generation. We use a data structure called a sparse voxel octree (SVO) to achieve this. A normal octree recursively subdivides a volume into 8 cells, called octants. This operation results in a tree data structure, with nodes representing octants at a certain level of subsidivision. The root node of the tree structure represents the entire volume while the leaf nodes represent batches of 1 or more data points. In the case of a sparse voxel octree the leaf nodes represent individual voxels, with only the octants containing an occupied voxel represented in the tree. 

\begin{figure}[h]
    \centering
    \includegraphics*[width=.7\textwidth]{./fig/morton_code.pdf}
    \caption{2D example of morton codes.}
    \label{fig:vg_morton}
\end{figure}

To generate the SVO we first create a Morton order for the voxel grid. A Morton order maps the three-dimensional coordinates of the voxels to one dimension while preserving locality. It does by interleaving the binary representation of the voxel's coordinates into a single binary string which is interpreted as a positive integer, a Morton code. The ascending sorted vector of Morton codes gives us the Morton order. We divide the Morton order into buckets with size 8, such that each bucket contains at most 8 Morton codes, with a maximum difference of 8. Each non-empty bucket represents a parent node of at most 8 child nodes in the octree. By recursively performing this step until only one bucket remains, the root node, we can construct an SVO.

\begin{figure}[h]
    \centering
    \includegraphics*[width=.5\textwidth]{./fig/svo.pdf}
    \caption{Example of a sparse voxel octree generated from the above Morton codes.}
    \label{fig:vg_svo}
\end{figure}

We denote the function that returns all \(n\) voxels within range \(r\) of a voxel as follows.

\begin{equation}
    radius: \mathbb{Z}^{3+},\ \mathbb{R} \mapsto \voxelset
\end{equation}

The voxels within a sphere around a point can be found by recursively intersecting the sphere with the octants of the SVO. If the sphere does not intersect with an octant then none of its leaf nodes do and the corresponding voxels are not within the sphere. If an octant does intersect with the sphere then its children are tested for intersection. The algorithm returns all leaf nodes that intersect with the sphere.

\paragraph{Voxel convolution}
Voxel convolution involves moving a sliding window, or kernel, over each voxel in the grid to retrieve its neighbourhood and then computing a new value for the voxel based on the weighted sum of its neighbours. We can define a kernel \(\mathcal{K}\) as a voxel grid with an associated weight for each voxel and an origin voxel.

\begin{equation}
weight: \mathbb{Z}^{3+} \mapsto \mathbb{R}
\end{equation}

\begin{equation}
\boldsymbol{o}_{\mathcal{K}} \in \mathbb{Z}^{3}
\end{equation}

To apply a kernel to a voxel we first translate the kernel so that its origin lies on the voxel.

\begin{equation}
    \label{eq:kv}
\mathcal{K}_t = \{\boldsymbol{v_{\mathcal{K}}} + (\boldsymbol{v} - \boldsymbol{o_{\mathcal{K}}})\ |\ \boldsymbol{v_{\mathcal{K}}} \in \mathcal{K}\}
\end{equation}

We then get the property which we wish to convolve of each neighbour, multiply it by the neighbour's weight and sum it.

\begin{equation}
    \label{eq:kw}
    \mathcal{K}_{property}(\boldsymbol{v}) = \sum \{weight(\boldsymbol{v})\mathcal{V}_{property}(\boldsymbol{v})\ |\ \boldsymbol{v} \in \mathcal{K}_{t} \cap \mathcal{V}\}
\end{equation}

We denote the convolution of a property of every voxel in \(\mathcal{V}\) with \(\mathcal{K}\) as follows.

\begin{equation}
    \label{eq:c}
    \mathcal{V}_{property,\  \mathcal{K}} = \mathcal{V} * \mathcal{K}_{property} = \{\mathcal{K}_{property}(\boldsymbol{v}) \mid \boldsymbol{v} \in \mathcal{V}\}
\end{equation}


\paragraph{Neighbourhood graph}

The neighbourhood graph of a voxel grid represents the connectivity between voxels as undirected, unweighted graph. The nodes of the neighbourhood graph correspond to individual voxels and the edges to whether two voxels can be considered neighbours. Whether two voxels are neighbours is defined by a kernel which has only 1 or 0-valued weights. If, when applying the kernel to a voxel, another voxel within that kernel is occupied and the kernel's weight for that position is 1 then the two voxels are neighbours. The neighbourhood graph allows us to perform graph operations, such as identifying connected components, on voxel grids. Figure \ref{fig:vg_nbs} shows two commonly used kernels for constructing neighbourhood graphs, the Von Neumann and Moore neighbourhoods, also respectively known as the 6-neighbourhood and the 26-neighbourhood.

\begin{figure}[h]
    \centering
    \includegraphics*[width=.7\textwidth]{./fig/voxel_neighbourhood.pdf}
    \caption{Two common connectivity kernels. Note that the center voxel's weight is 0, as a voxel does not neighbour with itself.}
    \label{fig:vg_nbs}
\end{figure}

\newpage


\subsubsection{Topological map}
\Gls{topologicalmap} are a graph representation of an environment's structure, where vertices represent locally distinctive places, often rooms, and edges represent traversable paths between them  (see figure \ref{fig:topo_ex}) \citep{thrun_learning_1998,kuipers_robust_1988}. Topological maps are based on observations that cognitive maps, the mental maps used by humans to navigate within an environment, consist of multiple layers with a topological description of the environment being a fundamental component \citep{kuipers_robust_1988,kuipers_modeling_1978}. 

We denote a topological map as shown in equations \ref{eq:G}, \ref{eq:V} and \ref{eq:E}. The topological map consists of a graph \(G\), where nodes \(N\) represent distinctive places \(n\) and edges \(E\) represent the presence of a navigable path between neighbouring pairs of places \((v_j,v_k)\) that does not pass through any other places. Whether a path is navigable depends on who or what is navigating. For the purpose of this thesis a navigable path is a path that can be reasonably used by humans to walk from one room to another. Following this definition, only a part of the environment can be used as a navigable path. This includes the parts of the floor that are at sufficient distance from a wall or ceiling, stairs, and ramps.


\begin{equation}
    \label{eq:G}
    G=(N, E)
\end{equation}
\begin{equation}
    \label{eq:V}
    N=\{n_i\}_{i=1}^k
\end{equation}
\begin{equation}
    \label{eq:E}
    E=\{(n_j,n_k)_i\}_{i=1}^m,\ n_j \in N,\ n_k \in N,\ n_j \neq n_k
\end{equation}

Figure \ref{fig:topomap} shows an example topological map of a house with five rooms and their connectivity.

\begin{figure}[h]
    \centering
    \includegraphics*[width=.3\textwidth]{./fig/topological_map.pdf}
    \caption{Example of a topological map.}
    \label{fig:topomap}
\end{figure}

\pagebreak

\subsubsection{Topometric map}
A hybrid map representation combining both the topological and geometric characteristics of the environment. This map representation allows the end-user to use either topological or metric information depending on the needs of the situation, e.g. the topological layer can be used for large-scale navigation and abstract reasoning while the metric layer can be used for place recognition and obstacle avoidance. In the context of this thesis a topometric map refers to a graph representation of an indoor environment where the nodes represent rooms and their associated geometry as a voxel grid and the edges represent the navigability relationship between them. It is thus a hybrid representation of the environment that combines the properties of the voxel grid and the topological map which we described above.We denote a topometric map \(\topometricmap\) as shown in equation \ref{eq:topometricmap}, where \(\voxelgrid\) represents the complete geometry of the environment and \(G\) the topological graph.

\begin{equation}
    \label{eq:topometricmap}
    \topometricmap = (\voxelgrid,\ G),\ \voxelgrid=\{\voxel_{i}\}_{i=1}^{n}
\end{equation}

The topological graph, which we denote as in \ref{eq:place_subset}, consists of a set of nodes \(N\) and a set of edges \(E\). Each node \(n \in N\) represents a room and contains a subset of \(\voxelgrid\) that describes the geometry of that room. The nodes' subsets of \(\voxelgrid\) are not allowed to overlap, which means they represent a segmentation of \(\voxelgrid\).

\begin{equation}
    \label{eq:place_subset}
    G=(N,\ E),\ N=\{n_i\}_{i=1}^k,\ n \subset \voxelgrid
\end{equation}

Figure \ref{fig:topometricmap} shows an example of a topometric map of an indoor environment.

\begin{figure}[h]
    \centering
    \includegraphics*[width=.7\textwidth]{./fig/area_1_topo_01.png}
    \caption{Example of a topometric map.}
    \label{fig:topometricmap}
\end{figure}



\pagebreak
\subsection{Map Extraction}

The goal of the topometric map extraction step is to transform a partial voxel grid map of an indoor environment, denoted by \(\voxelgrid\), into a topometric map, denoted by \(\topometricmap\). The algorithm should be robust the missing data and work across a wide range of data resolution. In this section we propose an algorithm to achieve this goal. In an overview, it works as follows.

We first extract a navigation graph \(\navgraph\), a graph connecting all of the voxels which a hypothetical agent could use to move through the environment, from the partial voxel grid map \(\voxelgrid\). Using \(\navgraph\) we compute points where the hypothetical agent would have an optimal view of the environment. We then find the visible voxels for each of these points. By clustering the resultant visibilities based on similarity, we segment \(\voxelgrid\) into submaps that align closely with how humans divide indoor environments into rooms. As such, we refer to the submaps of \(\voxelgrid\) when segmented using visibility clustering as 'rooms'. Afterwards, we construct the topological graph of the environment, denoted by \(\topologicalgraph\), with edges representing whether it is possible to navigate between two rooms without passing through another. Finally, we fuse the topological map with the segmented map to construct the topometric map \(\topometricmap\). 

Figure \ref{fig:map_extract_steps} shows an overview of our map extraction algorithm, its input, and intermediate outputs. In the rest of this subsection we will the algorithm in more detail.

\begin{figure}[h]
    \centering
    \includegraphics*[width=\textwidth]{./fig/map_extract.png}
    \caption{Diagram showing map extraction processes and intermediate data.}
    \label{fig:map_extract_steps}
\end{figure}

\subsubsection{Navigable volume}
In our first step we extract a navigation graph \(\navgraph\). The navigation graph tells us how a theoretical agent in the map's environment would navigate through that environment. In practice, assuming a human agent, this means the areas of the floor, ramps and stairs that are at a sufficient distance from a wall and a ceiling. We compute \(\navgraph\) using a three step algorithm which we describe below.

\subsubsection{Convolution}
The first step uses voxel convolution with a stick-shaped kernel \(\mathcal{K}_{stick}\) (shown in \ref{fig:stick_kernel}) to find all voxels that are unobstructed and are thus candidates for navigation. Each voxel in the kernel has a weight of 1, except the origin voxel which has weight 0. The associated function is summation. Convolving the voxel grid's occupancy property with the stick kernel gives us each voxel's obstruction property, denoted as \(\voxel_{obstruction}\) with an obstruction value of 0 when no other voxels are present in the stick kernel. This indicates that these voxels have enough space around and above them to be used for navigation. We then filter out all obstructed voxels, such that \(\mathcal{V}_{unobstructed}=\{\boldsymbol{v} \mid \voxel \in convolve(\mathcal{V},\ \mathcal{K}_{stick}),\ \voxel_{obstruction} = 0\}\). 

\begin{figure}[h]
    \centering
    \includegraphics*[width=0.5\textwidth]{./fig/structuring_element.png}
    \caption{Illustration of stick kernel with top and side views.}
    \label{fig:stick_kernel}
\end{figure}

\subsubsection{Dilation}
The next step of the algorithm is to dilate the unobstructed voxels upwards by a distance \(d_{dilate}\). This connects the unoccupied voxels separated by a small height differences into a connected volume. The value of \(d_{dilate}\) depends on the expected differences in height between the steps of stairs in the environment. Typically, we use a value of 0.2m for this parameter. The result is a new voxel grid \(\mathcal{V}_{dilated}\). Note that the dilation step may create new occupied voxels that are not in the original voxel grid, which means that \(\mathcal{V}_{dilated}\) is not a subset of \(\voxelgrid\).

\subsubsection{Connected components}
The final step of the algorithm is to split \(\mathcal{V}_{dilated}\) into one or more connected components. A connected component \(\mathcal{V}_c\) of a voxel grid is a subset of \(\mathcal{V}\) where there exists a path between every voxel in \(\mathcal{V}_c\). The neighbourhood graph \(\mathcal{G}_{\mathcal{V},\ \mathcal{K}} = (V,\ E)\) of \(\mathcal{V}\) represents the voxels in \(\mathcal{V}\) as nodes. Each node has incident edges towards all other voxels in its neighbourhood, which is defined by a kernel \(\mathcal{K}\), such that \(V = \mathcal{V},\ E_{V} = \{(\boldsymbol{v}, \boldsymbol{v_{nb}})\ |\ \boldsymbol{v} \in \mathcal{V},\ \boldsymbol{v_{nb}} \in \mathcal{K}_{\boldsymbol{v}}\},\ |E_{V}| \leq |\mathcal{K}|*|\mathcal{V}|\). 

There exists a path between two voxels \(\boldsymbol{v_a},\ \boldsymbol{v_b}\) in \(\mathcal{V}\) if there exists a path between their corresponding nodes \(V_a,\ V_b\) in \(\mathcal{G}_{\mathcal{V}}\). We denote the set of all possible paths between two nodes as \(paths(V_a,\ V_b)\). A connected component is then defined as \(\mathcal{V}_c=\{\boldsymbol{v_a}\ |\ \boldsymbol{v_a} \in \mathcal{V},\ \boldsymbol{v_b} \in \mathcal{V},\ |paths(V_a, V_b)| \neq 0\}\). We define the set of all connected components as \(\mathcal{V}_{cc}=\{\mathcal{V}_{c,\ i}\}_{i=1}^n\). We find the connected components using the following algorithm:

\begin{algorithm}
    \caption{Region growing connected components}\label{alg:cap}
    \begin{algorithmic}

    \Require \quad Voxel grid \(\mathcal{V}\)
    \Require \quad Connectivity kernel \(\mathcal{K}\)
    \Ensure \quad Connected components \(\mathcal{V}_{cc}\)

    \State \(\mathcal{G}_{\mathcal{V},\ \mathcal{K}} = (V,\ E),\ V \in \mathcal{V}\) \Comment{Convert voxel grid to neighbourhood graph using specified kernel}
    \State \(V_{unvisited} = V\)
    \State \(\mathcal{V}_{cc} = \{\}\)
    
    \State \(i = 0\)
    \While{$|V_{visited}| \neq 0$}
        \State Select random node \(n\) from \(V_{unvisited}\)
        \State \(n_{BFS} = BFS(n)\) \Comment{Use breadth-first search to find all nodes connected to \(n\)}
        \State Remove \(n_{BFS}\) from \(V_{unvisited}\)
        \State \(\mathcal{G}_i = (n_{BFS},\ E)\)
        \State Add \(\mathcal{V}_{c,\ i}\) to \(\mathcal{V}_{cc}\)
        \State \(i = i + 1\)
    \EndWhile
    \end{algorithmic}
\end{algorithm}

After doing so, we find the connected component in \(\mathcal{V}_{cc}\) with the largest amount of voxels \(\mathcal{V}_{max}\), which corresponds to the voxels used for navigation. The navigation graph \(\navgraph\) is the neighbourhood graph of \(\mathcal{V}_{max}\). We denote the intersubsection of the navigation graph's nodes and the whole voxel grid map as \(\mathcal{V}_{navigation} = \mathcal{V}_{max} \cap \mathcal{V}\).

\subsubsection{Maximum visibility estimation}
To compute the isovists necessary for room segmentation it is first necessary to estimate hypothetical scanning positions that maximize the view of the map. We compute these by finding the local maxima of the horizontal distance field of \(\mathcal{V}_{navigation}\). The steps to achieve this are as follows. 

\subsubsection{Horizontal distance field}
For each voxel in \(\voxelgrid\) we compute the horizontal Manhattan distance to the nearest boundary voxel. A boundary voxel is a voxel for which not every voxel in its Von Neumann neighbourhood is occupied. To compute this value we iteratively convolve the voxel grid with a circle-shaped kernel on the X-Z plane, where the radius of the circle is expanded by 1 voxel with each iteration, starting with a radius of 1. When the number of voxel neighbours within the kernel is less than the number of voxels in the kernel a boundary voxel has been reached. Thus, the number of radius expansions tells us the Manhattan distance to the boundary of a particular voxel. We denote the horizontal distance of a voxel to its boundary as \(dist: \mathbb{Z}^{3+} \mapsto \mathbb{Z}\). Computing the horizontal distance for every voxel in \(\voxelgrid\) gives us the horizontal distance field (HDF), such that \(HDF = \{dist(\voxel) \mid \voxel \in \voxelgrid\}\). We denote the horizontal distance of a given voxel \(\voxel\) as \(d_{\voxel}\). We implement this using the following algorithm.

\algnewcommand\algorithmicforeach{\textbf{for each}}
\algdef{S}[FOR]{ForEach}[1]{\algorithmicforeach\ #1\ \algorithmicdo}

\begin{algorithm}
    \caption{Horizontal distance field}
    \begin{algorithmic}

    \Require \quad Navigation voxel grid \(\mathcal{V}_{navigation}\)
    \Ensure \quad Horizontal distance field \(HDF = \{dist(\voxel) \mid \voxel \in \voxelgrid_{navigation}\}\)

    \\

    \ForEach {$\boldsymbol{v} \in \mathcal \voxelgrid_{navigation} $}
        \State \(r=1\)
        \State Create \(\mathcal{K}_{circle}\) with radius \(r\)
        \While{$convolve(\boldsymbol{v},\ \mathcal{K}_{circle}) = |\mathcal{K}_{circle}|)$}
            \State \(r = r+1\)
            \State Expand \(\mathcal{K}_{circle}\) with new radius \(r\)
        \EndWhile

        \State \(HDF = HDF \cup {r}\) \Comment{Add voxel's radius to horizontal distance field}
    \EndFor
    \end{algorithmic}
\end{algorithm}

We then find the maxima of the horizontal distance field within a given radius \(r \in \mathbb{R}\).  The local maxima of the horizontal distance field are all voxels that have a larger or equal horizontal distance than all voxels within \(r\), such that \(HDF_{max} = \{\voxel \mid dist(\voxel) \geq max \{dist(\boldsymbol{v_{r}} \mid \boldsymbol{v_{r}} \in radius(\voxel,\ r))\}\}\). Increasing the value of \(r\) reduces the number of local maxima and vice versa. All voxels in \(HDF_{max}\) lie within the geometry of the environment, which means the view of the environment is blocked by the surrounding voxels. To solve this, we take the centroids of the voxels in \(HDF_{max}\) and translate them upwards to a reasonable scanning height \(h\), to estimate the positions with the optimal view of the map. We denote these positions as \(views = \{\boldsymbol{v_c} + (0, h, 0) \mid \boldsymbol{v} \in HDF_{max}\}\). We use the following algorithm to compute \(views\).

\begin{algorithm}
    \caption{Horizontal Distance Field Maxima}
    \begin{algorithmic}

    \Require \quad Navigation voxel grid \(\mathcal{V}_{navigation}\)
    \Require \quad Horizontal distance field \(HDF\)
    \Require \quad Radius \(r \in \mathbb{R}^+\)
    \Require \quad Scanning height \(h \in \mathbb{R}\)

    \Ensure \quad Estimated optimal views $views \in \mathbb{R}^3$

    \\

    \State $views = \{\}$

    \ForEach {$\boldsymbol{v} \in \mathcal \voxelgrid_{navigation} $}
        \State \(neighbourhood = radius(\boldsymbol{v},\ r)\)
        \If{$d_{\voxel} \geq \max{\{d_{nb} \mid neighbourhood\}}$} \Comment{Check if voxel's horizontal distance is equal or greater than the horizontal distance of every voxel in its neighbourhood}
            \State $views = views \cup \{centroid(\voxel) + (0, h, 0)\}$
        \EndIf
    \EndFor
    
    \end{algorithmic}
\end{algorithm}

\begin{figure}[h]
    \centering
    \includegraphics*[width=0.8\textwidth]{./fig/hdf_simple.png}
    \caption{Illustration of horizontal distance field computation and extraction of local maxima.}
    \label{fig:hdf_simple}
\end{figure}

\begin{figure}[h]
    \centering
    \includegraphics*[width=0.7\textwidth]{./fig/horizontal_distance_field.png}
    \caption{Resulting horizontal distance field of partial map, with resultant optimal view points shown in red.}
    \label{fig:hdf}
\end{figure}

\subsubsection{Visibility}
The next step in the room segmentation algorithm is to compute the visibility from each position in \(views\). We denote the all voxels that are visible from a given position as \(visibility: \mathbb{R},\ \voxelset \mapsto \mathbb{Z}^{m \times 3},\ m \in \mathbb{R},\ n \geq m\). A target voxel is visible from a position if a ray cast from the position towards the centroid of the voxel does not intersect with any other voxel. To compute this we use the digital differential analyzer (DDA) algorithm to rasterize the ray onto the voxel grid in 3D. We then check if any of the voxels that the ray enters- except the target voxel- is occupied. If none are, the target voxel is visible from the point. We perform this raycasting operation from every position in \(views\) towards every voxel within a radius \(r_{visibility}\) of that position. Only taking into account voxels within a radius speeds up the visibility computation, and is justifiable based on the fact that real-world 3D scanners have limited range. We denote the set of visibilities from each point in views as \(visibility_{views} = \{visibility(\boldsymbol{x}) \mid \boldsymbol{x} \in views\}\). The below pseudocode shows how the visibility computation works.

\begin{algorithm}
    \caption{Visibility}
    \begin{algorithmic}

    \Require \quad Voxel grid \(\voxelgrid\)
    \Require \quad Origin \(o \in \mathbf{R}^3\)
    \Require \quad Range \(r_{visibility} \in \mathbb{R}^+\)
    \Ensure \quad $visibility \in \mathbb{Z}^{3}$

    \State $visibility = \{\voxel_c \mid \voxel_c \in radius(o, r_{visibility}) \land DDA(\voxelgrid,\ o,\ centroid(\voxel_c)) = \voxel_c\}$

    \end{algorithmic}
\end{algorithm}

% https://math.stackexchange.com/questions/83990/line-and-plane-intersubsection-in-3d
\begin{algorithm}
    \caption{DDA (Digital Differential Analyzer)}
    \begin{algorithmic}

    \Require \quad Voxel grid \(\voxelgrid\)
    \Require \quad Ray origin \(\boldsymbol{o} \in \mathbf{R}^3, \boldsymbol{o} \in [\voxelgrid_{min}, \voxelgrid_{max}]\)
    \Require \quad Ray target \(\boldsymbol{t} \in \mathbf{R}^3\)

    \Ensure \quad $hit \in \mathbb{Z}^{3}$ \Comment{First encountered collision}

    \State $\boldsymbol{p}_{current} = \boldsymbol{o}$
    \State $\voxel_{\boldsymbol{o}} = (\boldsymbol{o} - \voxelgrid_{min})//\voxelgrid_{e}$
    \State $\voxel_{current} = \voxel_{\boldsymbol{o}}$

    \State $\boldsymbol{d} = (\boldsymbol{t} - \boldsymbol{o})$ 
    \State $heading = \boldsymbol{d} \odot abs(\boldsymbol{d})^{-1}$ \Comment{Determine if ray points in positive or negative direction for every axis}

    \While{$(\voxel_{current} \notin \voxelgrid \lor \voxel_{current} = \voxel_{\boldsymbol{o}}) \land \boldsymbol{p}_{current} \in [\voxelgrid_{min}, \voxelgrid_{max}]$}
        \State $\boldsymbol{c} = centroid(\voxel_{current})$
        \State $d_{planes} = \boldsymbol{c} + heading*\voxelgrid_{e}/2$

        \State $d_{min} = \infty$
        \State $axis=1$

        \ForEach($d \in d_{planes}$)
            \State $t = \frac{d - \boldsymbol{n} \cdot \boldsymbol{p}_{current}}{\boldsymbol{n} \cdot (\boldsymbol{t} - \boldsymbol{p}_{current})}$
            \State $\boldsymbol{i} = \boldsymbol{p}_{current} + t(\boldsymbol{t} - \boldsymbol{o})$

            \If{$d_{min} \geq t$}
                \State $\boldsymbol{p}_{current} = \boldsymbol{i}$
                \State $\voxel_{current,\ axis} += heading_{axis}$
            \EndIf

            \State $axis=axis+1$
        \EndFor

    \State $hit = \voxel_{current}$
    \EndWhile

\end{algorithmic}
\end{algorithm}

\subsubsection{Room segmentation}
After computing the set of visibilities from the estimated optimal views we apply clustering to group the visibilities by similarity. This is based on the definition of a room as a region of similar visibility. Remember that each visibility is a subset of the voxel grid map. To compute the similarity of two sets we use the Jaccard index, which is given by \(J(A,B) = \frac{|A \cap B|}{|A \cup B|}\). Computing the Jaccard index for every combination of visibilities gives us a similarity matrix \(S^{n \times n} \in [0, 1]\). The similarity matrix is symmetric because \(J(A,B) = J(B,A)\). Its diagonals are 1, as \(J(A,A) = 1\). An example similarity matrix is showin in figure \ref{fig:jaccard}.

\begin{figure}
    \centering
    \includegraphics*[width=0.8\textwidth]{./fig/mutual_visibility_matrix.png}
    \caption{Similarity matrix extracted from set of visibilities. Each value represents the Jaccard index of two visibilities.}
    \label{fig:jaccard}
\end{figure}

We can also consider \(S^{n \times n}\) as an undirected weighted graph \(\graph_S\), where every node represents a visibility and the edges the Jaccard index of two visibilities. This means we can treat visibility clustering as a weighted graph clustering problem. To solve this problem we used the Markov Cluster (MCL) algorithm (CITE MCL), which has been shown by previous research to be state-of-the-art for visibility clustering. 


The main parameter of the MCL algorithm is inflation. By varying this parameter between an approximate range of \([1.2, 2.5]\) we get different clustering results. We find the optimal value for inflation within this range by maximizing the clustering's modularity. This value indicates the difference between the fraction of edges within a given cluster and the expected number of edges for that cluster if edges are randomly distributed. 

% TODO: Give mathematical description of MCL
% TODO: Give mathematical equation of modularity

We denote the clustering of \(visibility_{views}\) that results from the MCL algorithm as \(\mathbf{C}_{visibility} = \{c_i\}_{i=1}^{|views|},\ c_i \in \mathbb{Z}^+\), where the \(i\)th element of \(\mathbf{C}_{visibility}\) is the cluster that the \(i\)th element of \(visibility_{views}\) belongs to, such that for a given value of \(c\), the elements in \(visibility_{views}\) for which the corresponding \(c\) in \(\mathbf{C}_{visibility}\) has the same value belong to the same cluster. As each visibility is a subset of the map, each cluster of visibilities is also a subset of the map. We denote the union of the visibilities belonging to each cluster as \(\mathcal{V}_{c}\). 

It is possible for visibility clusters in \(\mathcal{V}_{c}\) to have overlapping voxels. This means that each voxel in the partial map may have multiple associated visibility clusters. However, the goal is to assign a single room to each voxel in the map. To solve this we assign to each voxel the cluster in which the most visibilities contain that voxel. The result is a mapping from voxels to visibility clusters (which we will from now on refer to as rooms), which we denote as \(room: \voxel \mapsto \integers\), such that \(room(\voxel) = c,\ c \in \mathbf{C}_{visibility},\ \voxel \in \voxelgrid\). This often results in noisy results, with small, disconnected islands of rooms surrounded by other rooms. Intuitively, this does not correspond to a reasonable room segmentation. To solve this, we apply a label propagation algorithm. This means that for every voxel we find the voxels within a neighbourhood as defined by a convolution kernel. We then assign to the voxel the most common label, in this case the room, of its neighbourhood, if that label is more common than the current label. We iteratively apply this step until the assigned labels stop changing. Depending on the size of the convolution kernel the results are smoothed and small islands are absorbed by the surrounding rooms.


\begin{algorithm}
    \caption{Label propagation}
    \begin{algorithmic}

    \Require \quad Voxel grid \(\voxelgrid\)
    \Require \quad Initial labeling \(label^{(0)}: \mathbb{Z}^{3+} \mapsto \mathbb{Z}\)
    \Require \quad Kernel \(\mathcal{K}\)
    \Ensure \quad Propagated labeling after \(t\) steps \(label^{(t)}: \mathbb{Z}^{3+} \mapsto \mathbb{Z}\)

    \State $t=0$

    \While($label^{(t)} \neq label^{(t+1)}$) \Comment{Keep iterating until labels stop changing}
        \ForEach($\voxel \in \voxelgrid$)
            \State $L = \{label^{(t)}(\voxel_{nb}) \mid \voxel_{nb} \in neighbours(\voxel, \mathcal{K})\}$4

            \State $l_{max} = \mathop{argmax}_{l} \ |\{l \mid l \in L\}|$ \Comment{Most common label in neighbourhood}
            \State $l_{current} = label^{(t)}(\voxel)$  \Comment{Label of current voxel}

            \If{$|\{l \mid l \in L \land l=l_{max}\}| > |\{l \mid l \in L \land l=l_{current}|$}
                \State \(label^{(t+1)}(\voxel) = l_{max}\)
            \Else
                \State \(label^{(t+1)}(\voxel) = label^{(t)}(\voxel)\)
            \EndIf
        \EndFor

        \State $t = t+1$ \Comment{Use propagated labeling as input for next iteration}
    \EndWhile
    \end{algorithmic}
\end{algorithm}

\subsubsection{Topometric map extraction}
The above steps segment the map into multiple non-overlapping rooms based on visibility clustering. In the next step we transform the map into a topometric representation \(\topometricmap = (\topologicalgraph, \voxelgrid)\), which consists of a topological graph \(\topologicalgraph=(V,E)\) and a voxel grid map \(\voxelgrid\). Each node in \(\topologicalgraph\) represents a room, and also has an associated voxel grid which is a subset of \(\voxelgrid\) and represents the geometry of that room. Edges in \(\topologicalgraph\) represent navigability between rooms, meaning that there is a path between them on the navigable volume that does not pass through any other rooms. This means that for two rooms to have a navigable relationship they need to have adjacent voxels that are both in the navigable volume. To construct the topometric map we thus add a node for every room in the segmented map with its associated voxels, we then add edges between every pair of nodes that satisfy the above navigability requirement.  


\begin{algorithm}
    \caption{Topology extraction}
    \begin{algorithmic}

    \Require \quad Voxel grid \(\voxelgrid\)
    \Require \quad Voxel grid navigation subset \(\voxelgrid_{navigation}\)
    \Require \quad Room segmentation \(room: \voxel \mapsto \integers\)
    \Require \quad Adjacency kernel $\mathcal{K}_{adjacency}$

    \Ensure \quad Topological spatial graph \(\widetilde{\mathcal{G}}_{topology} = (V_{topology},\ E_{topology})\)
    \Ensure \quad Node embedding \(f_{node}: V \mapsto \mathbb{R}^3\)

    \State $\widetilde{\mathcal{G}}_{topology} = (\{\},\ \{\})$

    \State Get each unique room label \(\mathbf{R} = \{room(\voxel) \mid \voxel \in \voxelgrid\}\)
    \State Split \(\voxelgrid\) by label, such that \(\mathbf{V} = \{\voxelgrid \cap \{\voxel \mid \voxel \in \voxelgrid \land room(\voxel) = r\} \mid r \in \mathbf{R}\}\)
    \State Store room label associated with each voxel grid \(room_{\voxelgrid}: r \mapsto \voxelgrid\)
    \State $V_{topology} = \mathbf{V}$
    \State $f_{node}(v) = centroid(v),\ v \in V_{topology}$

    \ForEach($\voxel \in \voxelgrid_{navigation}$)
        \State $v_r = room(\voxel)$
        \State $nbs_r = \{room(nb) \mid nb \in neighbourhood(\voxel,\ \mathcal{K}_{adjacency})\}$

        \State $r_{adjacency} = \{(r_a,\ r_b) \mid (r_a,\ r_b) \in {v_r} \times {nbs_r} \land r_a \neq r_b\}$
        \State $E_{topology} = E_{topology} \cup \{(room_{\voxelgrid}(r_a),\ room_{\voxelgrid}(r_b)) \mid (r_a,\ r_b) \in r_{adjacency}\}$
    \EndFor
    \end{algorithmic}
\end{algorithm}

\pagebreak
\subsection{Map Matching}

\subsubsection{Overview}
The process of identifying overlapping areas between partial maps is called map matching. In the case of topometric map matching, this refers to identifying which nodes represent the same rooms between two partial maps. We denote our two partial topometric maps as

\begin{equation}
    \label{eq:tmap_a}
    \topometricmap_a = (\mathcal{G}_a,\ \voxelgrid_a),\ \mathcal{G}_a=(N_a,E_a)
\end{equation}

\begin{equation}
    \label{eq:tmap_b}
    \topometricmap_b = (\mathcal{G}_b,\ \voxelgrid_b),\ \mathcal{G}_b=(N_b,E_b)
\end{equation}

The goal of map matching is to find a one-to-one mapping between the rooms of both partial maps which corresponds to the real world and is robust to differences in coordinate system, resolution and quality between partial maps. 

To identify matches between rooms we need to be able to compute their similarity. To do so, we first transform each room into a descriptor, an n-dimensional vector, which represents both the geometry of the room. The descriptor of two nodes with similar geometry should be close to eachother in feature space, meaning the distance between their vectors should be small. Conversely, the descriptors of two dissimilar rooms should be far away from eachother in feature space. 

We then use the topological properties of the topometric maps to improve map matching in two ways. The first is contextual embedding. This means that we combine the descriptor of each room with the descriptor of its neighbourhood in the topological graph. This improves matching because multiple rooms may have similar geometry but not necessarily similar neighbourhoods. The second is hypothesis growing, which means that we grow multiple matching hypotheses along the topological graph in a constrained manner and only use the hypothesis that contains the most matches. 

Figure \ref{fig:flowchart_match} shows an overview of the steps described above. In the rest of this section we will describe the aforementioned steps in depth.

\begin{figure}[h]
    \centering
    \includegraphics*[width=0.5\textwidth]{./fig/flowchart_match.pdf}
    \caption{Diagram showing map matching methodology.}
    \label{fig:flowchart_match}
\end{figure}


\subsubsection{Geometric descriptor}
Geometric feature embedding transforms a geometric object, in our case a voxel grid, into an m-dimensional descriptor \(f_{geometry}\), such that objects with similar geometry are nearby in feature space and vice versa.

\begin{equation}
    \label{eq:embed_geometry_01}
    embed_{geometry}: \voxelset \mapsto \mathbb{R}^m,\ embed_{geometry}(n) = \prescript{n}{}{f_{geometry}}
\end{equation}

We denote the function that embeds a set of voxels into a feature vector as in equation \ref{eq:embed_geometry_01}. We implement this function using two different approaches, which we discuss below.

\subsubsection{Spectral Features}
Our first approach to geometric feature embedding uses spectral shape analysis. This approach uses the first \(n\) sorted, non-zero eigenvalues of the graph Laplacian, in our case of the neighbourhood graph of a room's geometry, as a geometric descriptor. To compute this we first convert each room's neighbourhood graph \(\mathcal{G}\) to an adjacency matrix \(A\) and a degree matrix \(D\). We then find the Laplacian matrix of the neighbourhood graph by subtracting its adjacency matrix from its degree matrix, as shown in equation \ref{eq:laplacian_matrix}.

\begin{equation}
    \label{eq:laplacian_matrix}
L = D - A
\end{equation}

After computing the Laplacian matrix we find its eigenvalues, sort them in ascending order and use the first 256 non-zero values as the descriptor. 


\subsubsection{Deep Learning}
Our second approach to geometric feature embedding uses deep learning. Specifically, we use the LPDNet neural network architecture. This architecture is used for place recognition, it does so by learning descriptors, typically 2048 or 4096-dimensional, of point clouds that are theoretically independent of transformation, perspective and completeness. It does so by computing a local descriptor for every point in the point cloud and aggregrating them into a global descriptor. The LPDNet model we use is trained on outdoor maps which have different characteristics from indoor maps. However, the authors of LPDNet claim that a model trained on outdoor data can also effectively be used for indoor data. Figure \ref{fig:lpdnet_architecture} shows the network architecture of LPDNet.

% network architecture w/figure
% training data / transfer learning

\begin{figure}[h]
    \centering
    \includegraphics*[width=\textwidth]{./fig/network_architecture.png}
    \caption{Diagram showing LPDNet network architecture.}
    \label{fig:lpdnet_architecture}
\end{figure}

\subsubsection{Contextual Embedding}
After computing a descriptor for each individual room we augment them by taking into account the descriptor of the neighbourhood. For every room we find their neighbours and merge their geometry into one voxel grid, for which we compute a new descriptor. We do this step multiple times for neighbours that are at most one or multiple steps away from the room. We then append the descriptors of the neighbourhood to the room's descriptor. By doing so we can distinguish between rooms with similar geometry but dissimilar neighbourhoods, which are often present in indoor environments.

\subsubsection{Initial Matching}
The above steps are applied to both partial maps. This gives us two sets of descriptors \(\mathcal{E}_A,\ \mathcal{E}_B\) representing the embedding of the geometry of the rooms of both topometric maps.

To identify the most likely overlapping rooms between the partial maps we find the one-to-one mapping between the elements of \(\mathcal{E}_A\) and \(\mathcal{E}_B\) that maximizes the similarity (or minimizes the distances) between the chosen pairs. This is an example of the unbalanced assignment problem, which consists of finding a matching in a weighted bipartite graph that minimizes the sum of its edge weights. It is unbalanced because there may be more nodes in one part of the bipartite graph than the other, which means it is not possible to assign every node in one part to a node in the other. 

To construct the weighted bipartite graph we first find the Cartesian product of the feature vectors 

\begin{equation}
    \label{eq:E_ab}
    \mathcal{E}_{AB} = \mathcal{E}_A \times \mathcal{E}_B = \{(a,b) \mid a \in \mathcal{E}_A,\ b \in \mathcal{E}_B\}
\end{equation}

We then compute the Euclidean distance in feature space between every pair of nodes in \(\mathcal{E}_{AB}\), creating the cost matrix that represents the weighted bipartite graph 

\begin{equation}
    \label{eq:C}
    \mathbf{C} \in \mathbb{R}^{|V_a| \times |V_b|}
\end{equation}
\begin{equation}
    \label{eq:C}
    \mathbf{C}_{ij} = ||a - b||,\ (a,\ b) \in \mathcal{E}_{AB},\ a = \mathcal{E}_{A,\ i},\ b = \mathcal{E}_{B,\ j},\ \mathbf{C}_{ij} \in \mathbb{R}^+
\end{equation}

We can then find unbalanced assignment using the Jonker-Volgenant algorithm. We denote the resulting matching between the nodes of both partial maps and their distance in feature space as a set of triples 

\begin{equation}
    \label{eq:M}
    \mathbf{M} = \{(i,\ j,\ d) \mid  |V_a| \geq i,\ |V_b|\geq j,\ d = \mathbf{C}_{ij}\}
\end{equation}

\subsubsection{Hypothesis growing}

In practice it is unlikely that every match in \(\mathbf{M}\) is correct. However, we can use them as seeds to generate hypotheses similar to the approach described in Huang \citep{huang_topological_2005}. Starting at each initial match we get the neighbourhood of both nodes. We then construct a new cost matrix from the Euclidean distance between the embeddings of both neighbourhoods, again creating a weighted bipartite graph for which we can solve the assignment problem. By doing this we identify which neighbours of the nodes in the match are most likely to also match. We recursively apply this step to the matching neighbours to grow our initial matches into hypotheses. To decrease the risk of incorrectly identifying neighbourhood matches we constrain hypothesis growing in two ways. First, the cost of two potential matches must be below a given threshold \(c_{max}\). Second, a newly identified match may not bring the existing matching too much out of alignment. To check this, we perform a registration (see next section) between the centroids of the geometry of the identified matches at every step of the hypothesis growing. If the error increases between steps, and the increase is too large such that \(\triangle e \geq \triangle e_{max} \), then the matching is rejected. By adjusting the values of \(c_{max}\) and \(\triangle e_{max}\) more or less uncertainty is allowed when growing hypotheses.

\pagebreak
\subsection{Map Fusion}

\subsubsection{Overview}
The final step of the map merging process is map fusion, in general this means the problem of combining multiple partial maps into a global map. In our case, it specifically refers to the fusion of two partial topometric maps at the geometric and topological level to produce a global topometric map. 

To achieve geometric fusion we designate one partial map as the source and the other as the target and find a transformation that aligns the two maps. We constrain the transformation to a rotation around the y-axis and a translation because most 3D scans of indoor environments are gravity-aligned, which means that it is not necessary to consider rotations around the x- or z-axis. 

To find the transformation between the partial maps we first find the transformation between each pair of matched rooms. We do so by using RANSAC to find a global alignment which we then refine using the iterative closest point algorithm. Afterwards, we cluster the transformations and find the mean transformation of each cluster. We use the mean transformation which leads to the smallest difference between partial maps as our final rigid transformation. We then apply this transformation to the source map and fuse the geometry of the two maps into a global voxel grid map. Finally, we extract a new, global topometric map from the fused geometry. Figure \ref{fig:flowchart_fuse} shows an overview of our map fusion approach. In the rest of this section we will describe our map fusion approach in detail.

\afterpage{
    \begin{figure}[h]
        \centering
        \includegraphics*[width=.9\textwidth]{./fig/flowchart_fuse.pdf}
        \caption{Diagram showing overview of map fusion methodology.}
        \label{fig:flowchart_fuse}
    \end{figure}
}

\subsubsection{Registration}

The goal of registration is to find a rigid transformation \(\tau\) between two point clouds that minimizes the error, as defined by an error function \(e\), between them. For the error function we use point-to-plane distance, as shown in equation \ref{eq:point2plane}. In the context of indoor mapping data is usually aligned to gravity, with the direction of gravity being equal to the direction of the negative y-axis. We take this into account by constraining the transformation between partial maps to a translation along all three axes and a rotation around the y-axis. The rigid transformation can thus be expressed as a 4-dimensional vector, as shown in equation \ref{eq:tau}. Reducing the degrees of freedom of the problem from 6 to 4 can improve the alignment.


\begin{equation}
    \label{eq:tau}
    \tau = \underset{\tau}{argmin} \ e (\mathcal{P},\mathcal{Q}) = \begin{bmatrix}
        \gamma \\
        \bf{t}
    \end{bmatrix} = \begin{bmatrix}
        \gamma \\
        t_x    \\
        t_y    \\
        t_z
    \end{bmatrix}
\end{equation}

\begin{equation}
    \label{eq:point2plane}
    e = \sum_{k=1}^{K} || ((R(\gamma)\bf{p_k} + \bf{t}) - \bf{q_k})  \cdot \bf{n_k}||,\ \bf{p_k} \in \mathcal{P},\ \bf{q_k} \in \mathcal{Q}
\end{equation}

\cite{kubelka_gravity-constrained_2022} gives a method for restating gravity-constrained alignment between two sets of points as a system of equations, which is shown in equations \ref{eq:ck} and \ref{eq:at_b} (adapted to use y-axis instead of z-axis as the gravity direction). We can then solve this system of equations using least squares adjustment to find the optimal transformation \(\tau\).

\begin{equation}
    \label{eq:ck}
    c_k = (\begin{bmatrix}
        0  & 0 & 1 \\
        0  & 0 & 0 \\
        -1 & 0 & 0 \\
    \end{bmatrix} \bf{p_k}) \cdot \bf{n_k}
\end{equation}

\begin{equation}
    \label{eq:at_b}
    \sum_{k=1}^{K} \begin{bmatrix}
        c_k \\
        n_k
    \end{bmatrix}
    \begin{bmatrix}
        c_k & n_k
    \end{bmatrix}
    \tau = \sum_{k=1}^{K} \begin{bmatrix}
        c_k \\
        n_k
    \end{bmatrix} (\bf{d_k} \cdot \bf{n_k}) \rightarrow A \tau = \bf{b}
\end{equation}

\begin{equation}
    transform: (\mathbb{R}^{m \times 3},\ \mathbb{R}^{m \times 3}) \rightarrow \mathbb{R}^4,\ transform(\mathcal{P},\ \mathcal{Q}) = \tau
\end{equation}

The above is able to align two point clouds optimally if the correspondence between points is known exactly, meaning that the \(k\)-th point of both \(\mathcal{P}\) and \(\mathcal{Q}\) correspond to the same point in the real world. This is, however, usually not the case in real world scenarios. This means that registration requires another two steps, global and local registration, which we will discuss below.

\paragraph{Global registration}
The purpose of global registration is to find a rough alignment between point clouds, which can then be further refined in the local registration step. To do so we use a RANSAC based approach. This algorithm works as follows. For every point in the point clouds \(\mathcal{P}\) and \(\mathcal{Q}\) compute a feature embedding that can be used to find similar points in the other point clouds. We use Fast Point Feature Histograms (FPFH) features, which are commonly used for this purpose. We then randomly select 3 points from \(\mathcal{P}\) and find the corresponding points in \(\mathcal{Q}\) which have the smallest distance in feature space. We then compare the triangles formed by both sets of points. If the edge lengths of both triangles are too dissimilar, meaning that the ratio of their lengths is outside of a predetermined range, then the selected points are discarded and new ones are selected. If not, we find the gravity-constrained rigid transformation between the 3 pairs. We then evaluate how well the rigid transformation aligns the two point clouds by finding the mean distance of every point in \(\mathcal{P}\) to its nearest neighbour in \(\mathcal{Q}\). If the mean distance is smaller than the previous smallest mean distance then we store the transformation.
We repeat this for a set amount of iterations or until a mean distance threshold is reached. The rigid transformation with the smallest mean distance is then used for the next step, local registration.


\paragraph{Local registration}
The purpose of local registration is to refine the alignment found in the global registration step. We use the iterative closest point (ICP) algorithm to achieve this. The ICP algorithm is widely used, a detailed description of how it works can be found in (CITATION HERE). The only major modification is that we use the gravity-aligned transformation with the point-to-plane error function described above at each iteration.


\paragraph{Transform concatenation}
After finding the global and the local transformations we can find the final transformation \(\tau\) by multiplying their by 4 transformation matrices. We denote the function that converts \(\tau\) to a 4 by 4 transformation matrix as in equation \ref{eq:4by4}. Equation \ref{eq:trans_mult} shows how to compute the final transformation matrix, note that the order of multiplication is important, as matrix multiplication is not commutative.

\begin{equation}
    \label{eq:4by4}
    T: \mathbb{R}^4 \rightarrow \mathbb{R}^{4 \ times 4}
\end{equation}

\begin{equation}
    \label{eq:trans_mult}
    T(\tau) = T(\tau_{local})T(\tau_{global}),\ T: \mathbb{R}^4 \rightarrow \mathbb{R}^{4 \times 4}
\end{equation}

\subsubsection{Geometric fusion}
We apply the above steps to the geometry of each room and its match if their distance in feature space is below a threshold. This gives us a 4-dimensional vector representing the transformation between matches. We discard the matches where the registration error is too high and cluster the remaining transformations using the DBSCAN algorithm. This gives us multiple clusters of transformations that, within a cluster, result in a similar alignment between partial maps. For each of these clusters we find the mean transformation, apply it to the partial map and compute the point-to-plane error between the partial maps. We then select the cluster with the lowest error and use its mean transformation to align the geometry of the partial topometric maps. 

\subsubsection{Topological fusion}
After the geometric fusion step the geometry of the topometric maps is brought into alignment but their graphs haven't been fused yet. Fusing the graphs directly using the identified matches is possible but does not guarantuee a good global topometric map. This is because the geometric fusion may have changed the partial maps in a way that the global geometric map's topology is greater than the sum of the partial maps' topology. As a result, we have to reextract the topology from the results of the geometric fusion. To do so, we reuse the navigation graphs of the partial maps as the navigation graph of the global map. We also reuse the optimal viewpoints from both partial maps. This means that only the room segmentation step needs to be redone. The result is a global topometric map created from the fusion of two partial topometric maps.

