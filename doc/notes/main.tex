% !TeX root = main.tex
\documentclass{article}
\usepackage[english]{babel}
\usepackage[utf8x]{inputenc}
\usepackage{graphicx}
\usepackage{apacite}
\usepackage{natbib}
\usepackage{amsfonts} 
\usepackage{amsmath}
\usepackage{mathtools}
\usepackage{bm} 
\usepackage{algpseudocode}
\usepackage{algorithm}
\usepackage{glossaries}
\usepackage{afterpage}

\makenoidxglossaries

\newglossaryentry{map}
{
    name=map,
    description={A symbolic representation of an environment which contains information about its characteristics.}
}

\newglossaryentry{maprepresentation}
{
    name=map representation,
    description={The choice of characteristics of the environment that a map shows. Hybrid map representations are representations that shows a combination of multiple characteristics of the environment, e.g. a hybrid topological-metric map containing both the environment's large-scale structure and small-scale geometry.}
}

\newglossaryentry{agent}
{
    name=agent,
    description={According to \citet{russell_artificial_2010} "An agent is anything that can be viewed as perceiving its environment through sensors and
    acting upon that environment through actuators.". In the context of this thesis an agent specifically refers to any human or robot that is capable of creating a 3D map of an environment and.}
}

\newglossaryentry{partialmap}
{
    name=partial map,
    description={A collection of maps without a common coordinate frame that each represent a part of the environment.}
}

\newglossaryentry{globalmap}
{
    name=global map,
    description={A single, more complete map constructed by merging multiple partial maps.}
}

\newglossaryentry{mapmerging}
{
    name=map merging,
    description={The problem of identifying overlapping areas between partial maps and using these to combine the partial maps into a global map.}
}

\newglossaryentry{mapmatching}
{
    name=map matching,
    description={The problem of identifying overlapping areas between partial maps.}
}

\newglossaryentry{mapfusion}
{
    name=map fusion,
    description={The problem of combining multiple partial maps into a single global map based on their overlapping areas.}
}

\newglossaryentry{mapextraction}
{
    name=map extraction,
    description={The problem of converting one map representation to another. For the purposes of this research this mostly refers to the extraction of topometric maps from voxel grids.}
}


\newglossaryentry{localdescriptor}
{
    name=local descriptor,
    description={An n-dimensional vector representing the properties of a single point in a point cloud that can be used to find corresponding points between point clouds.}
}

\newglossaryentry{globaldescriptor}
{
    name=global descriptor,
    description={An n-dimensional vector representing the properties of a whole point cloud which can be used to compare the similarity of point clouds.}
}


\newglossaryentry{topologicalmap}
{
    name=topological map,
    description={Topological maps are a qualitative graph representation of an environment's structure, where vertices represent locally distinctive places, often rooms, and edges represent traversable paths between them  \citep{thrun_learning_1998,kuipers_robust_1988}. Topological maps are inspired by the fact that humans are capable of spatial learning despite limited sensory and processing capability and only having partial knowledge of the environment. This is based on observations that cognitive maps, the mental maps used by humans to navigate within an environment, consist of multiple layers with a topological description of the environment being a fundamental component \citep{kuipers_robust_1988}.}
}

\newglossaryentry{pointcloud}
{
    name=point cloud,
    description={An unordered collection of points representing the geometry of an object or environment in 3D euclidean space \citep{volodine_point_2007}.}
}

\newglossaryentry{voxelgrid}
{
    name=voxel grid,
    description={Also known as an occupancy grid, a voxel grid is a "multi-dimensional (typically 2D or 3D) tesselation of space into cells, where each cell stores a probabilistic estimate of its state." \citep{elfes_occupancy_1990}.}
}

\newglossaryentry{topometricmap}
{
    name=topometric map,
    description={A hybrid map representation combining both the topological and metric characteristics of the environment. This map representation allows the end-user to use either topological or metric information depending on the needs of the situation, e.g. the topological layer can be used for large-scale navigation and abstract reasoning while the metric layer can be used for landmark detection or obstacle avoidance.}
}

\newglossaryentry{registration}
{
    name=registration,
    description={The problem of finding a transformation that optimally aligns two point clouds.}
}

\title{Extraction and Merging of Topometric Maps of Indoor Environments}

\author{
  Maximiliaan van Schendel\\
  student \#4384644 \\
  \url{m.vanschendel@tudelft.nl}\\
  \\
  1st supervisor: Edward Verbree \\
  2nd supervisor: Pirouz Nourian \\
  external supervisor: Robert Voûte \\
}

\date{30/09/2022}
\begin{document}
\pagenumbering{gobble}

\newcommand{\voxel}{\boldsymbol{v}}
\newcommand{\voxelgrid}{\mathcal{V}}
\newcommand{\voxelset}{\mathbb{Z}^{n \times 3}}
\newcommand{\graph}{\mathcal{G}}
\newcommand{\integers}{\mathbb{Z}}

\newcommand{\topometricmap}{\mathcal{T}}
\newcommand{\topologicalgraph}{}
\include{glossary.tex}
\maketitle
\pagebreak

\tableofcontents
\newpage

\pagenumbering{arabic}

% abstract

\paragraph{abstract}
The merging of multiple partial \gls{map}s of indoor environments created by teams of human or robot agents into a single global map is a key problem that, when solved, can improve mapping speed and quality. Existing map merging approaches generally depend on external signals which are not available indoors or only use the geometric properties of an environment. Inspired by the human understanding of environments in relationship to their context we propose a map merging system that extracts and uses topometric maps, a map representation containing both the geometric and topological characteristics of an environments, to solve the map merging problem in indoor spaces. In this research we demonstrate an intuitive approach to extracting topometric maps of 3D, multi-floor, indoor environments and use both the topological and geometric characteristics contained in the topometric maps to perform context-aware map matching and fusion. 


\clearpage
\printnoidxglossaries

%TODO: MODIFY INTRO TO WORK WITH CHANGE OF SCOPE AND CONTENTS

\section{Introduction}
Collaborative mapping allows multiple agents to work together to create a single, global map of their environment. By working together large areas can be mapped in a short amount of time. Collaborative mapping of indoor environments is especially challenging, because external positioning signals are highly attenuated. In this case, a global map can only be created by merging the partial maps of the environment created by each agent individually based on their overlapping areas. This is called map merging (see figure \ref{fig:map_merging}). Partial maps might represent the environment at different a different scale, resolution or accuracy, which further complicates map merging as overlapping areas may not appear the same between partial maps.

\begin{figure}[h]
    \centering
    \includegraphics*[width=\textwidth]{./fig/map_merging.png}
    \caption{Partial maps (1) captured by three different agents and resultant global map (2) after map merging.}
    \label{fig:map_merging}
\end{figure}

% the method of solving the problem, often stated as a claim or a working thesis;
In this thesis we propose to use both the topological relationships of indoor environments, meaning the connectivity between distinctive places, and their metric characteristics, the geometry of the environment, to solve the map merging problem.  We do this by extracting hybrid 3D topometric maps from heterogeneous partial maps. We then use both the topological and metric characteristics of the partial maps to robustly identify overlapping areas that might appear differently.  Our hypothesis is that the connectivity of places within the environment are identifiable between heterogeneous partial maps. We further hypothesize that using the metric characteristics of the environment in conjunction with its topological characteristics will improve identification of overlapping areas over a purely topological approach, despite geometrical differences between heterogeneous partial maps. The most important contributions of our work are: 1) applying 3D topometric map merging to indoor environments. 2) extracting 3D topometric maps from heterogeneous partial maps.
% research background
\section{Research Questions}

\subsection{Main question}
How can we apply topometric representations of indoor environments to solve the map merging problem?

\subsection{Subquestions}
\begin{enumerate}
    \item In what way can partial topometric maps be extracted from partial point cloud maps?
    \item What approach is best suited for identifying matches between partial topometric maps?
    \item How can the identified matches be used to fuse two or more partial topometric maps into a global topometric map?
\end{enumerate}

\subsection{Scope}
To better delineate the scope of the thesis we provide several aspects that will \textbf{not} be researched or discussed. 

\begin{enumerate}
    \item Map merging using known relative poses between agents or meeting strategies. Agent behaviour is assumed to be independent and agents are not able to sense eachother.
    \item Map merging using observations unrelated to the environment's geometrical and topological characteristics. E.g. the environment's colour or actively transmitted beacon signals.
    \item Map merging assisted by a priori knowledge of the environment. E.g. building information models (BIM) or floor plans.
    \item Map merging using the pose graphs of agents. Agent poses are assumed to be unknown.
    \item Achieving (near) real-time performance.
\end{enumerate}
\pagebreak
\section{Related work}
Previous research on mapping and map merging has used various map representations are used depending on the map's intended purpose \citep{tomatis_hybrid_2003,huang_topological_2005,bonanni_3-d_2017,gholamishahbandi_2d_2019}. According to \citet{andersone_heterogeneous_2019} and \citet{yu_review_2020} these representations can be subdivided into three types: metric-, feature-, and topological maps. Hybrid maps that are combinations of two or more map types also exist, such as topological-metric maps \citep{yu_review_2020}. Map types that are not one of the three main types or a hybrid are rarely used \citep{yu_review_2020}. In this section we will discuss the characteristics of the metric, topological and topological-metric map representations and the work that has been done on extracting and merging them.

\subsection{Metric Maps}
\label{section:metric_map_merging}
Metric map merging is a mature area of research and thus various approaches have been proposed, many of which use a variation of the Iterative Closest Point (ICP) algorithm. This algorithm finds the transformation between partial maps by iteratively applying rigid transformations that minimize the distance between point-pairs \citep{rusinkiewicz_efficient_2001}. Since its first introduction a number of variations on the ICP algorithm have been proposed that improve its accuracy and performance. An example of this is the NICP algorithm, which improves data-association by taking into account the normal vector of a point's neighbourhood \citep{serafin_nicp_2015}.  

Another group of approaches to metric map merging use feature matching to identify overlaps between maps. These approaches try to extract distinctive features in the metric map, such as corners, lines, planes or other points of interest, e.g. SIFT, SURF or Harris points \citep{andersone_heterogeneous_2019}. Feature matching approaches have the advantage over ICP-based approaches with regards to required computational power \citep{andersone_heterogeneous_2019}. Some feature matching approaches are better suited for different environments and input data. For example, the feature extractor approach by \citet{li_general_2010} is scale-independent and the approach of \citet{yang_fast_2016} is able to deal with differences in resolution. \citet{yang_fast_2016} also proposes a combination of feature-based and ICP-based metric map merging that uses features to find a rough alignment which is then further refined using a variation on the ICP algorithm, as shown in figure \ref{fig:feature}.

\begin{figure}
    \centering
    \includegraphics*[width=\textwidth]{./fig/feature_matching.png}
    \caption{Illustration of combined feature-based and ICP metric map merging approach from \citet{yang_fast_2016}.}
    \label{fig:feature}
\end{figure}

In conclusion, much work has been done on the merging of metric maps using brute-force ICP-based methods. However, these methods have several downsides and are thus losing popularity in favour of feature-based approaches. Research on many different kinds of features is available, with state-of-the-art approaches being able to handle large differences in appearance of features. Both approaches can also be combined to compensate for their respective downsides.
                        
\subsection{Topological(-Metric) Maps}
Various approaches for extracting topological maps from raw sensor data or metric maps have been proposed. \citet{kuipers_robot_1991} proposes identifying distinctive places, vertices of the topological map, directly from sensor data by finding local maxima of a distinctiveness measure within a neighbourhood. Edges are identified by having the robot try to move between vertices, if this is possible an edge is created. Local metric information in the form of an occupancy grid is associated with the nearest vertex in the graph, resulting in a hybrid topological-metric map. Note that this approach is dependent on the mapping agent's exploration strategy. As heterogeneous agents can't be assumed to behave in a similar way, approaches based on exploration strategy are not directly applicable for the purposes of this thesis. 

\citet{thrun_learning_1998} extracts a 2D topological map from a 2D metric map by identifying narrow passages with the use of a Voronoi diagram. They then partition the metric map into areas divided by passages, which are respectively the vertices and the edges of the graph. Again, metric information is associated to create a hybrid topological-metric map. This is the first approach that is independent of exploration strategy. 


\citet{ochmann_towards_2014} describes extracting hierarchical topological-metric maps directly from point clouds. In the case of \citet{ochmann_towards_2014} the hierachy is divided into four encapsulating layers, building - storey - room - object. Entities within the graph are represented as vertices, with edges representing the topological and spatial relationships between entities. Each vertex is linked to a local metric map of the entity's geometry. All entities are also linked to the layer above them that they are in, e.g. objects are linked to their encompassing room, which is in turn linked to the storey it is in. 

\citet{gorte_navigation_2019} provides a novel approach for extracting the walkable floor space from a 3D occupancy grid across multiple storeys. They do so by first applying a 3D convolution filter using a stick-shaped structuring element to extract the parts of the floor without obstructions. They then apply an upwards dilation to connect steps of stairways into a connected volume. Because the topology of the environment depends on the traversability between spaces, the extraction of navigable floor space is essential for the extraction of topological maps.

\citet{he_hierarchical_2021} describes an approach for extracting a hierarchical topological-metric map with three layers: storey - region - volume, from an occupancy grid map. To extract the map they use a novel approach to room segmentation using raycasting. A downside of their methodology is that it depends on the presence of ceilings in the metric map, which are often not captured in practice when using handheld scanners. They also describe a novel approach for storey segmentation using a peak detector on the metric map's histogram of z-coordinates. The results of their approach are shown in figure \ref{fig:topometric_map}. \\\\
In comparison, relatively little work has been done on the subject of topological(-metric) map merging. \citet{dudek_topological_1998} first proposes an approach to topological map merging which depends on a robot meeting strategy to merge partial maps created by each robot. When new distinctive places are recognized at the frontier of the global map the other robots will travel towards it and synchronize their maps. As with other early approaches to map extraction and map merging, this one also depends on a coordinated exploration strategy, making it unsuitable for the purposes of this thesis.

The work of \citet{huang_topological_2005} is a significant milestone in topological map merging, demonstrating that topological maps can be merged using both map structure and map geometry. They first identify vertex matches by comparing the similarity of their attributes, such as their degree, and the spatial relationships of incident edges. Vertex matches are then expanded using a region growing approach, where every added edge and vertex is compared for similarity and rejected if too dissimilar. The results are multiple hypotheses for overlapping areas between partial maps. They then estimate a rigid transformation between the partial maps based on each hypothesis. Afterwards, hypotheses that result in similar transformations are grouped into hypothesis clusters. They then select the most appropriate hypothesis cluster by using a heuristic that includes the number of vertices in the cluster, the error between matched vertices after transformation and the number of hypotheses in the cluster.

\citet{bonanni_3-d_2017} provides a unique approach to topological-metric map merging using the pose graph of mapping agents as the topological component and the point cloud captured at each node as the metric component. Matches between nodes are identified by computing the similarity of their associated point cloud. In comparison to most other map merging approaches they fuse the maps using a non-rigid transformation, meaning the partial maps are deformed to improve their alignment. This gives their approach the ability to correct inconsistencies between partial maps that might be caused by differences between scanning agents.

To conclude, a variety of representations of topological and topological-metric maps have been proposed, as well as ways of extracting them from metric maps. Early approaches to topological and topological-metric map extraction depend on coordinated robot exploration strategies and only work in 2D, making them unsuitable for the purposes of this thesis. State-of-the art approaches are trending towards 3D hierarchical representations of buildings as technology improves and more ways of dealing with 3-dimensional data are discovered. However, little work has been done on extracting topological maps from heterogeneous metric maps, as well as on extracting them from incomplete partial maps. 
Relative to the amount of work on the extraction of topological(-metric) maps of indoor environments, less has been done on merging them. Most approaches compare labels of vertices and their incident edges to find matches. To our knowledge, \citet{bonanni_3-d_2017} is the only work that identifies vertex matching by comparing local metric map geometry. It is also the only approach that is able to handle deformed partial maps. 








\pagebreak
\section{Methodology}


\subsection{Map Representations}
In this section we will give a description of the different kinds of map representation that are used in this research, their mathematical notation, and the operations that we perform on them.

\subsubsection{Point Cloud}
An unordered collection of points representing the geometry of an object or environment in 3D euclidean space \citep{volodine_point_2007}.

\begin{equation}
\mathcal{P}=\{p_i\}_{i=1}^m, p_i \in \mathbb{R}^3
\end{equation}

Where \(\mathcal{P}\) denotes the point cloud and \(n\) the number of points that it contains.

\subsubsection{Voxel Grid}
A voxel is the 3D equivalent of a pixel. A voxel represents a single cell in a bounded 3D volume divided into a regular voxel grid. A voxel represents information about its volume, such as whether it is occupied, what color it is, or any other property. Unless otherwise specified, voxels in this research represent whether a given volume is occupied by any kind of obstruction, such as an object or the environment. A voxel can be represented by a three-dimensional vector containing its coordinates along the x, y and z axes of the voxel grid, as shown in equation \ref{eq:voxel_coords}. 

\begin{equation}
    \label{eq:voxel_coords}
    \boldsymbol{v} = (x, y, z)^T \in \mathbb{N}^{3}
\end{equation}

We define a \gls{voxelgrid} as a set of voxels with an associated size \(e_{l}\), as shown in equation \ref{eq:voxelgrid}. Unoccupied voxels are not present in the set, making it a sparse representation. Figure \ref{fig:vg_basic} shows an example voxel grid and its components.

\begin{equation}
    \label{eq:voxelgrid}
    \mathcal{V}=\{\voxel_{i}\}_{i=1}^{n}
\end{equation}

To generate a voxel grid we divide a 3D axis-aligned volume bounding box  defined by minimum and maximum bounds \(\mathbf{b}_{min}, \mathbf{b}_{max} \in \mathbb{R}^{3}\) into a grid of cubic cells with distance \(e \in \mathbb{R}^+\) between their centers. A voxel represents a subvolume of the bounding box bounded by a single cell. A voxel coordinate only consists of integer values that represent the position of the voxel in the grid along each axis. Voxel \(\boldsymbol{u} = (0,0,0)^T\) represents the first cell along each of the voxel grid's axes and the minimum of the volume's bounds, voxel \((0,1,1)^T\) represents the first cell along the x- and the second along the y- and z-axes, etc. We also restrict voxel coordinates to only be positive as negative coordinates would fall outside of the bounds of the volume. For the same reason a voxel's coordinates can not be larger than that of the voxel representing the volume's maximum bounds \(\boldsymbol{w} = (\mathbf{b}_{max} - \mathbf{b}_{min})\ \rfloor \ e\), where \(\rfloor\) denotes floor division. 


\begin{equation}
    \label{eq:vmin}
\boldsymbol{v}_{min} = \mathbf{b}_{min} + \boldsymbol{v}e
\end{equation}
\begin{equation}
    \label{eq:vmax}
\boldsymbol{v}_{max} = \boldsymbol{v}_{min}+e
\end{equation}

The minimum and maximum bounds of this subvolume are given by equation \ref{eq:vmin} and \ref{eq:vmax}. The voxel's centroid is given by equation \ref{eq:vc}. Given a point \(\boldsymbol{p}\) within the bounds of \(V\), the corresponding voxel is given by equation \ref{eq:vp} . 
\begin{equation}
    \label{eq:vc}
    \boldsymbol{v_c} = (\boldsymbol{v_{min}} + 0.5\boldsymbol{v_{max}})
\end{equation}
\begin{equation}
    \label{eq:vp}
    \boldsymbol{v_p} = (\boldsymbol{p} - \mathbf{b}_{min})\ \rfloor\ e_l
\end{equation}


\begin{figure}[h]
    \centering
    \includegraphics*[width=\textwidth]{./fig/voxel_basics.pdf}
    \caption{A voxel grid and its components.}
    \label{fig:vg_basic}
\end{figure}


\paragraph{Sparse Voxel Octree}
Several operations on voxel grids benefit from using a spatial index, including radius searching and level of detail generation. We use a data structure called a sparse voxel octree (SVO) to achieve this. A normal octree recursively subdivides a volume into 8 cells, called octants. This operation results in a tree data structure, with nodes representing octants at a certain level of subsidivision. The root node of the tree structure represents the entire volume while the leaf nodes represent batches of 1 or more data points. In the case of a sparse voxel octree the leaf nodes represent individual voxels, with only the octants containing an occupied voxel represented in the tree. 

\begin{figure}[h]
    \centering
    \includegraphics*[width=.7\textwidth]{./fig/morton_code.pdf}
    \caption{2D example of morton codes.}
    \label{fig:vg_morton}
\end{figure}

To generate the SVO we first create a Morton order for the voxel grid, this is illustrated in figure \ref{fig:vg_morton}. A Morton order maps the three-dimensional coordinates of the voxels to one dimension while preserving locality. It does by interleaving the binary representation of the voxel's coordinates into a single binary string which is interpreted as a positive integer, a Morton code. The ascending sorted vector of Morton codes gives us the Morton order. We divide the Morton order into buckets with size 8, such that each bucket contains at most 8 Morton codes, with a maximum difference of 8. Each non-empty bucket represents a parent node of at most 8 child nodes in the octree. By recursively performing this step until only one bucket remains, the root node, we can construct an SVO. The SVO corresponding to the Morton code in figure \ref{fig:vg_morton} is shown in figure \ref{fig:vg_svo}.

\begin{figure}[h]
    \centering
    \includegraphics*[width=.5\textwidth]{./fig/svo.pdf}
    \caption{Example of a sparse voxel octree generated from the above Morton codes.}
    \label{fig:vg_svo}
\end{figure}

We denote the function that returns all \(n\) voxels within range \(r\) of a voxel as follows.

\begin{equation}
    radius: \mathbb{N}^{3},\ \mathbb{R} \mapsto \voxelset
\end{equation}

The voxels within a sphere around a point can be found by recursively intersecting the sphere with the octants of the SVO. If the sphere does not intersect with an octant then none of its leaf nodes do and the corresponding voxels are not within the sphere. If an octant does intersect with the sphere then its children are tested for intersection. The algorithm returns all leaf nodes that intersect with the sphere.

Levels of detail can be generated by using the occupied octants at the levels above leaf nodes as a simplified voxel grid. This can enable certain operations that are not computationally feasible at the original level of detail.

\paragraph{Voxel convolution}
Voxel convolution involves moving a sliding window, or kernel, over each voxel in the grid to retrieve its neighbourhood and then computing a new value for the voxel based on the weighted sum of its neighbours. We can define a kernel \(\mathcal{K}\) as a voxel grid with an associated weight for each voxel and an origin voxel \(\boldsymbol{o}_{\mathcal{K}}\).

\begin{equation}
weight: \mathbb{N}^{3} \mapsto \mathbb{R}
\end{equation}

\begin{equation}
\boldsymbol{o}_{\mathcal{K}} \in \mathbb{N}^{3}
\end{equation}

To apply a kernel to a voxel we first translate the kernel so that its origin lies on the voxel.

\begin{equation}
    \label{eq:kv}
\mathcal{K}_t = \{\boldsymbol{v_{\mathcal{K}}} + (\boldsymbol{v} - \boldsymbol{o_{\mathcal{K}}})\ |\ \boldsymbol{v_{\mathcal{K}}} \in \mathcal{K}\}
\end{equation}

We then get the property which we wish to convolve of each neighbour, multiply it by the neighbour's weight and sum it.

\begin{equation}
    \label{eq:kw}
    \mathcal{K}_{property}(\boldsymbol{v}) = \sum \{weight(\boldsymbol{v})\mathcal{V}_{property}(\boldsymbol{v})\ |\ \boldsymbol{v} \in \mathcal{K}_{t} \cap \mathcal{V}\}
\end{equation}

We denote the convolution of a property of every voxel in \(\mathcal{V}\) with \(\mathcal{K}\) as follows.

\begin{equation}
    \label{eq:c}
    \mathcal{V}_{property,\  \mathcal{K}} = \mathcal{V} * \mathcal{K}_{property} = \{\mathcal{K}_{property}(\boldsymbol{v}) \mid \boldsymbol{v} \in \mathcal{V}\}
\end{equation}


\paragraph{Neighbourhood graph}

The neighbourhood graph of a voxel grid represents the connectivity between voxels as undirected, unweighted graph. The nodes of the neighbourhood graph correspond to individual voxels and the edges to whether two voxels can be considered neighbours. Whether two voxels are neighbours is defined by a kernel which has only 1 or 0-valued weights. If, when applying the kernel to a voxel, another voxel within that kernel is occupied and the kernel's weight for that position is 1 then the two voxels are neighbours. The neighbourhood graph allows us to perform graph operations, such as identifying connected components, on voxel grids. Figure \ref{fig:vg_nbs} shows two commonly used kernels for constructing neighbourhood graphs, the Von Neumann and Moore neighbourhoods, also respectively known as the 6-neighbourhood and the 26-neighbourhood.

\begin{figure}[h]
    \centering
    \includegraphics*[width=.7\textwidth]{./fig/voxel_neighbourhood.pdf}
    \caption{Two common connectivity kernels. Note that the center voxel's weight is 0, as a voxel does not neighbour with itself.}
    \label{fig:vg_nbs}
\end{figure}

\newpage


\subsubsection{Topological map}
\Gls{topologicalmap}s are graph representations of an environment's structure, where nodes represent locally distinctive places and edges represent traversable paths between them  (see figure \ref{fig:topomap}) \citep{thrun_learning_1998,kuipers_robust_1988}. Topological maps are based on observations that cognitive maps, the mental maps used by humans to navigate within an environment, consist of multiple layers with a topological description of the environment being a fundamental component \citep{kuipers_robust_1988,kuipers_modeling_1978}. 

We denote a topological map as shown in equations \ref{eq:G}, \ref{eq:V} and \ref{eq:E}. The topological map consists of a graph \(G\), where nodes \(N\) represent distinctive places \(n_i\) and edges \(E\) represent the presence of a navigable path between neighbouring pairs of places \((n_j,n_k)\) that does not pass through any other places. Whether a path is navigable depends on who or what is navigating. For the purpose of this thesis a navigable path is a path that can be reasonably used by humans to walk from one room to another. Following this definition, only a part of the environment can be used as a navigable path. This includes the parts of the floor, stairs or ramps that are at sufficient distance from a wall, the ceiling or other obstructions.


\begin{equation}
    \label{eq:G}
    G=(N, E)
\end{equation}
\begin{equation}
    \label{eq:V}
    N=\{n_i\}_{i=1}^k
\end{equation}
\begin{equation}
    \label{eq:E}
    E=\{(n_j,n_k)_i\}_{i=1}^m,\ n_j \in N,\ n_k \in N,\ n_j \neq n_k\}
\end{equation}

Figure \ref{fig:topomap} shows an example topological map of a house with five rooms and their connectivity.

\begin{figure}[h]
    \centering
    \includegraphics*[width=.3\textwidth]{./fig/topological_map.pdf}
    \caption{Example of a topological map.}
    \label{fig:topomap}
\end{figure}

\pagebreak

\subsubsection{Topometric map}
A hybrid map representation combining both the topological and geometric characteristics of the environment. This map representation allows the end-user to use either topological or metric information depending on the needs of the situation, e.g. the topological layer can be used for large-scale navigation and abstract reasoning while the metric layer can be used for place recognition and obstacle avoidance. In the context of this thesis a topometric map refers to a graph representation of an indoor environment where the nodes represent rooms and their associated geometry as a voxel grid and the edges represent the navigability relationship between them. It is thus a hybrid representation of the environment that combines the properties of the voxel grid and the topological map which we described above. We denote a topometric map \(\topometricmap\) as shown in equation \ref{eq:topometricmap}, where \(\voxelgrid\) represents the complete geometry of the environment and \(G\) the topological graph.

\begin{equation}
    \label{eq:topometricmap}
    \topometricmap = (\voxelgrid,\ G),\ \voxelgrid=\{\voxel_{i}\}_{i=1}^{n}
\end{equation}

The topological graph, which we denote as shown in equation \ref{eq:place_subset}, consists of a set of nodes \(N\) and a set of edges \(E\). Each node \(n \in N\) represents a room and contains a subset of \(\voxelgrid\) that describes the geometry of that room. The nodes' subsets of \(\voxelgrid\) are not allowed to overlap, which means they represent a segmentation of \(\voxelgrid\).

\begin{equation}
    \label{eq:place_subset}
    G=(N,\ E),\ N=\{n_i\}_{i=1}^k,\ n \subset \voxelgrid
\end{equation}

Figure \ref{fig:topometricmap} shows an example of a topometric map of an indoor environment.

\begin{figure}[h]
    \centering
    \includegraphics*[width=.7\textwidth]{./fig/area_1_topo_01.png}
    \caption{Example of a topometric map.}
    \label{fig:topometricmap}
\end{figure}



\pagebreak
\subsection{Map Extraction}
The first step of our approach is topometric map extraction.
The purpose of this step is to transform a partial voxel grid map of an indoor environment, denoted by \(\voxelgrid\), into a topometric map, denoted by \(\topometricmap\). In this section we propose an algorithm to achieve this goal. In an overview, it works as follows.

\subsubsection{Overview}
We first extract a navigation graph \(\navgraph\), the neighbourhood graph of all voxels which a hypothetical human agent could use to move through the environment, from \(\voxelgrid\). Using \(\navgraph\) we compute points where the hypothetical agent would have an optimal view of the environment. We then find the visible voxels for each of these points. By clustering the resultant visibilities based on similarity we segment \(\voxelgrid\) into submaps that align closely with how humans may divide indoor environments into rooms. As such, we refer to the submaps of \(\voxelgrid\) when segmented using visibility clustering as 'rooms'. We then construct the topological graph of the environment by finding which rooms have adjacent voxels in \(\navgraph\). Finally, we fuse the topological graph with the segmented map to construct the topometric map \(\topometricmap\). 

Figure \ref{fig:map_extract_steps} shows an overview of our map extraction algorithm, its input, and intermediate outputs. In the rest of this subsection we will discuss the algorithm in detail.

\begin{figure}[h]
    \centering
    \includegraphics*[width=1\textwidth]{./fig/flowchart_complete-Map extract.drawio.pdf}
    \caption{Diagram showing map extraction processes and intermediate data.}
    \label{fig:map_extract_steps}
\end{figure}

\subsubsection{Navigation graph}
We first extract a navigation graph \(\navgraph\). The navigation graph is a connected graph which tells us how a theoretical agent in the environment could move through the environment from one point to another. In practice, assuming a human agent, this means the areas of the floor, ramps and stairs that are at a sufficient distance from a wall, the ceiling or any other obstruction. We compute \(\navgraph\) using a three step algorithm which we describe below.

\paragraph{Convolution}
The first step of navigation graph extraction uses voxel convolution with a stick-shaped kernel \(\mathcal{K}_{stick}\) (shown in figure \ref{fig:stick_kernel}) to find all voxels that are unobstructed and may thus be used to navigate the environment. This approach is based on \citet{gorte_navigation_2019}. Each voxel in the kernel has a weight of 1, except the origin voxel which has weight 0. Convolving the voxel grid's occupancy property with the stick kernel gives us each voxel's obstruction property, which has a value of 0 when no other voxels are present in the stick kernel. This indicates that these voxels have enough space around and above them to be used for navigation. We then filter out all obstructed voxels leaving only the voxels that could be used for navigation.

\begin{equation}
    \label{eq:convolve}
\mathcal{V}_{unobstructed}=\{\boldsymbol{v} \mid \voxel \in \mathcal{V} * \mathcal{K}_{stick},\ \mathcal{K}_{stick}(\voxel) = 0\}
\end{equation}

\begin{figure}[h]
    \centering
    \includegraphics*[width=0.5\textwidth]{./fig/structuring_element.drawio.pdf}
    \caption{Illustration of stick kernel with top and side views.}
    \label{fig:stick_kernel}
\end{figure}

\paragraph{Upwards dilation}
The next step of the algorithm is to dilate the unobstructed voxels upwards by a distance \(d_{dilate}\). This connects the voxels separated by a small height differences into a connected volume, which is necessary for the navigation graph to be able to connect stairs. The value of \(d_{dilate}\) depends on the expected differences in height between the steps of stairs in the environment. Typically we use a value of 0.2m. The result is a new voxel grid \(\mathcal{V}_{dilated}\). Note that the dilation step may create new occupied voxels that are not in the original voxel grid which means that \(\mathcal{V}_{dilated}\) is not necessarily a subset of \(\voxelgrid\).

\paragraph{Connected components}
The final step of the algorithm is to split \(\mathcal{V}_{dilated}\) into one or more connected components. A connected component \(\mathcal{V}_i\) of a voxel grid is a subset of \(\mathcal{V}\) where there exists a path between every voxel in \(\mathcal{V}_i\). We denote the set of all connected components as \(\mathcal{C}=\{\mathcal{V}_{i}\}_{i=1}^n\). To find the connected components we first find the neighbourhood graph of \(\mathcal{V}_{dilated}\) using the Von Neumann neighbourhood kernel \(\mathcal{K}_6\), which is shown in equation \ref{eq:connected_components_01}. 

\begin{equation}
    \label{eq:connected_components_01}
    \mathcal{G}_{\mathcal{K}_6} = (N,\ E),\ N = \mathcal{V}_{dilated}
\end{equation}

We then find the connected components of the neighbourhood graph using the below algorithm. After doing so we find the connected component with the largest amount of voxels and use it as the navigation graph \(\navgraph\). We denote the intersection of the voxels in the navigation graph with \(\voxelgrid\) as \(\voxelgrid_{navigation}\).


\begin{algorithm}
    
    \caption{Region growing connected components}\label{alg:cap}
    \hspace*{\algorithmicindent} \textbf{Input} Dilated voxel grid \(\mathcal{V}_{dilated}\) \\
    \hspace*{\algorithmicindent} \textbf{Input} Von Neumann Connectivity kernel \(\mathcal{K}_6\) \\
    \hspace*{\algorithmicindent} \textbf{Output} Connected components \(\mathcal{C}\) \\

    \begin{algorithmic}
        \label{algo:connected}

    \State \(\mathcal{G}_{\mathcal{K}_6} = (N,\ E),\ N = \mathcal{V}_{dilated}\) \Comment{Convert voxel grid to neighbourhood graph}
    \State \(N_{unvisited} = N\)
    \State \(\mathcal{C} = \{\}\)
    
    \While{$|N_{unvisited}| \neq 0$}
        \State Select random node \(n\) from \(N_{unvisited}\)
        \State Remove \(BFS(n)\) from \(N_{unvisited}\) \Comment{Breadth-first search to find connected nodes}
        \State Add \(BFS(n)\) to \(\mathcal{C}\)
    \EndWhile
    \end{algorithmic}
\end{algorithm}

\subsubsection{Room segmentation}
The next step of our approach is to segment the complete voxel grid map \(\voxelgrid\) into non-overlapping rooms. We do so by using a visibility clustering approach. We will now describe the algorithm that we use to achieve this.

\paragraph{Maximum visibility estimation}
To segment the map into rooms using visibility clustering it is first necessary to identify the viewpoints that will be used to compute the visibilities. Ideally, we want viewpoints that maximize the view of the environment. We find these viewpoints by finding the points that are at a maximum distance from the boundary of the navigation graph within their local neighbourhood. The reasoning behind this is that the points that maximize the view of the environment should be equally spaced and as far away from any obstruction as possible. We compute these points as follows.

For each voxel in the navigation graph we compute the horizontal distance to the nearest boundary voxel. A boundary voxel is a voxel for which not every voxel in its Von Neumann neighbourhood is occupied. To compute this value we iteratively convolve the voxel grid with a circle-shaped kernel on the X-Z plane, where the radius of the circle is expanded by 1 voxel with each iteration, starting with a radius of 1. When the number of voxel neighbours within the kernel is less than the number of voxels in the kernel a boundary voxel has been reached. The number of radius expansions that were performed tells us the distance to the boundary of a particular voxel. We denote the horizontal distance of a voxel to its boundary as \(dist: \mathbb{Z}^{3+} \mapsto \mathbb{Z}\). Computing the horizontal distance for every voxel in \(\voxelgrid\) gives us the horizontal distance field (HDF), as shown in equation \ref{eq:hdf}.

\begin{equation}
    \label{eq:hdf}
HDF = \{dist(\voxel) \mid \voxel \in \voxelgrid\}
\end{equation}
\begin{equation}
    \label{eq:hdfmax}
HDF_{max} = \{\voxel \mid dist(\voxel) \geq max \{dist(\boldsymbol{v_{r}} \mid \boldsymbol{v_{r}} \in radius(\voxel,\ r))\}\}
\end{equation}

We denote the horizontal distance of a given voxel \(\voxel\) as \(d_{\voxel}\). We implement this using the following algorithm.

\algnewcommand\algorithmicforeach{\textbf{for each}}
\algdef{S}[FOR]{ForEach}[1]{\algorithmicforeach\ #1\ \algorithmicdo}

\begin{algorithm}
    \caption{Horizontal distance field}
    \hspace*{\algorithmicindent} \textbf{Input} Navigation voxel grid \(\mathcal{V}_{navigation}\) \\
    \hspace*{\algorithmicindent} \textbf{Output} Horizontal distance field \(HDF = \{dist(\voxel) \mid \voxel \in \voxelgrid_{navigation}\}\) \\

    \begin{algorithmic}
    \ForEach {$\boldsymbol{v} \in \mathcal \voxelgrid_{navigation} $}
        \State \(r=1\)
        \State Create \(\mathcal{K}_{circle}\) with radius \(r\)
        \While{$\mathcal{K}_{circle}(\voxel) = |\mathcal{K}_{circle}|)$}
            \State \(r = r+1\)
            \State Expand \(\mathcal{K}_{circle}\) with new radius \(r\)
        \EndWhile

        \State \(HDF = HDF \cup {r}\) \Comment{Add voxel's radius to horizontal distance field}
    \EndFor
    \end{algorithmic}
\end{algorithm}

We then find the maxima of the horizontal distance field within a given radius \(r \in \mathbb{R}^+\).  The local maxima of the horizontal distance field are all voxels that have a larger or equal horizontal distance than all voxels within \(r\), such that equation \ref{eq:hdfmax} follows. Increasing the value of \(r\) reduces the number of local maxima and vice versa. All voxels in \(HDF_{max}\) lie within the geometry of the environment, which means the view of the environment is blocked by the surrounding voxels. To solve this, we take the centroids of the voxels in \(HDF_{max}\) and translate them upwards to a reasonable scanning height \(h\) for a human agent, we use 1.8m. We denote these positions as:

\begin{equation}
    \label{eq:views}
views = \{\boldsymbol{v_c} + (0, h, 0) \mid \boldsymbol{v} \in HDF_{max}\}
\end{equation}

Figure \ref{fig:hdf_simple} shows an illustration of the horizontal distance field computation and the identification of its local maxima. Figure \ref{fig:hdf} shows a real example of the horizontal distance field extracted from a small two-storey environment in grayscale and the associated \(views\) in red.

\begin{figure}[h]
    \centering
    \includegraphics*[width=0.8\textwidth]{./fig/hdf_simple.png}
    \caption{Illustration of horizontal distance field computation and extraction of local maxima.}
    \label{fig:hdf_simple}
\end{figure}

\begin{figure}[h]
    \centering
    \includegraphics*[width=0.7\textwidth]{./fig/horizontal_distance_field.png}
    \caption{Resulting horizontal distance field of partial map, with resultant optimal view points shown in red.}
    \label{fig:hdf}
\end{figure}

\paragraph{Visibility computation}
The next step in the room segmentation algorithm is to compute the visibility from each position in \(views\). We denote the set of voxels that are visible from a given position as:

\begin{equation}
    \label{eq:visibility}
    visibility: \mathbb{R},\ \voxelset \mapsto \mathbb{Z}^{m \times 3},\ m \in \mathbb{R},\ n \geq m
\end{equation}

A target voxel is visible from a position if a ray cast from the position towards the centroid of the voxel does not intersect with any other voxel. To compute this we use the fast voxel traversal (FVT) algorithm to rasterize the ray onto the voxel grid in 3D (see algorithm 3) \citep{amanatides_fast_1987}. We then check if any of the voxels that the ray enters, except the target voxel, is occupied. If none are, the target voxel is visible from the point. Figure \ref{fig:voxel_raycast} shows a 2D representation of how the FVT algorithm works. Figure \ref{fig:visibility} shows a 2D representation of a visibility computation.

\begin{figure}[h]
    \centering
    \includegraphics*[width=.5\textwidth]{./fig/dda.pdf}
    \caption{2D representation of voxel raycasting.}
    \label{fig:voxel_raycast}

\end{figure}

\begin{figure}[h]
    \centering
    \includegraphics*[width=.7\textwidth]{./fig/visibility.pdf}
    \caption{2D representation of visibility computation.}
    \label{fig:visibility}

\end{figure}

We perform this raycasting operation from every position in \(views\) towards every voxel within a radius \(r_{v}\) of that position. Only taking into account voxels within a radius speeds up the visibility computation, and is justifiable based on the fact that real-world 3D scanners have limited range. We denote the set of visibilities from each point in views as:

\begin{equation}
    \label{eq:visibility_views}
visibility_{views} = \{visibility(\boldsymbol{x}) \mid \boldsymbol{x} \in views\}
\end{equation}
\begin{equation}
    visibility(\boldsymbol{o}) = \{\voxel \mid \voxel \in radius(o, r_{v}) \land FVT(\voxelgrid,\boldsymbol{o},\voxel) = \voxel\}
\end{equation}


% https://math.stackexchange.com/questions/83990/line-and-plane-intersubsection-in-3d
\begin{algorithm}
    \caption{Fast Voxel Traversal}
    \hspace*{\algorithmicindent} \textbf{Input} Voxel grid \(\voxelgrid\) \\
    \hspace*{\algorithmicindent} \textbf{Input} Ray origin \(\boldsymbol{o} \in \mathbf{R}^3, \boldsymbol{o} \in [\voxelgrid_{min}, \voxelgrid_{max}]\) \\
    \hspace*{\algorithmicindent} \textbf{Input} Ray target \(\boldsymbol{t} \in \mathbf{R}^3\) \\
    \hspace*{\algorithmicindent} \textbf{Output} $hit \in \mathbb{Z}^{3}$ \Comment{First encountered collision} \\
    \begin{algorithmic}
    \label{algo:dda}

    \State $\boldsymbol{p}_{current} = \boldsymbol{o}$
    \State $\voxel_{\boldsymbol{o}} = (\boldsymbol{o} - \voxelgrid_{min})//\voxelgrid_{e}$
    \State $\voxel_{current} = \voxel_{\boldsymbol{o}}$

    \State $\boldsymbol{d} = (\boldsymbol{t} - \boldsymbol{o})$ 
    \State $heading = \boldsymbol{d} \odot abs(\boldsymbol{d})^{-1}$ \Comment{Determine if ray points in positive or negative direction for every axis}

    \While{$(\voxel_{current} \notin \voxelgrid \lor \voxel_{current} = \voxel_{\boldsymbol{o}}) \land \boldsymbol{p}_{current} \in [\voxelgrid_{min}, \voxelgrid_{max}]$}
        \State $\boldsymbol{c} = centroid(\voxel_{current})$
        \State $d_{planes} = \boldsymbol{c} + heading*\voxelgrid_{e}/2$

        \State $d_{min} = \infty$
        \State $axis=1$

        \ForEach($d \in d_{planes}$)
            \State $t = \frac{d - \boldsymbol{n} \cdot \boldsymbol{p}_{current}}{\boldsymbol{n} \cdot (\boldsymbol{t} - \boldsymbol{p}_{current})}$
            \State $\boldsymbol{i} = \boldsymbol{p}_{current} + t(\boldsymbol{t} - \boldsymbol{o})$

            \If{$d_{min} \geq t$}
                \State $\boldsymbol{p}_{current} = \boldsymbol{i}$
                \State $\voxel_{current,\ axis} += heading_{axis}$
            \EndIf

            \State $axis=axis+1$
        \EndFor

    \State $hit = \voxel_{current}$
    \EndWhile

\end{algorithmic}
\end{algorithm}

\paragraph{Visibility clustering}
After computing the set of visibilities from the estimated optimal views we apply clustering to group the visibilities by similarity. Remember that each visibility is a subset of the voxel grid map. To compute the similarity of two sets we use the Jaccard index, which is given by equation \ref{eq:jacc_01}.
\begin{equation}
    \label{eq:jacc_01}
    J(A,B) = \frac{|A \cap B|}{|A \cup B|}
\end{equation}

Computing the Jaccard index for every combination of visibilities gives us a similarity matrix \(S^{n \times n} \in [0, 1]\).  An example similarity matrix is shown in figure \ref{fig:jaccard}. We can also consider \(S^{n \times n}\) as an undirected weighted graph \(\graph_S\), where every node represents a visibility and the edges the Jaccard index of two visibilities, as illustrated in figure \ref{fig:similarity_graph}. This means we can treat visibility clustering as a weighted graph clustering problem. To solve this problem we used the Markov Cluster (MCL) algorithm \citep{van_dongen_cluster_2000}. The main parameter of the MCL algorithm is inflation. By varying this parameter between an approximate range of \([1.2, 2.5]\) we get different clustering results. We find the optimal value for inflation within this range by maximizing the clustering's modularity. This value indicates the difference between the fraction of edges within a given cluster and the expected number of edges for that cluster if edges are randomly distributed. We denote the clustering of \(visibility_{views}\) that results from the MCL algorithm as:


\begin{figure}[h]
    \centering
    \includegraphics*[width=0.8\textwidth]{./fig/mutual_visibility_matrix.png}
    \caption{Similarity matrix extracted from set of visibilities. Each value represents the Jaccard index of two visibilities.}
    \label{fig:jaccard}
\end{figure}

\begin{figure}[h]
    \centering
    \includegraphics*[width=0.8\textwidth]{./fig/mutual_visibility_graph.png}
    \caption{Graph representation of the similarity matrix, edges under a threshold similarity are removed. Nodes represent visibilities.}
    \label{fig:similarity_graph}
\end{figure}

\begin{equation}
    \label{eq:c_visibility}
    \mathbf{C}_{visibility} = \{c_i\}_{i=1}^{|views|},\ c_i \in \mathbb{Z}^+
\end{equation}

Where the \(i\)th element of \(\mathbf{C}_{visibility}\) is the cluster that the \(i\)th element of \(visibility_{views}\) belongs to, such that for a given value of \(c\) the elements in \(visibility_{views}\) for which the corresponding \(c\) in \(\mathbf{C}_{visibility}\) have the same value belong to the same cluster. As each visibility is a subset of the map, each cluster of visibilities is also a subset of the map. We denote the union of the visibilities belonging to each cluster as \(\mathcal{V}_{c}\). 

\paragraph{Label propagation}
It is possible for visibility clusters in \(\mathcal{V}_{c}\) to have overlapping voxels. This means that each voxel in the partial map may have multiple associated visibility clusters. However, the goal is to assign a single cluster to each voxel in the map to create a non-overlapping segmentation. To solve this we assign to each voxel the cluster which contains the most visibilities that include that voxel. The result is a mapping from voxels to visibility clusters, which we will from now on refer to as rooms, as shown in equations \ref{eq:room_01} and \ref{eq:room_02}.

\begin{equation}
    \label{eq:room_01}
room: \voxel \mapsto \integers
\end{equation}
\begin{equation}
    \label{eq:room_02}
room(\voxel) = c,\ c \in \mathbf{C}_{visibility},\ \voxel \in \voxelgrid
\end{equation}

This often results in noisy results, with small, disconnected islands of rooms surrounded by other rooms. Intuitively, this does not correspond to a reasonable room segmentation. To solve this we apply a label propagation algorithm, meaning that for every voxel we find the voxels within a neighbourhood as defined by a convolution kernel. We then assign to the voxel the most common label of its neighbourhood if that label is more common than the current label. We iteratively apply this step until the assigned labels stop changing. Depending on the size of the convolution kernel the results are smoothed and small islands are absorbed by the surrounding rooms. Algorithm 4 shows our approach to label propagation.

\pagebreak

\begin{algorithm}
    \caption{Label propagation}
    \hspace*{\algorithmicindent} \textbf{Input} Voxel grid \(\voxelgrid\) \\
    \hspace*{\algorithmicindent} \textbf{Input} Initial labeling \(label^{(0)}: \mathbb{Z}^{3+} \mapsto \mathbb{Z}\) \\
    \hspace*{\algorithmicindent} \textbf{Input}  Kernel \(\mathcal{K}\) \\
    \hspace*{\algorithmicindent} \textbf{Output} Propagated labeling after \(t\) steps \(label^{(t)}: \mathbb{Z}^{3+} \mapsto \mathbb{Z}\) \\

    \begin{algorithmic}
    \label{algo:label_prop}
    \State $t=0$

    \While($\ label^{(t)} \neq label^{(t+1)}$) \Comment{Keep iterating until labels stop changing}
        \ForEach($\voxel \in \voxelgrid$)
            \State $L = \{label^{(t)}(\voxel_{nb}) \mid \voxel_{nb} \in neighbours(\voxel, \mathcal{K})\}$4

            \State $l_{max} = \mathop{argmax}_{l} \ |\{l \mid l \in L\}|$ \Comment{Most common label in neighbourhood}
            \State $l_{current} = label^{(t)}(\voxel)$  \Comment{Label of current voxel}

            \If{$|\{l \mid l \in L \land l=l_{max}\}| > |\{l \mid l \in L \land l=l_{current}|$}
                \State \(label^{(t+1)}(\voxel) = l_{max}\)
            \Else
                \State \(label^{(t+1)}(\voxel) = label^{(t)}(\voxel)\)
            \EndIf
        \EndFor

        \State $t = t+1$ \Comment{Use propagated labeling as input for next iteration}
    \EndWhile
    \end{algorithmic}
\end{algorithm}

\subsubsection{Topometric map extraction}
The above steps segment the map into multiple non-overlapping rooms using visibility clustering. In the next step we extract the topometric representation \(\topometricmap = (\mathcal{G}, \voxelgrid)\), which consists of a topological graph \(\mathcal{G}=(N,E)\) and a voxel grid map \(\voxelgrid\). Each node in \(\mathcal{G}\) represents a room and also has an associated voxel grid which is a subset of \(\voxelgrid\) and represents the geometry of that room. Edges in \(\mathcal{G}\) represent navigability between rooms, meaning that there is a path between them on the navigable volume that does not pass through any other rooms. This means that for two rooms to have a navigable relationship they need to have adjacent voxels that are both in the navigable volume. To construct the topometric map we thus add a node for every room in the segmented map with its associated voxels, we then add edges between every pair of nodes that satisfy the above navigability requirement. The exact method for extracting a topometric map from a room segmentations and the navigable voxels is shown in algorithm 5.

\begin{algorithm}
    \caption{Topometric map extraction}
    \hspace*{\algorithmicindent} \textbf{Input} Voxel grid \(\voxelgrid\) \\
    \hspace*{\algorithmicindent} \textbf{Input} Voxel grid navigation subset \(\voxelgrid_{navigation}\) \\
    \hspace*{\algorithmicindent} \textbf{Input} Room segmentation \(room: \voxel \mapsto \integers\) \\
    \hspace*{\algorithmicindent} \textbf{Output} Topological graph \(\mathcal{G}_{topology} = (V_{topology},\ E_{topology})\) \\

    \begin{algorithmic}
    \label{algo:topo_extract}
    \State Get each unique room label \(\mathbf{R} = \{room(\voxel) \mid \voxel \in \voxelgrid\}\)
    \State Split \(\voxelgrid\) by label, such that \(\mathbf{V} = \{\voxelgrid \cap \{\voxel \mid \voxel \in \voxelgrid \land room(\voxel) = r\} \mid r \in \mathbf{R}\}\)
    \State Store room label associated with each voxel grid \(room_{\voxelgrid}: r \mapsto \voxelgrid\)
    \State $V_{topology} = \mathbf{V}$

    \ForEach($\voxel \in \voxelgrid_{navigation}$)
        \State $v_r = room(\voxel)$
        \State $nbs_r = \{room(nb) \mid nb \in neighbourhood(\voxel,\ \mathcal{K}_{adjacency})\}$

        \State $r_{adjacency} = \{(r_a,\ r_b) \mid (r_a,\ r_b) \in {v_r} \times {nbs_r} \land r_a \neq r_b\}$
        \State $E_{topology} = E_{topology} \cup \{(room_{\voxelgrid}(r_a),\ room_{\voxelgrid}(r_b)) \mid (r_a,\ r_b) \in r_{adjacency}\}$
    \EndFor
    \end{algorithmic}
\end{algorithm}
\pagebreak
\subsection{Map Matching}
The process of identifying overlapping areas between partial maps is called map matching. In the case of topometric map matching, this refers to identifying which nodes represent the same rooms between two partial maps. We denote our two partial topometric maps as \(\topometricmap_A = (\mathcal{G}_A,\ \voxelgrid_A)\) and \(\topometricmap_B = (\mathcal{G}_B,\ \voxelgrid_B)\). The goal of map matching is to find a mapping \(match: v_A \mapsto v_B,\ v_A \in \mathcal{G}_A,\ v_B \in \mathcal{G}_B\) which corresponds to the real world and is robust to differences in coordinate system, resolution and quality between partial maps. To identify matches between nodes we need to be able to compute the similarity between them. To do so, we must first transform each node into a feature vector which represents both the node itself and its relationship to its neighbourhood. The feature vectors of two nodes with similar geometry and a similar neighbourhood should be close to eachother, meaning their distance in feature space is small. Conversely, the feature vectors of two dissimilar nodes should be far away from eachother. The first step of this process, encoding the node's geometry into a feature vector, is called geometrical feature embedding. The second step, encoding both the geometrical feature embedding of the node itself and of its neighbourhood into a new feature vector is called attributed node embedding. 

We hypothesize that the attributed node embedding will have better performance for map matching, especially when differences between partial maps are large, because it involves not just the node itself but also its neighbourhood in the similarity measure. This can be compared to human place recognition, where places are identified not just by their appearance but also by their relationship to their context. In this subsection we will discuss multiple algorithms used for geometrical feature embedding and attributed node embedding. We will also discuss how we identify matches between nodes based on their feature vectors.

\subsubsection{Geometrical Feature Embedding}
Geometrical feature embedding means transforming a geometric object into a feature vector \(f \in \mathbb{R}^m\), where \(m\) is the dimensionality of the vector. We denote the function that embeds a set of voxels into a feature vector as \(embed_{geometry}: \voxelset \mapsto \mathbb{R}^m\). We implement this function using three different approaches, which we discuss below.

\subsubsection{Engineered Features}
The first approach uses a number of manually engineered features to construct the feature vector from a room's geometry. These features include, for example, the height of the room and its volume. A full list of features and their explanation is given below. More features were tried, but only the ones that were found to contribute to the accuracy of the clustering by trial and error are included here. The features are computed using a point cloud derived from centroids of the occupied voxels of the voxel grid, which we will denote here as \(\mathbf{P}\).

\paragraph{Volume}
The axis aligned bounding box (aabb) of \(\mathbf{P}\) is given by the minimum and maximum value along each axis. This results in two 3-dimensional vectors \(aabb_{min}\) and \(aabb_{max}\). With these vectors we compute the length of each axis of the aabb by computing \(\mathbf{l} = aabb_{max} - aabb_{min}\). We then find the volume of the aabb by finding the product of each element of \(\mathbf{l}\).

\paragraph{Height} 
Using the same approach as described above we compute the length of each axis of the aabb, \(\mathbf{l}\). The height of the point cloud is simply the y-value of \(\mathbf{l}\).

\paragraph{Horizontal Area} 
Once again we first compute \(\mathbf{l}\). To find the horizontal area of \(\mathbf{P}\) we multiply the x- and y-values of \(\mathbf{l}\). 

\paragraph{Mean distance to centroid} 
To compute the centroid \(\mathbf{c}\) of \(\mathbf{P}\) we compute the mean value of each axis of all points in \(\mathbf{P}\). We then compute the Euclidean distance of each point in \(\mathbf{P}\) to \(\mathbf{c}\) and compute the mean distance. This metric is closely correlated with an object's volume. 

\paragraph{Number of points} 
To compute this value we simply count the number of points in the point cloud. Larger objects will generally contain more points.

\paragraph{Quotient of eigenvalues} 
By using principal component analysis we can determine the 3 eigenvectors and eigenvalues of \(\mathbf{P}\), which indicate the directions of maximal variance in the point cloud and the amount of variance along those directions. If all eigenvalues are approximately equal then no direction dominates. We compute if this is the case by finding the quotient of eigenvalues (the first eigenvalue divided by the second and third). If the quotient is close to 1 then no direction dominates.

\paragraph{Ratio of smallest eigenvalue to sum of two largest eigenvalues} 
We take the two largest eigenvalues and find their sum, then we divide the smallest eigenvalue by it. Objects for which the largest two eigenvalues are much larger than the smallest eigenvector have a single direction that is non-dominant.

\paragraph{Verticality of largest eigenvector} 
We take the largest eigenvector, normalize it, and then find the dot product of the eigenvector and the unit vector in the z-direction. This gives us the degree to which the point cloud is vertically aligned. If the largest eigenvalue is non-vertical then the object is mostly horizontal, which is the case for fences, buildings and cars. If the largest eigenvalue is vertical then the object is vertical, which is the case for poles and trees.

\paragraph{Roughness}
For each point we find their \(n\) nearest neighbours. We then fit a plane through the point's neighbourhood, we do this by finding the eigenvectors of the neighbourhood. The smallest eigenvector gives us the normal vector of the neighbourhood, which along with the neighbourhood's centroid gives us the best fit plane. We then determine the sum distance of each point in the neighbourhood in the plane. The roughness of the point cloud is then given by the mean of the sum distance of each point's neighbourhood to its best fit plane. 

\subsubsection{Spectral Features}
Another approach to feature embedding uses the first \(n\) sorted nonzero eigenvalues of the adjacency or Laplacian matrix of a graph. In our case, the graph refers to the neighbourhood graph of the voxel grid associated with each node in the topometric map. By adjusting the size of the kernel used to construct the neighbourhood graph we can adjust the number of edges, which influences the resulting embedding. We use singular value decomposition to find the eigenvalues.

\subsubsection{Deep Learning}
Our final approach to feature embedding is deep learning. In this approach, the geometry of the nodes is fed into a neural network that is trained on a specific task, such as semantic segmentation or classification, and the intermediate output of a hidden layer is used as a feature vector. We use an architecture called Point Completion Network (PCN), which is an autoencoder used for the specific task of completing incomplete point clouds. Autoencoders consist of two components: an encoder and a decoder. The encoder component is responsible for reducing the dimensionality of the input data into a single n-dimensional feature vector which captures the input's defining characteristics. The decoder is responsible for recovering the input data from the encoder's output as accurately as possible. For our purposes, only the output of the encoder is required. The advantage of using an autoencoder network is that they are trained in an unsupervised manner, so no manually labelled data is required. We train our model on the Stanford 3D Indoor Scene dataset (see results section). To generate incomplete views we compute visibilities from one or multiple random poses in each room.

\subsubsection{Attributed Node Embedding}
Attributed node embedding aims to find a feature embedding for each node in a graph that uses both an attribute of the node, in our case a geometrical feature embedding, and the node's relationship to the rest of the graph. We denote the function that embeds a node's attribute \(f_{attr}\) and its graph into a feature vector as \(embed_{node}: \mathbb{R}^m,\ \graph \mapsto \mathbb{R}^m\), such that \(embed_{node}(f_{attr},\ \graph_{attr}) = f_{node}\). Finding the attributed node embedding of a node \(n\) in the topometric map \(\topometricmap\) with topological graph \(\graph_{T}\) is then equal to computing \(f_{node}=embed_{node}(embed_{geometry}(n),\ \graph_{T})\). 

We compute the attributed node embedding using graph convolution. This means that for every node, we find its adjacent nodes and add their feature vectors weighted by a factor \(w\) to the origin node's feature vector. If we denote the adjacent nodes of \(v\) in graph \(G\) as \(N_{G}(v) = \{v_i\}_{i=1}^{k}\), then each node's feature vector after graph convolution becomes \(v^{t+1} = v^t + \sum_{i=1}^k wN_{G}(v^t)_i\). By repeating this step the embedding of a node's neighbourhood is integrated in each node's embedding, making nodes that have a similar neighbourhood more similar and vice versa. Increasing the number of iterations also increases the distance at which a neighbour influences a node's embedding. Changing the value of \(w\) increases the influence of neighbours. 

\subsubsection{Map Matching}
The above steps are applied to both partial maps. This gives us two sets of feature vectors \(\mathcal{E}_A,\ \mathcal{E}_B\) representing the embedding of the nodes of both topometric maps.

To identify the most likely overlapping rooms between the partial maps we find the one-to-one mapping between the elements of \(\mathcal{E}_A\) and \(\mathcal{E}_B\) that maximizes the similarity (or minimizes the distances) between the chosen pairs. This is an example of the unbalanced assignment problem, which consists of finding a matching in a weighted bipartite graph that minimizes the sum of its edge weights. It is unbalanced because there may be more nodes in one part of the bipartite graph than the other, which means it is not possible to assign every node in one part to a node in the other. 

To construct the weighted bipartite graph we first find the Cartesian product of the feature vectors \(\mathcal{E}_{AB} = \mathcal{E}_A \times \mathcal{E}_B = \{(a,b) \mid a \in \mathcal{E}_A,\ b \in \mathcal{E}_B\}\). We then compute the Euclidean distance in feature space between every pair of nodes in \(\mathcal{E}_{AB}\), creating the cost matrix that represents the weighted bipartite graph \(\mathbf{C} \in \mathbb{R}^{|V_a| \times |V_b|},\ \mathbf{C}_{ij} = ||a - b||,\ (a,\ b) \in \mathcal{E}_{AB},\ a = \mathcal{E}_{A,\ i},\ b = \mathcal{E}_{B,\ j},\ \mathbf{C}_{ij} \in \mathbb{R},\ \mathbf{C}_{ij} \geq 0\).

We can then find unbalanced assignment using various approaches, in our case the Jonker-Volgenant algorithm [CITE]. We denote the resulting matching between the nodes of both partial maps and their distance in feature space as a set of triples \(\mathbf{M} = \{(i,\ j,\ d) \mid  |V_a| \geq i,\ |V_b|\geq j,\ d = \mathbf{C}_{ij}\}\). 

In practice it is unlikely that every match in \(\mathbf{M}\) is correct. However, we can use them as seeds to generate hypotheses similar to the approach described in Huang \citep{huang_topological_2005}. Starting at each match \((v_i,\ v_j) = (V_{A,\ i},\ V_{B,\ j}),\ (i,\ j) \in \mathbf{M}\) we get the neighbourhood of both nodes. We then construct a new cost matrix from the Euclidean distance between the embeddings of both neighbourhoods, again creating a weighted bipartite graph for which we can solve the assignment problem. By doing this we identify which neighbours of the nodes in the match are most likely to also match. We recursively apply this step to the matching neighbours to grow our initial matches into hypotheses. To decrease the risk of incorrectly identifying neighbourhood matches we constrain hypothesis growing in two ways. First, the cost of two potential matches must be below a given threshold \(c_{max}\). Second, a newly identified match may not bring the existing matching too much out of alignment. To check this, we perform coarse registration between the centroids of the geometry of the identified matches at every step of the hypothesis growing using least squares adjustment. If the error increases between steps, and the increase is too large such that \(\triangle e \geq \triangle e_{max} \), then the matching is rejected. By adjusting the values of \(c_{max}\) and \(\triangle e_{max}\) more or less uncertainty is allowed when growing hypotheses.

The hypothesis growing step produces multiple hypotheses, one for every initial matching. As described in Huang \citep{huang_topological_2005} we then cluster the hypotheses by similarity and compatibility. Two hypotheses are compatible if there are no contradicting matches between them. They are similar if the distance between their coarse registration computed during the hypothesis growing process is small. If two hypotheses are incompatible their distance is set to infinite. We use the OPTICS algorithm to cluster hypotheses. After clustering the hypotheses multiple hypotheses might still remain. We select the hypothesis with the largest number of matches as the most likely hypothesis. If there are multiple hypotheses with an equal largest number of matches then the hypothesis with the lowest mean distance in feature space between matches is selected.

\pagebreak
\subsection{Map Fusion}
The final step of the map merging process is map fusion. In the case of topometric maps this means fusion at the geometric and topological level. We arbitrarily designate one partial map as the source and the other as the target. The goal of geometric fusion is to find a rigid transformation that aligns the source map with the target. 

At the topological level, we first merge the nodes of the source map with their corresponding matches in the target map. If two adjacent nodes in one map match with two other adjacent nodes in the other partial map then their edges are also merged. Afterwards, a transformation between the geometry (a point cloud derived from the voxel grid's centroids) of each matching node is computed using a fine registration approach. We use two different algorithms two do this: iterative closest point (ICP) and deep closest point (DCP).

\paragraph{Iterative closest point}
The iterative closest point (ICP) algorithm finds a transformation between two point clouds by iteratively aligning \(n\) random points from the source point cloud to the \(n\) points that are closest to them in the target point cloud using least squares adjustment. This approach is sensitive to local minima but can often lead to good results if enough data is available. Its simple nature also makes it easy to modify. For example, constraining the registration to only allow rotation around a single axis is a matter of changing the equations used in the least square adjustment step. This

\paragraph{Deep closest point}
Deep closest point is a variation on the ICP algorithm which uses deep learning to identify matching points instead of using the closest points. This approach is less sensitive to local minima and data quality. However, constraining the registration is difficult as it requires retraining the network.

After finding a transformation between every match outliers that represent an incorrect registration are removed. We then find the mean of the remaining transformations and apply this to the source map. The result is a global topometric map \(\mathcal{T}_{global}\).
\pagebreak                                                     
\section{Results}
We evaluated the algorithm described in the methodology section on several datasets. In this section we describe these datasets and show the results we achieved with them. The shown results are divided into three sections based on the major components of our methodology: map extraction, map matching and map fusion. 

\subsection{Datasets}
In this section we describe the datasets that we used to test our algorithm. We also describe how we prepared the ground truth datasets so as to be able to compare our results to them and objectively measure their performance. 

\subsubsection{Simulated Scan}
To objectively evaluate the performance of our methodology it is necessary to compare the results to a ground truth. This ground truth must contain the following elements: room labels and their topological relationships to evaluate map extraction, room matches between partial maps to evaluate map matching, and the transformations between partial maps to evaluate map fusion. When capturing real-world data, the second and third elements are especially difficult to determine. Furthermore, little pre-existing data is available that contains exactly these elements. To solve this problem we simulate partial maps from annotated global maps from various sources. The annotations are integer labels for every point representing the ground truth room segmentation and a graph representing the ground truth topological map. 

We create the partials maps by manually defining a trajectory for each desired partial map and simulating an agent moving along them, scanning the global map at a set interval. We do this by using the DDA algorithm described in the methodology section. This allows us to objectively evaluate our approach for a large number of different scenarios at the cost of missing some of the subtleties inherent in non-simulated partial maps, such as changes in the environment and measurement error. To mitigate this, we apply pre-processing steps to the global map. These pre-processing steps are: random removal of points, adding noise to points and random rotation and translation of the point cloud. The latter's purpose is to simulate the unknown transformation between real-world partial maps. By comparing the results of our approach to the annotations in the ground truth global map we can objectively measure the performance of map extraction, matching and fusion. Figure \ref{fig:simulated} shows an example global map with simulated viewpoints. Figure \ref{fig:area_1_partial_01} and \ref{fig:area_1_partial_02} show two simulated partial maps created from a single map.

\pagebreak

\begin{figure}[h]
    \centering
    \includegraphics*[width=.6\textwidth]{./fig/simulated_views.png}
    \caption{Global map with simulated viewpoints. Each coloured dot represents a viewpoint, the union of all views with the same color forms a partial map.}
    \label{fig:simulated}
\end{figure}

\subsubsection{Stanford 3D Indoor Scene Dataset}
The Stanford 3D Indoor Scene Dataset (S3DIS) consists of a collection of 3D scans of six different indoor environments (Area 1 through 6) and the raw measurements used to create them. The environments all span a single floor. Each environment scan in its original state consists of a number of point clouds, one for each room in the building. We merge these separate point clouds into a single cloud but retain the room labels as point attributes, as we use these as our ground truth room labels for the map extraction step. We manually created the topological graph of each area. The S3DIS was captured using high-end 3D scanners and thus has a high point density.

\subsubsection{Collaborative SLAM Dataset}
The Collaborative SLAM Dataset (CSLAMD) is a dataset meant specifically for collaborative SLAM. It consists of three environments (House, flat and lab), each consisting of multiple partial maps and their ground truth transformations. Two of the environments consist of multiple storeys. The partial maps were captured using a low-end 3D scanner and thus has a low point density. We merge the partial maps of each environment into a single global map to create the simulated partial maps described above. We then manually annotate each environment with our interpretation of an appropriate room segmentation and manually create the topological graph.

\subsubsection{Various Sources}
Finally, we also include datasets from various sources. These include a two-storey house with basement and an office conference room, both captured using high-end 3D scanners. We annotated these datasets by hand according to our interpretation of the spaces and their segmentation into rooms and manually create the topological graph. 

\begin{figure}[h]
    \centering
    \includegraphics*[width=\textwidth]{./fig/area_1_partial.pdf}
    \caption{First simulated partial map extracted from S3DIS area 1 dataset.}
    \label{fig:area_1_partial_01}
\end{figure}

\pagebreak

\subsection{Map Extraction}
In this section we show the results of our map extraction approach. To evaluate the results of our map extraction approach we compare the extracted topometric partial maps to the ground truth global map. For each node's geometry in a topometric map we find its Jaccard index with every node in the global map's geometry. This gives us a weighted bipartite graph, one part being the nodes in the partial map and the other being the nodes in the global map, with the weight representing the Jaccard index between nodes' geometry. We assign each node in the partial map to a single node in the global map using linear assignment, as is described in the map matching section. This gives us the correspondence between nodes in the partial map and nodes in the global map.

\paragraph{Room segmentation}
After finding the correspondences between the partial maps and the ground truth global map we compute the mean Jaccard index of the correspondences. This metric is called Mean Intersection over Union (MIoU) and it measures the quality of our room segmentation. Figure \ref{fig:map_extract_perf} shows the MIoU for each of the partial maps in the S3DIS dataset. Partial maps where map extraction has completely failed are indicated by a dash pattern.

\begin{figure}[h]
    \centering
    \includegraphics*[width=\textwidth]{./fig/map_extract_chart.pdf}
    \caption{First topometric map extracted from S3DIS area 1 dataset.}
    \label{fig:map_extract_perf}
\end{figure}

\paragraph{Sparse voxel octree}
To measure the impact that the sparse voxel octree data structure has on ball query performance we compare its computation time for balls of different radii versus using a hash table to check each if each voxel in the ball is occupied. The results of this measurement are shown in figure \ref{fig:svo_perf}.

\begin{figure}[h]
    \centering
    \includegraphics*[width=\textwidth]{./fig/svo_chart.pdf}
    \caption{First topometric map extracted from S3DIS area 1 dataset.}
    \label{fig:svo_perf}
\end{figure}

\paragraph{Result maps}
The results of our map extraction approach are plotted as top-down views in figures \ref{fig:area_1_topo_01} and \ref{fig:area_1_topo_02}. Each room is coloured using a unique random colour. The topological edges between rooms are shown as solid black lines.

\begin{figure}[h]
    \centering
    \includegraphics*[width=\textwidth]{./fig/area_1_topo.pdf}
    \caption{First topometric map extracted from S3DIS area 1 dataset.}
    \label{fig:area_1_topo_01}
\end{figure}

\subsection{Map Matching}
In this section we show the results of our map matching approach. Using to partial map to global map node correspondences described in the previous section we can distinguish incorrect matches from correct matches; a match between two partial maps is correct if both nodes in the match correspond to the same node in the global map. 

\paragraph{Matching precision}
Figures \ref{fig:lpdnet_match} and \ref{fig:shapedna_match} show the precision of our map matching approaches divided by descriptor type, the number of steps included in the contextual embedding and whether hypothesis growing was used. 

\begin{figure}[h]
    \centering
    \includegraphics*[width=\textwidth]{./fig/match_charts.pdf}
    \caption{First topometric map extracted from S3DIS area 1 dataset.}
    \label{fig:lpdnet_match}
\end{figure}


\paragraph{Matching results}
Figure \ref{fig:area_1_match} shows the matches between the partial maps of area 1 of the S3DIS dataset, the same partial maps as in the previous sections. Each match is given a unique random colour, with a room in the first partial map having the same colour as its match in the second. These matches were generated using hypothesis growing and a one step contextual embedding with ShapeDNA descriptors


\begin{figure}[h]
    \centering
    \includegraphics*[width=\textwidth]{./fig/area_1_match.pdf}
    \caption{First topometric map extracted from S3DIS area 1 dataset.}
    \label{fig:area_1_match}
\end{figure}

\pagebreak

\subsection{Map Fusion}
In this section we show the results of our map merging approach. 

\paragraph{Transformation error}
We evaluate the performance of our map merging approach by finding the difference between the computed transformation and the transformation applied to the ground truth when simulating the partial maps. We split the difference into two parts, translation and rotation error. The translation error is given in meters and is computed by finding the magnitude of the difference between the computed and ground truth translation. The rotation error is the difference between computed and the ground truth rotation around the y-axis and is given in degrees.

\paragraph{Relationship between descriptor and registration error}
Figure \ref{fig:lpdnet_registration} shows the relationship between the distance in feature space between two rooms and their point-to-plane distance error after registration. This measures the degree to which descriptor similarity predicts how well two rooms align. Both results use two step context embedding.


\begin{figure}[h]
    \centering
    \includegraphics*[width=\textwidth]{./fig/registration_charts.pdf}
    \caption{First topometric map extracted from S3DIS area 1 dataset.}
    \label{fig:lpdnet_registration}
\end{figure}

\paragraph{RANSAC}
Figure \ref{fig:ransac_results} shows the number of optimal transformations that were found by RANSAC within a range of iterations. 

\begin{figure}[h]
    \centering
    \includegraphics*[width=\textwidth]{./fig/ransac_optima.pdf}
    \caption{Second topometric map extracted from S3DIS area 1 dataset.}
    \label{fig:ransac_results}
\end{figure}

\paragraph{ICP}
Figure \ref{fig:icp_convergence} shows the relationship between the number of iterations of the iterative closest point algorithm and the point-to-plane error for a large number of matches. Individual iterative closest point runs are coloured in grey, the mean of all runs is coloured in red.

\begin{figure}[h]
    \centering
    \includegraphics*[width=\textwidth]{./fig/icp_convergence.pdf}
    \caption{Second topometric map extracted from S3DIS area 1 dataset.}
    \label{fig:icp_convergence}
\end{figure}

\begin{figure}[h]
    \centering
    \includegraphics*[width=\textwidth]{./fig/map_fuse_chart.pdf}
    \caption{Second topometric map extracted from S3DIS area 1 dataset.}
    \label{fig:area_1_match_02}
\end{figure}

\paragraph{Fuse results}
Figure \ref{fig:area_1_global} shows the global map created by fusing the partial maps of area 1 of the S3DIS dataset. 

\begin{figure}[h]
    \centering
    \includegraphics*[width=\textwidth]{./fig/area_1_global.jpg}
    \caption{Second topometric map extracted from S3DIS area 1 dataset.}
    \label{fig:area_1_global}
\end{figure}

\pagebreak
\section{Discussion}

\subsection{Map Extraction}
In the previous section we showed the results of our map extraction approach. In this section we will discuss these results. For the majority of tested environments the resultant room segmentation closely matches the ground truth room segmentation. This is especially the case in environments where rooms have clear delineations; environments where rooms have walls between them and are only connected by small openings. The results of our room segmentation approach match less closely in environments where this is not the case. Our approach often splits single rooms that are large in one or multiple dimensions, such as hallways or auditoriums, into multiple rooms. Additionally, rooms where there are obstructions to the view from inside the room are also split into multiple parts. However, these effects do not necessarily indicate a failure of our approach. The segmentation in the ground truth data is based on human intuition about what separates a room from its neighbours. Although room segmentation based on visibility clustering often closely matches this intuition it is inherently different as it does not take into account the intended use of rooms. Where humans might recognize that a long hallway or a large hall serves a single purpose, and should therefore be considered as the same room, visibility clustering fails to take this subjective interpretation of purpose into account. Nevertheless, the objective visibility clustering approach comes remarkably close to the subjective human approach. The opposite also occurs, where two or more rooms that are separate in the ground truth data are not separate in the room segmentation. This mostly occurs when there are no obstructions between two adjacent rooms. This can often be resolved by changing the clustering parameters. However, when the only separation between two rooms is based on purpose and not on visibility then our approach cannot separate them.

A common failure mode of our approach, which causes room segmentation to fail completely, is when the input point cloud data is of insufficient density to construct a connected navigation graph. In this case, the voxels belonging to the navigation graph are identified correctly but there are gaps between voxels. This can be solved by increasing the size of the kernel used for constructing the neighbourhood graph of the navigable voxels. However, this has the side effect that voxels that are not actually navigable are added to the navigation graph. The result is that low elevated surfaces with sloping sides, beds for example, are added to the navigation graph. While this usually does not have a large effect on map extraction in extreme cases it can also cause the ceiling to become part of the navigation graph. This will usually cause significant errors in room segmentation, as the view from above the ceiling towards the rest of the map is often completely unobstructed.

Another way that room segmentation may fail is when stairs have very shallow treads and steep rises (respectively the horizontal and vertical part of its steps). This causes the stick kernel approach to fail to label the stairs' voxels as navigable, which means it will not be included in the navigation graph. This is because the wide part of the stick kernel placed on one step may intersect with the next step. If there are no other connections between two storeys then this will cause one or multiple storeys to become disconnected from the navigation graph, excluding it from the extracted topometric map. This problem can be resolved by changing the dimensions of the stick kernel. However, this may in turn cause other problems. Increasing the height of the thin part of the stick kernel causes low elevated surfaces with sloping sides to be included in the navigation graph, as described in the previous paragraph. Decreasing the radius of the stick kernel's wide part will include parts of the map that are not actually navigable in the navigation graph. The problem can also be solved by increasing the size of the kernel used to construct the navigable voxels' neighbourhood graph to force the stairs'  voxels to become connected even though some are missing but this causes the same issues as described in the previous paragraph.

Differences between the ground truth topological graph and the extracted topological graph are caused by differences in room segmentation. One such case is when a hallway connected to a room is split into multiple rooms around the opening towards the connected room. This will result in a triangular subgraph between the two parts of the hallway and the connected room, which in reality should just be a single edge between the hallway and the room. 

\subsection{Map Matching}
\subsection{Map Fusion}
\subsection{Future Works}



\bibliographystyle{apacite}
\bibliography{bibliography}



\end{document}
