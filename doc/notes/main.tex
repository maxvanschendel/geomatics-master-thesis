% !TeX root = main.tex
\documentclass{article}
\usepackage[english]{babel}
\usepackage[utf8x]{inputenc}
\usepackage{graphicx}
\usepackage{apacite}
\usepackage{natbib}
\usepackage{amsfonts} 
\usepackage{amsmath}
\usepackage{mathtools}
\usepackage{bm} 
\usepackage{algpseudocode}
\usepackage{algorithm}
\usepackage{glossaries}
\usepackage{afterpage}

\title{Extraction and Merging of Topometric Maps}

\author{
  Maximiliaan van Schendel\\
  student \#4384644 \\
  \url{m.vanschendel@tudelft.nl}\\
  \\
  1st supervisor: Edward Verbree \\
  2nd supervisor: Pirouz Nourian \\
  external supervisor: Robert Voûte \\
}

\date{24/01/2022}
\begin{document}
\usepackage{amsfonts}
\usepackage{bm}
\usepackage[utf8]{inputenc}

\newcommand{\voxel}{\boldsymbol{v}}
\newcommand{\voxelgrid}{\mathcal{V}}
\newcommand{\voxelset}{\mathbb{Z}^{n \times 3}}
\newcommand{\graph}{\mathcal{G}}
\newcommand{\integers}{\mathbb{Z}}

\newcommand{\topometricmap}{\mathcal{T}}
\newcommand{\topologicalgraph}{}
% TODO: FIX DEFINITIONS TO MATCH CHANGE OF NOMENCLATURE AND NOTATION


\newglossaryentry{map}
{
    name=map,
    description={A symbolic representation of an environment which contains information about its characteristics.}
}

\newglossaryentry{maprepresentation}
{
    name=map representation,
    description={The choice of characteristics of the environment that a map shows. Hybrid map representations are representations that shows a combination of multiple characteristics of the environment, e.g. a hybrid topological-metric map containing both the environment's large-scale structure and small-scale geometry.}
}

\newglossaryentry{agent}
{
    name=agent,
    description={According to \citet{russell_artificial_2010} "An agent is anything that can be viewed as perceiving its environment through sensors and
    acting upon that environment through actuators.". In the context of this thesis an agent specifically refers to any human or robot that is capable of perceiving its environment by means of a monocular camera, stereo camera or lidar and is capable of exploring their environment. We further define heterogeneous agents as having different sensing capabilities and being unable to communicate or position themselves relative to eachother.}
}

\newglossaryentry{partialmap}
{
    name=partial map,
    description={A collection of maps without a common coordinate frame that each represent a part of the environment. We denote the set of all partials maps as \(I = \{m_i\}_{i=1}^n\), where \(n\) is the number of partial maps. We define heterogeneous partial maps as partial maps captured by heterogeneous agents, thus having different scale, resolution, accuracy or precision.}
}

\newglossaryentry{globalmap}
{
    name=global map,
    description={A single, more complete map constructed by merging multiple partial maps. To do so, a coordinate transformation \(T\) must be applied to the partial maps to bring them into a common coordinate frame. We denote the global map constructed by merging a subset of partial maps \(J \subset I\) using coordinate transformations \(\{T_i\}_{i=1}^{|J|}\) as \(G_J\), such that \(G_J=\{T_i(m_i)\}_{i=1}^{|J|}\).}
}

\newglossaryentry{mapmerging}
{
    name=map merging,
    description={The map merging problem can be stated as follows: given two partial maps \(m_1\in I, m_2\in I\), find the coordinate transformation \(T\) that minimizes a dissimilarity function \(\psi(m_1,T(m_2))\) \citep{carpin_map_2005}. The goal of this is to "maximize the overlap of regions that appear in two or more partial maps" \citep{carpin_map_2005}. Depending on the approach, the coordinate transformation, dissimilarity function or optimization method differ.}
}

\newglossaryentry{metricmap}
{
    name=metric map,
    description={Metric maps represent the geometry of an environment. They are usually derived from sensor range measurements, either directly or by using structure from motion or simultaneous localization and mapping algorithms. A common metric map representation is the occupancy grid \citep{andersone_heterogeneous_2019}. Another common metric map representation is the point cloud, which represents the surface of the environment as a collection of points. Both map representations are shown in figure \ref{fig:voxel_example}.}
}

\newglossaryentry{topologicalmap}
{
    name=topological map,
    description={Topological maps are a qualitative graph representation of an environment's structure, where vertices represent locally distinctive places, often rooms, and edges represent traversable paths between them  (see figure \ref{fig:topo_ex}) \citep{thrun_learning_1998,kuipers_robust_1988}. Topological maps are inspired by the fact that humans are capable of spatial learning despite limited sensory and processing capability and only having partial knowledge of the environment. This is based on observations that cognitive maps, the mental maps used by humans to navigate within an environment, consist of multiple layers with a topological description of the environment being a fundamental component \citep{kuipers_robust_1988,kuipers_modeling_1978}. We denote the topology of an environment as a graph \(G\), where vertices \(V\) represent \(n\) distinctive places \(v_i\) and edges \(E\) represent the presence of \(m\) traversable paths between neighbouring pairs of places \(\{v_j,v_k\}\), such that \(G=(V, E)\), \(V=\{v_i\}_{i=1}^n, E=\{\{v_j,v_k\}_i\}_{i=1}^m, v_j \in V, v_k \in V\). In the context of indoor mapping the graph is often embedded in 2D or 3D euclidean space as a spatial graph. Given the embedding \(f : G \rightarrow R^n\), \( \widetilde{G}:=f(G)\), we denote \(\widetilde{G}\) as the spatial graph of \(G\) \citep{kobayashi_spatial_1994}. Each vertex in \(\widetilde{G}\) has an associated 3D coordinate describing its position in the coordinate frame of the map. If the specific paths between vertices are known they can be associated with the edges in \(\widetilde{G}\), otherwise an edge only represents a possible path between two points in space.}
}

\newglossaryentry{pointcloud}
{
    name=point cloud,
    description={An unordered collection of points representing the geometry of an object or environment in 3D euclidean space, defined as \(\mathcal{P}=\{p_i\}_{i=1}^n, p_i \in \mathbb{R}^3\), where \(n\) denotes the number of points \citep{volodine_point_2007}.}
}

\newglossaryentry{voxelgrid}
{
    name=voxel grid,
    description={Also known as an occupancy grid, a voxel grid is a "multi-dimensional (typically 2D or 3D) tesselation of space into cells, where each cell stores a probabilistic estimate of its state." \citep{elfes_occupancy_1990}. A voxel grid with a number of cells along each dimension, \(O_{dim} \in \mathbb{Z}^3_{\geq0}\) can be represented by a collection of positive integer cell indices that have a non-zero chance of being occupied \(O=\{c_i\}_{i=1}^{n}\), \(n \in \mathbb{R}\), \(n \in [1, \prod_{i=1}^{3}O_{dim,i}], c \in \mathbb{Z}^3_{\geq 0} \). The probability of a cell being occupied is given by a cell's associated state variable \(s(c) \in \mathbb{R}\), \(s(c) \in [0,1]\). Cells that are within the extent of the grid that are not stored have zero chance of being occupied, such that \(\forall c \in \{c \notin O | c_i \in [0, O_{dim,i}]\} \Rightarrow s(c)=0\). In the case of a binary occupancy grid a cell can either be occupied or not, such that \(s(c) \in \{0,1\}\). In this case, explicitly storing a state variable is not necessary as only the cells in \(O\) are occupied, as given by \(\forall c \in O \Rightarrow s(c)=1\). }
}

\newglossaryentry{topometricmap}
{
    name=topometric map,
    description={A hybrid map representation combining both the topological and metric characteristics of the environment. This map representation allows the end-user to use either topological or metric information depending on the needs of the situation, e.g. the topological layer can be used for large-scale navigation and abstract reasoning while the metric layer can be used for landmark detection or obstacle avoidance. In the context of this thesis a topological-metric map refers to a 3D representation of an environment containing both a metric occupancy grid map \(\mathbb{M}_M\) representing its geometry and a spatial graph \(\widetilde{\mathbb{M}_T}=(V, E)\) representing its topology. Each vertex \(v \in V\) has an associated variable \(m(v)\) representing a subset of the full metric map describing that place, such that \(m(v) \subset \mathbb{M}_M\). }
}
\maketitle
\pagebreak

\tableofcontents
\newpage

\pagenumbering{arabic}

% abstract

\paragraph{abstract}
The merging of multiple partial \gls{map}s of indoor environments created by teams of human or robot agents into a single global map is a key problem that, when solved, can enable co-localization and collaborative mapping of large environments. Existing map merging approaches generally depend on external signals which are not available indoors or only use the  geometric properties of an environment. Inspired by the human understanding of environments in relationship to their context we propose a map merging system that extracts and uses topometric maps, a map representation containing both the geometrical and topological characteristics of an environments, to solve the map merging problem in indoor spaces. In this research we demonstrate an intuitive approach to extracting topometric maps of 3D, multi-floor, indoor environments. We then use both the topological and geometric characteristics contained in the topometric maps to perform context-aware map matching and fusion. 



%TODO: MODIFY INTRO TO WORK WITH CHANGE OF SCOPE AND CONTENTS

\section{Introduction}
Collaborative mapping allows multiple agents to work together to create a single, global map of their environment. By working together large areas can be mapped in a short amount of time. Collaborative mapping of indoor environments is especially challenging, because external positioning signals are highly attenuated. In this case, a global map can only be created by merging the partial maps of the environment created by each agent individually based on their overlapping areas. This is called map merging (see figure \ref{fig:map_merging}). Partial maps might represent the environment at different a different scale, resolution or accuracy, which further complicates map merging as overlapping areas may not appear the same between partial maps.

\begin{figure}[h]
    \centering
    \includegraphics*[width=\textwidth]{./fig/map_merging.png}
    \caption{Partial maps (1) captured by three different agents and resultant global map (2) after map merging.}
    \label{fig:map_merging}
\end{figure}

% the method of solving the problem, often stated as a claim or a working thesis;
In this thesis we propose to use both the topological relationships of indoor environments, meaning the connectivity between distinctive places, and their metric characteristics, the geometry of the environment, to solve the map merging problem.  We do this by extracting hybrid 3D topometric maps from heterogeneous partial maps. We then use both the topological and metric characteristics of the partial maps to robustly identify overlapping areas that might appear differently.  Our hypothesis is that the connectivity of places within the environment are identifiable between heterogeneous partial maps. We further hypothesize that using the metric characteristics of the environment in conjunction with its topological characteristics will improve identification of overlapping areas over a purely topological approach, despite geometrical differences between heterogeneous partial maps. The most important contributions of our work are: 1) applying 3D topometric map merging to indoor environments. 2) extracting 3D topometric maps from heterogeneous partial maps.
% research background
\section{Research Questions}

\subsection{Main question}
How can we apply topometric representations of indoor environments to solve the map merging problem?

\subsection{Subquestions}
\begin{enumerate}
    \item In what way can partial topometric maps be extracted from partial point cloud maps?
    \item What approach is best suited for identifying matches between partial topometric maps?
    \item How can the identified matches be used to fuse two or more partial topometric maps into a global topometric map?
\end{enumerate}

\subsection{Scope}

During this thesis the following will be created.

\begin{enumerate}
    \item A program that is able of topometric map extraction from point clouds and topometric map merging.
    \item An analysis of the program's performance on publically available standard datasets.
    \item Reports containing documentation and background research.
\end{enumerate}

To better delineate the scope of the thesis we provide several aspects that will \textbf{not} be researched or discussed. 

\begin{enumerate}
    \item Map merging using known relative poses between agents or meeting strategies. Agent behaviour is assumed to be independent and agents are not able to sense eachother.
    \item Map merging using observations unrelated to the environment's geometric or topological characteristics. E.g. the environment's colour or actively transmitted beacon signals.
    \item Map merging assisted by a priori knowledge of the environment. E.g. building information models or floor plans.
    \item Map merging using the pose graphs of agents. Agent poses are assumed to be unknown.
    \item Achieving near real-time performance.
\end{enumerate}
\pagebreak
\section{Related work}
Previous research on mapping and map merging has considered various map representations \citep{tomatis_hybrid_2003,huang_topological_2005,bonanni_3-d_2017,gholamishahbandi_2d_2019}. According to \citet{andersone_heterogeneous_2019} and \citet{yu_review_2020} these representations can be subdivided into three types: metric-, feature-, and topological maps. Hybrid maps that are combinations of two or more map types also exist, such as topometric maps, which are a combination of metric and topological maps \citep{yu_review_2020}. Map types that are not one of the three main types or a hybrid are rarely used but do exist \citep{yu_review_2020}. In this section we will discuss the work that has been done on extracting and merging metric and topometric maps. Figure \ref{fig:euler} shows a diagram of the fields of research that are relevant for this thesis.

\begin{figure}[h]
    \centering
    \includegraphics*[width=.8\textwidth]{./fig/euler_diagram.drawio.pdf}
    \caption{Euler diagram showing the overlapping fields of research that are relevant for this thesis.}
    \label{fig:euler}
\end{figure}


\subsection{Metric Maps}
\label{section:metric_map_merging}
In this section we give an overview of the existing research into metric map extraction and map merging.

\subsubsection{Metric map extraction}
Metric maps are a map representation that represent the geometry of an environment. Two common metric map types that are relevant for our research are point clouds and voxel grids. Point clouds are usually the direct output of 3D mapping sensors and algorithms so it is not necessary to extract them from another map representation \citep{rusu_3d_2011}. \citet{volodine_point_2007} gives an overview of point clouds and how to process them. \citet{elfes_occupancy_1990} gives a description of voxel grids and how to extract them from point cloud maps.

\subsubsection{Metric map merging}
Metric map matching is the problem of recognizing overlapping areas between partial maps based purely on their geometry. Metric map matching is a mature area of research that has applications for 3D mapping and place recognition.  Some approaches use \gls{localdescriptor}s that describe the properties of each point in the point cloud to identify corresponding points between point clouds. Descriptors may include corners, lines, planes or other points of interest, e.g. SIFT, SURF, FPFH or Harris points \citep{andersone_heterogeneous_2019,rusu_fast_2009}. Some descriptors are better suited for different environments and input data. For example, the descriptor approach by \citet{li_general_2010} is scale-independent and the approach of \citet{yang_fast_2016} is able to deal with differences in resolution. Recent research into using deep learning for local descriptor extraction has also shown great potential, examples include PointNet and DGCNN \citep{qi_pointnet_2017,phan_dgcnn_2018}.

Another approach to metric map matching uses \gls{globaldescriptor}s that describe the properties of segments of the point cloud, overlapping segments can then be identified based on the similarity of their global descriptor \citep{dube_segmatch_2017}. A large number of global descriptors have been proposed which describe the point cloud based on various properties, such as its volume, planarity or roughness \citep{han_comprehensive_2018}. Another approach to global descriptor extraction aggregrates the local descriptor of each point into a global descriptor, examples include DBoW and VLAD \citep{shan_robust_2021,arandjelovic_all_2013}. Global descriptors can also be based on the spectral characteristics of graphs derived from the point cloud, including approaches such as ShapeDNA and heat kernel signatures \citep{reuter_laplacebeltrami_2006,bronstein_scale-invariant_2010}. An advantage of these approaches is that they do not require manual selection of relevant descriptors. Finally, recent research has shown that deep learning can be used to extract global descriptors, with approaches such as PointNetVLAD, LPDNet and MinkLoc3D \citep{uy_pointnetvlad_2018-1,liu_lpd-net_2019,komorowski_minkloc3d_2021}. Although these approaches claim to have better resilience against differences between partial maps than other approaches they often require large amounts of training data and may not be able to deal with environments that are not similar to the training data. Note that the above deep learning approaches often use a trainable variation of the local descriptor aggregration described above \citep{arandjelovic_netvlad_2016}. 

Metric map fusion is a mature area of research and various approaches have been proposed. The problem of metric map fusion comes down to finding a transformation between partial maps that brings the geometry of their overlapping areas into alignment, this is called registration. Most approaches to registration include a variation of the iterative closest point (ICP) algorithm. This algorithm finds the transformation between two point clouds by iteratively applying rigid transformations that minimize the distance between the points in one point clouds and their closest points in the other \citep{rusinkiewicz_efficient_2001}. Since its first introduction a number of variations on the ICP algorithm have been proposed that improve its accuracy and performance. An example of this is the normal iterative closest point algorithm, which improves data-association by taking into account the normal vector of a point's neighbourhood and the gravity-aligned ICP algorithm, which constrains the transformation to use with maps that have a consistent up direction \citep{serafin_nicp_2015,kubelka_gravity-constrained_2022}. 

The ICP algorithm and its variations are not guarantueed to find a globally optimal transformation and they are sensitive to the initial transformation between point clouds. To mitigate this \citet{yang_fast_2016} proposes a two step algorithm that first performs a rough, global registration followed by a precise, local registration (see figure \ref{fig:global_local}). The function of the global registration step is to find a good initialization that will allow the local registration step to find the global optimum. Various approaches to global registration have been proposed, with most depending on the detection of correspondences between point clouds using one of the descriptor approaches described above. A global transformation can then be estimated using RANSAC \citep{koguciuk_parallel_2017}.

\begin{figure}[h]
    \centering
    \includegraphics*[width=\textwidth]{./fig/feature_matching.png}
    \caption{Illustration of two-step approach to registration \citep{yang_fast_2016}.}
    \label{fig:global_local}
\end{figure}

More recently, deep learning approaches for point cloud registration have been proposed, such as PointNetLK and DCP \citep{aoki_pointnetlk_2019,wang_deep_2019}. However, these methods require training and their results can not be constrained without retraining the model.

\pagebreak

\subsection{Topometric Maps}
In this section we discuss the existing research into the extraction and merging of topometric maps.

\subsubsection{Topometric map extraction}
Various approaches for extracting topological maps from raw sensor data or metric maps have been proposed. \citet{bormann_room_2016} and \citet{pintore_state---art_2020} both give a review of different methods for extracting structured maps of indoor spaces in which different room segmentation methods are discussed and compared.

\citet{kuipers_robot_1991} first proposes identifying distinctive places, nodes of the topological map, directly from sensor data by finding local maxima of a distinctiveness measure within a neighbourhood. Edges are identified by having the robot try to move between vertices, if this is possible an edge is created. Local metric information in the form of an occupancy grid is associated with the nearest vertex in the graph, resulting in a topometric map. Note that this approach is dependent on the mapping agent's exploration strategy. 

% visibility clustering
\citet{thrun_learning_1998} extracts a 2D topological map from a 2D metric map by identifying narrow passages. They then partition the metric map into areas divided by passages which are respectively the vertices and the edges of the graph. This approach is not able to deal with 3D environments with multiple storeys and it assumes that rooms are always separated by narrow passages.

\citet{mura_automatic_2014} proposes an approach to room segmentation that divides the ground plane of the environment into polygons which are then iteratively clustered to form rooms. \citet{mura_piecewise-planar_2016} avoids this clustering step by directly clustering scanner positions using a graph clustering approach. This approach is then further improved by \citet{ambrus_automatic_2017} by generating synthetic scanning positions along the medial axis of the map. Figure \ref{fig:vis_cluster} illustrates the concept behind visibility clustering-based room segmentation.

\begin{figure}[h]
    \centering
    \includegraphics*[width=.7\textwidth]{./fig/visibility_clustering.png}
    \caption{Illustration of visibility clustering-based room segmentation \citep{pintore_state---art_2020}.}
    \label{fig:vis_cluster}
\end{figure}

\citet{ochmann_towards_2014} describes extracting hierarchical topometric maps directly from point clouds. The hierachy is divided into four encapsulating layers, building - storey - room - object. Entities within the graph are represented as vertices, with edges representing the topological and spatial relationships between entities. Each vertex is linked to a local metric map of the entity's geometry. 

\citet{gorte_navigation_2019} provides a novel approach for extracting the walkable floor space from a voxel grid across multiple storeys (see figure \ref{fig:gorte}). They do so by first applying a 3D convolution filter using a stick-shaped kernel to extract the parts of the floor without obstructions. They then apply an upwards dilation to connect steps of stairways into a connected volume. Because the topology of the environment depends on the traversability between spaces the extraction of navigable floor space is essential for the extraction of topological maps.

\begin{figure}[h]
    \centering
    \includegraphics*[width=.7\textwidth]{./fig/gorte.png}
    \caption{Illustration of walkable floor space extraction \citep{gorte_navigation_2019}.}
    \label{fig:gorte}
\end{figure}

\citet{he_hierarchical_2021} describes an approach for extracting a hierarchical topological-metric map with three layers: storey - region - volume, from a voxel grid map. To extract the map they use a novel approach to room segmentation using raycasting. A downside of their methodology is that it depends on the presence of ceilings in the metric map, which are often not captured in practice when using handheld scanners. 

% deep learning
\citet{ma_semantic_2020} and \citet{tang_bim_2022} both propose a deep learning approach to semantic segmentation of indoor spaces. Despite the fact that the primary goal of these methods is not room segmentation but instance segmentation (splitting the map into walls, floor, furniture, etc.) their findings can also be applied to room segmentation. In contrast to previous approaches these approaches require training a model on a labelled dataset.

\subsubsection{Topometric map merging}
In comparison to topometric map extraction relatively little work has been done on the subject of topometric map merging. \citet{dudek_topological_1998} first proposes an approach to topological map merging which depends on a robot meeting strategy to merge partial maps created by each robot. When new distinctive places are recognized at the frontier of the global map the other robots will travel towards it and synchronize their maps. As with other early approaches to map extraction and map merging this approach depends on a coordinated exploration strategy.

The work of \citet{huang_topological_2005} is a significant milestone in topological map merging, demonstrating that topological maps can be merged using both map structure and map geometry. They first identify vertex matches by comparing the similarity of their attributes, such as their degree, and the spatial relationships of incident edges. Vertex matches are then expanded using a region growing approach, where every added edge and vertex is compared for similarity and rejected if too dissimilar. The results are multiple hypotheses for overlapping areas between partial maps. They then estimate a rigid transformation between the partial maps for each hypothesis. Afterwards, hypotheses that result in similar transformations are grouped into hypothesis clusters. They then select the most appropriate hypothesis cluster by using a heuristic that includes the number of vertices in the cluster, the error between matched vertices after transformation and the number of hypotheses in the cluster.

\citet{bonanni_3-d_2017} provides a unique approach to topometric map merging using the pose graph of mapping agents as the topological component and the point cloud captured at each node as the metric component. Matches between nodes are identified by computing the similarity of their associated point cloud. In comparison to most other map merging approaches they fuse the maps using a non-rigid transformation, meaning the partial maps are deformed to improve their alignment. 

\citet{garcia-fidalgo_hierarchical_2017} proposes a hierarchical approach for place recognition in topological maps in which images of the environment are grouped by similarity and described by both a local descriptor of their properties and a global descriptor of their grouping's properties. This approach reduces the search space when recognizing places. Note that this approach does not use a 3D metric component but an image-based one.

\citet{rincon_map_2019} proposes an approach to topological map merging that is based on both the similarity of nodes and their context within the graph based on a model of human object recognition. This approach depends on the previous alignment of partial maps.

\citet{rozemberczki_multi-scale_2021} propose an approach to feature embedding in graphs that combines each node's associated attribute with the distribution of its neighbourhood's attributes over multiple scales. While they do not use this for 3D mapping their approach can be applied to topometric maps.
\pagebreak

\subsection{Map Extraction}

The goal of the topometric map extraction step is to transform a partial voxel grid map of an indoor environment, denoted by \(\voxelgrid\), into a topometric map, denoted by \(\topometricmap\). The algorithm should be robust the missing data and work across a wide range of data resolution. In this section we propose an algorithm to achieve this goal. In an overview, it works as follows.

We first extract a navigation graph \(\navgraph\), a graph connecting all of the voxels which a hypothetical agent could use to move through the environment, from the partial voxel grid map \(\voxelgrid\). Using \(\navgraph\) we compute points where the hypothetical agent would have an optimal view of the environment. We then find the visible voxels for each of these points. By clustering the resultant visibilities based on similarity, we segment \(\voxelgrid\) into submaps that align closely with how humans divide indoor environments into rooms. As such, we refer to the submaps of \(\voxelgrid\) when segmented using visibility clustering as 'rooms'. Afterwards, we construct the topological graph of the environment, denoted by \(\topologicalgraph\), with edges representing whether it is possible to navigate between two rooms without passing through another. Finally, we fuse the topological map with the segmented map to construct the topometric map \(\topometricmap\). 

Figure \ref{fig:map_extract_steps} shows an overview of our map extraction algorithm, its input, and intermediate outputs. In the rest of this subsection we will the algorithm in more detail.

\begin{figure}[h]
    \centering
    \includegraphics*[width=\textwidth]{./fig/map_extract.png}
    \caption{Diagram showing map extraction processes and intermediate data.}
    \label{fig:map_extract_steps}
\end{figure}

\subsubsection{Navigable volume}
In our first step we extract a navigation graph \(\navgraph\). The navigation graph tells us how a theoretical agent in the map's environment would navigate through that environment. In practice, assuming a human agent, this means the areas of the floor, ramps and stairs that are at a sufficient distance from a wall and a ceiling. We compute \(\navgraph\) using a three step algorithm which we describe below.

\subsubsection{Convolution}
The first step uses voxel convolution with a stick-shaped kernel \(\mathcal{K}_{stick}\) (shown in \ref{fig:stick_kernel}) to find all voxels that are unobstructed and are thus candidates for navigation. Each voxel in the kernel has a weight of 1, except the origin voxel which has weight 0. The associated function is summation. Convolving the voxel grid's occupancy property with the stick kernel gives us each voxel's obstruction property, denoted as \(\voxel_{obstruction}\) with an obstruction value of 0 when no other voxels are present in the stick kernel. This indicates that these voxels have enough space around and above them to be used for navigation. We then filter out all obstructed voxels, such that \(\mathcal{V}_{unobstructed}=\{\boldsymbol{v} \mid \voxel \in convolve(\mathcal{V},\ \mathcal{K}_{stick}),\ \voxel_{obstruction} = 0\}\). 

\begin{figure}[h]
    \centering
    \includegraphics*[width=0.5\textwidth]{./fig/structuring_element.png}
    \caption{Illustration of stick kernel with top and side views.}
    \label{fig:stick_kernel}
\end{figure}

\subsubsection{Dilation}
The next step of the algorithm is to dilate the unobstructed voxels upwards by a distance \(d_{dilate}\). This connects the unoccupied voxels separated by a small height differences into a connected volume. The value of \(d_{dilate}\) depends on the expected differences in height between the steps of stairs in the environment. Typically, we use a value of 0.2m for this parameter. The result is a new voxel grid \(\mathcal{V}_{dilated}\). Note that the dilation step may create new occupied voxels that are not in the original voxel grid, which means that \(\mathcal{V}_{dilated}\) is not a subset of \(\voxelgrid\).

\subsubsection{Connected components}
The final step of the algorithm is to split \(\mathcal{V}_{dilated}\) into one or more connected components. A connected component \(\mathcal{V}_c\) of a voxel grid is a subset of \(\mathcal{V}\) where there exists a path between every voxel in \(\mathcal{V}_c\). The neighbourhood graph \(\mathcal{G}_{\mathcal{V},\ \mathcal{K}} = (V,\ E)\) of \(\mathcal{V}\) represents the voxels in \(\mathcal{V}\) as nodes. Each node has incident edges towards all other voxels in its neighbourhood, which is defined by a kernel \(\mathcal{K}\), such that \(V = \mathcal{V},\ E_{V} = \{(\boldsymbol{v}, \boldsymbol{v_{nb}})\ |\ \boldsymbol{v} \in \mathcal{V},\ \boldsymbol{v_{nb}} \in \mathcal{K}_{\boldsymbol{v}}\},\ |E_{V}| \leq |\mathcal{K}|*|\mathcal{V}|\). 

There exists a path between two voxels \(\boldsymbol{v_a},\ \boldsymbol{v_b}\) in \(\mathcal{V}\) if there exists a path between their corresponding nodes \(V_a,\ V_b\) in \(\mathcal{G}_{\mathcal{V}}\). We denote the set of all possible paths between two nodes as \(paths(V_a,\ V_b)\). A connected component is then defined as \(\mathcal{V}_c=\{\boldsymbol{v_a}\ |\ \boldsymbol{v_a} \in \mathcal{V},\ \boldsymbol{v_b} \in \mathcal{V},\ |paths(V_a, V_b)| \neq 0\}\). We define the set of all connected components as \(\mathcal{V}_{cc}=\{\mathcal{V}_{c,\ i}\}_{i=1}^n\). We find the connected components using the following algorithm:

\begin{algorithm}
    \caption{Region growing connected components}\label{alg:cap}
    \begin{algorithmic}

    \Require \quad Voxel grid \(\mathcal{V}\)
    \Require \quad Connectivity kernel \(\mathcal{K}\)
    \Ensure \quad Connected components \(\mathcal{V}_{cc}\)

    \State \(\mathcal{G}_{\mathcal{V},\ \mathcal{K}} = (V,\ E),\ V \in \mathcal{V}\) \Comment{Convert voxel grid to neighbourhood graph using specified kernel}
    \State \(V_{unvisited} = V\)
    \State \(\mathcal{V}_{cc} = \{\}\)
    
    \State \(i = 0\)
    \While{$|V_{visited}| \neq 0$}
        \State Select random node \(n\) from \(V_{unvisited}\)
        \State \(n_{BFS} = BFS(n)\) \Comment{Use breadth-first search to find all nodes connected to \(n\)}
        \State Remove \(n_{BFS}\) from \(V_{unvisited}\)
        \State \(\mathcal{G}_i = (n_{BFS},\ E)\)
        \State Add \(\mathcal{V}_{c,\ i}\) to \(\mathcal{V}_{cc}\)
        \State \(i = i + 1\)
    \EndWhile
    \end{algorithmic}
\end{algorithm}

After doing so, we find the connected component in \(\mathcal{V}_{cc}\) with the largest amount of voxels \(\mathcal{V}_{max}\), which corresponds to the voxels used for navigation. The navigation graph \(\navgraph\) is the neighbourhood graph of \(\mathcal{V}_{max}\). We denote the intersubsection of the navigation graph's nodes and the whole voxel grid map as \(\mathcal{V}_{navigation} = \mathcal{V}_{max} \cap \mathcal{V}\).

\subsubsection{Maximum visibility estimation}
To compute the isovists necessary for room segmentation it is first necessary to estimate hypothetical scanning positions that maximize the view of the map. We compute these by finding the local maxima of the horizontal distance field of \(\mathcal{V}_{navigation}\). The steps to achieve this are as follows. 

\subsubsection{Horizontal distance field}
For each voxel in \(\voxelgrid\) we compute the horizontal Manhattan distance to the nearest boundary voxel. A boundary voxel is a voxel for which not every voxel in its Von Neumann neighbourhood is occupied. To compute this value we iteratively convolve the voxel grid with a circle-shaped kernel on the X-Z plane, where the radius of the circle is expanded by 1 voxel with each iteration, starting with a radius of 1. When the number of voxel neighbours within the kernel is less than the number of voxels in the kernel a boundary voxel has been reached. Thus, the number of radius expansions tells us the Manhattan distance to the boundary of a particular voxel. We denote the horizontal distance of a voxel to its boundary as \(dist: \mathbb{Z}^{3+} \mapsto \mathbb{Z}\). Computing the horizontal distance for every voxel in \(\voxelgrid\) gives us the horizontal distance field (HDF), such that \(HDF = \{dist(\voxel) \mid \voxel \in \voxelgrid\}\). We denote the horizontal distance of a given voxel \(\voxel\) as \(d_{\voxel}\). We implement this using the following algorithm.

\algnewcommand\algorithmicforeach{\textbf{for each}}
\algdef{S}[FOR]{ForEach}[1]{\algorithmicforeach\ #1\ \algorithmicdo}

\begin{algorithm}
    \caption{Horizontal distance field}
    \begin{algorithmic}

    \Require \quad Navigation voxel grid \(\mathcal{V}_{navigation}\)
    \Ensure \quad Horizontal distance field \(HDF = \{dist(\voxel) \mid \voxel \in \voxelgrid_{navigation}\}\)

    \\

    \ForEach {$\boldsymbol{v} \in \mathcal \voxelgrid_{navigation} $}
        \State \(r=1\)
        \State Create \(\mathcal{K}_{circle}\) with radius \(r\)
        \While{$convolve(\boldsymbol{v},\ \mathcal{K}_{circle}) = |\mathcal{K}_{circle}|)$}
            \State \(r = r+1\)
            \State Expand \(\mathcal{K}_{circle}\) with new radius \(r\)
        \EndWhile

        \State \(HDF = HDF \cup {r}\) \Comment{Add voxel's radius to horizontal distance field}
    \EndFor
    \end{algorithmic}
\end{algorithm}

We then find the maxima of the horizontal distance field within a given radius \(r \in \mathbb{R}\).  The local maxima of the horizontal distance field are all voxels that have a larger or equal horizontal distance than all voxels within \(r\), such that \(HDF_{max} = \{\voxel \mid dist(\voxel) \geq max \{dist(\boldsymbol{v_{r}} \mid \boldsymbol{v_{r}} \in radius(\voxel,\ r))\}\}\). Increasing the value of \(r\) reduces the number of local maxima and vice versa. All voxels in \(HDF_{max}\) lie within the geometry of the environment, which means the view of the environment is blocked by the surrounding voxels. To solve this, we take the centroids of the voxels in \(HDF_{max}\) and translate them upwards to a reasonable scanning height \(h\), to estimate the positions with the optimal view of the map. We denote these positions as \(views = \{\boldsymbol{v_c} + (0, h, 0) \mid \boldsymbol{v} \in HDF_{max}\}\). We use the following algorithm to compute \(views\).

\begin{algorithm}
    \caption{Horizontal Distance Field Maxima}
    \begin{algorithmic}

    \Require \quad Navigation voxel grid \(\mathcal{V}_{navigation}\)
    \Require \quad Horizontal distance field \(HDF\)
    \Require \quad Radius \(r \in \mathbb{R}^+\)
    \Require \quad Scanning height \(h \in \mathbb{R}\)

    \Ensure \quad Estimated optimal views $views \in \mathbb{R}^3$

    \\

    \State $views = \{\}$

    \ForEach {$\boldsymbol{v} \in \mathcal \voxelgrid_{navigation} $}
        \State \(neighbourhood = radius(\boldsymbol{v},\ r)\)
        \If{$d_{\voxel} \geq \max{\{d_{nb} \mid neighbourhood\}}$} \Comment{Check if voxel's horizontal distance is equal or greater than the horizontal distance of every voxel in its neighbourhood}
            \State $views = views \cup \{centroid(\voxel) + (0, h, 0)\}$
        \EndIf
    \EndFor
    
    \end{algorithmic}
\end{algorithm}

\begin{figure}[h]
    \centering
    \includegraphics*[width=0.8\textwidth]{./fig/hdf_simple.png}
    \caption{Illustration of horizontal distance field computation and extraction of local maxima.}
    \label{fig:hdf_simple}
\end{figure}

\begin{figure}[h]
    \centering
    \includegraphics*[width=0.7\textwidth]{./fig/horizontal_distance_field.png}
    \caption{Resulting horizontal distance field of partial map, with resultant optimal view points shown in red.}
    \label{fig:hdf}
\end{figure}

\subsubsection{Visibility}
The next step in the room segmentation algorithm is to compute the visibility from each position in \(views\). We denote the all voxels that are visible from a given position as \(visibility: \mathbb{R},\ \voxelset \mapsto \mathbb{Z}^{m \times 3},\ m \in \mathbb{R},\ n \geq m\). A target voxel is visible from a position if a ray cast from the position towards the centroid of the voxel does not intersect with any other voxel. To compute this we use the digital differential analyzer (DDA) algorithm to rasterize the ray onto the voxel grid in 3D. We then check if any of the voxels that the ray enters- except the target voxel- is occupied. If none are, the target voxel is visible from the point. We perform this raycasting operation from every position in \(views\) towards every voxel within a radius \(r_{visibility}\) of that position. Only taking into account voxels within a radius speeds up the visibility computation, and is justifiable based on the fact that real-world 3D scanners have limited range. We denote the set of visibilities from each point in views as \(visibility_{views} = \{visibility(\boldsymbol{x}) \mid \boldsymbol{x} \in views\}\). The below pseudocode shows how the visibility computation works.

\begin{algorithm}
    \caption{Visibility}
    \begin{algorithmic}

    \Require \quad Voxel grid \(\voxelgrid\)
    \Require \quad Origin \(o \in \mathbf{R}^3\)
    \Require \quad Range \(r_{visibility} \in \mathbb{R}^+\)
    \Ensure \quad $visibility \in \mathbb{Z}^{3}$

    \State $visibility = \{\voxel_c \mid \voxel_c \in radius(o, r_{visibility}) \land DDA(\voxelgrid,\ o,\ centroid(\voxel_c)) = \voxel_c\}$

    \end{algorithmic}
\end{algorithm}

% https://math.stackexchange.com/questions/83990/line-and-plane-intersubsection-in-3d
\begin{algorithm}
    \caption{DDA (Digital Differential Analyzer)}
    \begin{algorithmic}

    \Require \quad Voxel grid \(\voxelgrid\)
    \Require \quad Ray origin \(\boldsymbol{o} \in \mathbf{R}^3, \boldsymbol{o} \in [\voxelgrid_{min}, \voxelgrid_{max}]\)
    \Require \quad Ray target \(\boldsymbol{t} \in \mathbf{R}^3\)

    \Ensure \quad $hit \in \mathbb{Z}^{3}$ \Comment{First encountered collision}

    \State $\boldsymbol{p}_{current} = \boldsymbol{o}$
    \State $\voxel_{\boldsymbol{o}} = (\boldsymbol{o} - \voxelgrid_{min})//\voxelgrid_{e}$
    \State $\voxel_{current} = \voxel_{\boldsymbol{o}}$

    \State $\boldsymbol{d} = (\boldsymbol{t} - \boldsymbol{o})$ 
    \State $heading = \boldsymbol{d} \odot abs(\boldsymbol{d})^{-1}$ \Comment{Determine if ray points in positive or negative direction for every axis}

    \While{$(\voxel_{current} \notin \voxelgrid \lor \voxel_{current} = \voxel_{\boldsymbol{o}}) \land \boldsymbol{p}_{current} \in [\voxelgrid_{min}, \voxelgrid_{max}]$}
        \State $\boldsymbol{c} = centroid(\voxel_{current})$
        \State $d_{planes} = \boldsymbol{c} + heading*\voxelgrid_{e}/2$

        \State $d_{min} = \infty$
        \State $axis=1$

        \ForEach($d \in d_{planes}$)
            \State $t = \frac{d - \boldsymbol{n} \cdot \boldsymbol{p}_{current}}{\boldsymbol{n} \cdot (\boldsymbol{t} - \boldsymbol{p}_{current})}$
            \State $\boldsymbol{i} = \boldsymbol{p}_{current} + t(\boldsymbol{t} - \boldsymbol{o})$

            \If{$d_{min} \geq t$}
                \State $\boldsymbol{p}_{current} = \boldsymbol{i}$
                \State $\voxel_{current,\ axis} += heading_{axis}$
            \EndIf

            \State $axis=axis+1$
        \EndFor

    \State $hit = \voxel_{current}$
    \EndWhile

\end{algorithmic}
\end{algorithm}

\subsubsection{Room segmentation}
After computing the set of visibilities from the estimated optimal views we apply clustering to group the visibilities by similarity. This is based on the definition of a room as a region of similar visibility. Remember that each visibility is a subset of the voxel grid map. To compute the similarity of two sets we use the Jaccard index, which is given by \(J(A,B) = \frac{|A \cap B|}{|A \cup B|}\). Computing the Jaccard index for every combination of visibilities gives us a similarity matrix \(S^{n \times n} \in [0, 1]\). The similarity matrix is symmetric because \(J(A,B) = J(B,A)\). Its diagonals are 1, as \(J(A,A) = 1\). An example similarity matrix is showin in figure \ref{fig:jaccard}.

\begin{figure}
    \centering
    \includegraphics*[width=0.8\textwidth]{./fig/mutual_visibility_matrix.png}
    \caption{Similarity matrix extracted from set of visibilities. Each value represents the Jaccard index of two visibilities.}
    \label{fig:jaccard}
\end{figure}

We can also consider \(S^{n \times n}\) as an undirected weighted graph \(\graph_S\), where every node represents a visibility and the edges the Jaccard index of two visibilities. This means we can treat visibility clustering as a weighted graph clustering problem. To solve this problem we used the Markov Cluster (MCL) algorithm (CITE MCL), which has been shown by previous research to be state-of-the-art for visibility clustering. 


The main parameter of the MCL algorithm is inflation. By varying this parameter between an approximate range of \([1.2, 2.5]\) we get different clustering results. We find the optimal value for inflation within this range by maximizing the clustering's modularity. This value indicates the difference between the fraction of edges within a given cluster and the expected number of edges for that cluster if edges are randomly distributed. 

% TODO: Give mathematical description of MCL
% TODO: Give mathematical equation of modularity

We denote the clustering of \(visibility_{views}\) that results from the MCL algorithm as \(\mathbf{C}_{visibility} = \{c_i\}_{i=1}^{|views|},\ c_i \in \mathbb{Z}^+\), where the \(i\)th element of \(\mathbf{C}_{visibility}\) is the cluster that the \(i\)th element of \(visibility_{views}\) belongs to, such that for a given value of \(c\), the elements in \(visibility_{views}\) for which the corresponding \(c\) in \(\mathbf{C}_{visibility}\) has the same value belong to the same cluster. As each visibility is a subset of the map, each cluster of visibilities is also a subset of the map. We denote the union of the visibilities belonging to each cluster as \(\mathcal{V}_{c}\). 

It is possible for visibility clusters in \(\mathcal{V}_{c}\) to have overlapping voxels. This means that each voxel in the partial map may have multiple associated visibility clusters. However, the goal is to assign a single room to each voxel in the map. To solve this we assign to each voxel the cluster in which the most visibilities contain that voxel. The result is a mapping from voxels to visibility clusters (which we will from now on refer to as rooms), which we denote as \(room: \voxel \mapsto \integers\), such that \(room(\voxel) = c,\ c \in \mathbf{C}_{visibility},\ \voxel \in \voxelgrid\). This often results in noisy results, with small, disconnected islands of rooms surrounded by other rooms. Intuitively, this does not correspond to a reasonable room segmentation. To solve this, we apply a label propagation algorithm. This means that for every voxel we find the voxels within a neighbourhood as defined by a convolution kernel. We then assign to the voxel the most common label, in this case the room, of its neighbourhood, if that label is more common than the current label. We iteratively apply this step until the assigned labels stop changing. Depending on the size of the convolution kernel the results are smoothed and small islands are absorbed by the surrounding rooms.


\begin{algorithm}
    \caption{Label propagation}
    \begin{algorithmic}

    \Require \quad Voxel grid \(\voxelgrid\)
    \Require \quad Initial labeling \(label^{(0)}: \mathbb{Z}^{3+} \mapsto \mathbb{Z}\)
    \Require \quad Kernel \(\mathcal{K}\)
    \Ensure \quad Propagated labeling after \(t\) steps \(label^{(t)}: \mathbb{Z}^{3+} \mapsto \mathbb{Z}\)

    \State $t=0$

    \While($label^{(t)} \neq label^{(t+1)}$) \Comment{Keep iterating until labels stop changing}
        \ForEach($\voxel \in \voxelgrid$)
            \State $L = \{label^{(t)}(\voxel_{nb}) \mid \voxel_{nb} \in neighbours(\voxel, \mathcal{K})\}$4

            \State $l_{max} = \mathop{argmax}_{l} \ |\{l \mid l \in L\}|$ \Comment{Most common label in neighbourhood}
            \State $l_{current} = label^{(t)}(\voxel)$  \Comment{Label of current voxel}

            \If{$|\{l \mid l \in L \land l=l_{max}\}| > |\{l \mid l \in L \land l=l_{current}|$}
                \State \(label^{(t+1)}(\voxel) = l_{max}\)
            \Else
                \State \(label^{(t+1)}(\voxel) = label^{(t)}(\voxel)\)
            \EndIf
        \EndFor

        \State $t = t+1$ \Comment{Use propagated labeling as input for next iteration}
    \EndWhile
    \end{algorithmic}
\end{algorithm}

\subsubsection{Topometric map extraction}
The above steps segment the map into multiple non-overlapping rooms based on visibility clustering. In the next step we transform the map into a topometric representation \(\topometricmap = (\topologicalgraph, \voxelgrid)\), which consists of a topological graph \(\topologicalgraph=(V,E)\) and a voxel grid map \(\voxelgrid\). Each node in \(\topologicalgraph\) represents a room, and also has an associated voxel grid which is a subset of \(\voxelgrid\) and represents the geometry of that room. Edges in \(\topologicalgraph\) represent navigability between rooms, meaning that there is a path between them on the navigable volume that does not pass through any other rooms. This means that for two rooms to have a navigable relationship they need to have adjacent voxels that are both in the navigable volume. To construct the topometric map we thus add a node for every room in the segmented map with its associated voxels, we then add edges between every pair of nodes that satisfy the above navigability requirement.  


\begin{algorithm}
    \caption{Topology extraction}
    \begin{algorithmic}

    \Require \quad Voxel grid \(\voxelgrid\)
    \Require \quad Voxel grid navigation subset \(\voxelgrid_{navigation}\)
    \Require \quad Room segmentation \(room: \voxel \mapsto \integers\)
    \Require \quad Adjacency kernel $\mathcal{K}_{adjacency}$

    \Ensure \quad Topological spatial graph \(\widetilde{\mathcal{G}}_{topology} = (V_{topology},\ E_{topology})\)
    \Ensure \quad Node embedding \(f_{node}: V \mapsto \mathbb{R}^3\)

    \State $\widetilde{\mathcal{G}}_{topology} = (\{\},\ \{\})$

    \State Get each unique room label \(\mathbf{R} = \{room(\voxel) \mid \voxel \in \voxelgrid\}\)
    \State Split \(\voxelgrid\) by label, such that \(\mathbf{V} = \{\voxelgrid \cap \{\voxel \mid \voxel \in \voxelgrid \land room(\voxel) = r\} \mid r \in \mathbf{R}\}\)
    \State Store room label associated with each voxel grid \(room_{\voxelgrid}: r \mapsto \voxelgrid\)
    \State $V_{topology} = \mathbf{V}$
    \State $f_{node}(v) = centroid(v),\ v \in V_{topology}$

    \ForEach($\voxel \in \voxelgrid_{navigation}$)
        \State $v_r = room(\voxel)$
        \State $nbs_r = \{room(nb) \mid nb \in neighbourhood(\voxel,\ \mathcal{K}_{adjacency})\}$

        \State $r_{adjacency} = \{(r_a,\ r_b) \mid (r_a,\ r_b) \in {v_r} \times {nbs_r} \land r_a \neq r_b\}$
        \State $E_{topology} = E_{topology} \cup \{(room_{\voxelgrid}(r_a),\ room_{\voxelgrid}(r_b)) \mid (r_a,\ r_b) \in r_{adjacency}\}$
    \EndFor
    \end{algorithmic}
\end{algorithm}

\subsection{Map Matching}

\subsubsection{Overview}
The process of identifying overlapping areas between partial maps is called map matching. In the case of topometric map matching, this refers to identifying which nodes represent the same rooms between two partial maps. We denote our two partial topometric maps as

\begin{equation}
    \label{eq:tmap_a}
    \topometricmap_a = (\mathcal{G}_a,\ \voxelgrid_a),\ \mathcal{G}_a=(N_a,E_a)
\end{equation}

\begin{equation}
    \label{eq:tmap_b}
    \topometricmap_b = (\mathcal{G}_b,\ \voxelgrid_b),\ \mathcal{G}_b=(N_b,E_b)
\end{equation}

The goal of map matching is to find a one-to-one mapping between the rooms of both partial maps which corresponds to the real world and is robust to differences in coordinate system, resolution and quality between partial maps. 

To identify matches between rooms we need to be able to compute their similarity. To do so, we first transform each room into a descriptor, an n-dimensional vector, which represents both the geometry of the room. The descriptor of two nodes with similar geometry should be close to eachother in feature space, meaning the distance between their vectors should be small. Conversely, the descriptors of two dissimilar rooms should be far away from eachother in feature space. 

We then use the topological properties of the topometric maps to improve map matching in two ways. The first is contextual embedding. This means that we combine the descriptor of each room with the descriptor of its neighbourhood in the topological graph. This improves matching because multiple rooms may have similar geometry but not necessarily similar neighbourhoods. The second is hypothesis growing, which means that we grow multiple matching hypotheses along the topological graph in a constrained manner and only use the hypothesis that contains the most matches. 

Figure \ref{fig:flowchart_match} shows an overview of the steps described above. In the rest of this section we will describe the aforementioned steps in depth.

\begin{figure}[h]
    \centering
    \includegraphics*[width=0.5\textwidth]{./fig/flowchart_match.pdf}
    \caption{Diagram showing map matching methodology.}
    \label{fig:flowchart_match}
\end{figure}


\subsubsection{Geometric descriptor}
Geometric feature embedding transforms a geometric object, in our case a voxel grid, into an m-dimensional descriptor \(f_{geometry}\), such that objects with similar geometry are nearby in feature space and vice versa.

\begin{equation}
    \label{eq:embed_geometry_01}
    embed_{geometry}: \voxelset \mapsto \mathbb{R}^m,\ embed_{geometry}(n) = \prescript{n}{}{f_{geometry}}
\end{equation}

We denote the function that embeds a set of voxels into a feature vector as in equation \ref{eq:embed_geometry_01}. We implement this function using two different approaches, which we discuss below.

\subsubsection{Spectral Features}
Our first approach to geometric feature embedding uses spectral shape analysis. This approach uses the first \(n\) sorted, non-zero eigenvalues of the graph Laplacian, in our case of the neighbourhood graph of a room's geometry, as a geometric descriptor. To compute this we first convert each room's neighbourhood graph \(\mathcal{G}\) to an adjacency matrix \(A\) and a degree matrix \(D\). We then find the Laplacian matrix of the neighbourhood graph by subtracting its adjacency matrix from its degree matrix, as shown in equation \ref{eq:laplacian_matrix}.

\begin{equation}
    \label{eq:laplacian_matrix}
L = D - A
\end{equation}

After computing the Laplacian matrix we find its eigenvalues, sort them in ascending order and use the first 256 non-zero values as the descriptor. 


\subsubsection{Deep Learning}
Our second approach to geometric feature embedding uses deep learning. Specifically, we use the LPDNet neural network architecture. This architecture is used for place recognition, it does so by learning descriptors, typically 2048 or 4096-dimensional, of point clouds that are theoretically independent of transformation, perspective and completeness. It does so by computing a local descriptor for every point in the point cloud and aggregrating them into a global descriptor. The LPDNet model we use is trained on outdoor maps which have different characteristics from indoor maps. However, the authors of LPDNet claim that a model trained on outdoor data can also effectively be used for indoor data. Figure \ref{fig:lpdnet_architecture} shows the network architecture of LPDNet.

% network architecture w/figure
% training data / transfer learning

\begin{figure}[h]
    \centering
    \includegraphics*[width=\textwidth]{./fig/network_architecture.png}
    \caption{Diagram showing LPDNet network architecture.}
    \label{fig:lpdnet_architecture}
\end{figure}

\subsubsection{Contextual Embedding}
After computing a descriptor for each individual room we augment them by taking into account the descriptor of the neighbourhood. For every room we find their neighbours and merge their geometry into one voxel grid, for which we compute a new descriptor. We do this step multiple times for neighbours that are at most one or multiple steps away from the room. We then append the descriptors of the neighbourhood to the room's descriptor. By doing so we can distinguish between rooms with similar geometry but dissimilar neighbourhoods, which are often present in indoor environments.

\subsubsection{Initial Matching}
The above steps are applied to both partial maps. This gives us two sets of descriptors \(\mathcal{E}_A,\ \mathcal{E}_B\) representing the embedding of the geometry of the rooms of both topometric maps.

To identify the most likely overlapping rooms between the partial maps we find the one-to-one mapping between the elements of \(\mathcal{E}_A\) and \(\mathcal{E}_B\) that maximizes the similarity (or minimizes the distances) between the chosen pairs. This is an example of the unbalanced assignment problem, which consists of finding a matching in a weighted bipartite graph that minimizes the sum of its edge weights. It is unbalanced because there may be more nodes in one part of the bipartite graph than the other, which means it is not possible to assign every node in one part to a node in the other. 

To construct the weighted bipartite graph we first find the Cartesian product of the feature vectors 

\begin{equation}
    \label{eq:E_ab}
    \mathcal{E}_{AB} = \mathcal{E}_A \times \mathcal{E}_B = \{(a,b) \mid a \in \mathcal{E}_A,\ b \in \mathcal{E}_B\}
\end{equation}

We then compute the Euclidean distance in feature space between every pair of nodes in \(\mathcal{E}_{AB}\), creating the cost matrix that represents the weighted bipartite graph 

\begin{equation}
    \label{eq:C}
    \mathbf{C} \in \mathbb{R}^{|V_a| \times |V_b|}
\end{equation}
\begin{equation}
    \label{eq:C}
    \mathbf{C}_{ij} = ||a - b||,\ (a,\ b) \in \mathcal{E}_{AB},\ a = \mathcal{E}_{A,\ i},\ b = \mathcal{E}_{B,\ j},\ \mathbf{C}_{ij} \in \mathbb{R}^+
\end{equation}

We can then find unbalanced assignment using the Jonker-Volgenant algorithm. We denote the resulting matching between the nodes of both partial maps and their distance in feature space as a set of triples 

\begin{equation}
    \label{eq:M}
    \mathbf{M} = \{(i,\ j,\ d) \mid  |V_a| \geq i,\ |V_b|\geq j,\ d = \mathbf{C}_{ij}\}
\end{equation}

\subsubsection{Hypothesis growing}

In practice it is unlikely that every match in \(\mathbf{M}\) is correct. However, we can use them as seeds to generate hypotheses similar to the approach described in Huang \citep{huang_topological_2005}. Starting at each initial match we get the neighbourhood of both nodes. We then construct a new cost matrix from the Euclidean distance between the embeddings of both neighbourhoods, again creating a weighted bipartite graph for which we can solve the assignment problem. By doing this we identify which neighbours of the nodes in the match are most likely to also match. We recursively apply this step to the matching neighbours to grow our initial matches into hypotheses. To decrease the risk of incorrectly identifying neighbourhood matches we constrain hypothesis growing in two ways. First, the cost of two potential matches must be below a given threshold \(c_{max}\). Second, a newly identified match may not bring the existing matching too much out of alignment. To check this, we perform a registration (see next section) between the centroids of the geometry of the identified matches at every step of the hypothesis growing. If the error increases between steps, and the increase is too large such that \(\triangle e \geq \triangle e_{max} \), then the matching is rejected. By adjusting the values of \(c_{max}\) and \(\triangle e_{max}\) more or less uncertainty is allowed when growing hypotheses.

\pagebreak                                                     
\section{Results}
We evaluated the algorithm described in the methodology section on several datasets. In this section we describe these datasets and show the results we achieved with them. The shown results are divided into three sections based on the major components of our methodology: map extraction, map matching and map fusion. 

\subsection{Datasets}
In this section we describe the datasets that we used to test our algorithm. We also describe how we prepared the ground truth datasets so as to be able to compare our results to them and objectively measure their performance. 

\subsubsection{Simulated Scan}
To objectively evaluate the performance of our methodology it is necessary to compare the results to a ground truth. This ground truth must contain the following elements: room labels and their topological relationships to evaluate map extraction, room matches between partial maps to evaluate map matching, and the transformations between partial maps to evaluate map fusion. When capturing real-world data, the second and third elements are especially difficult to determine. Furthermore, little pre-existing data is available that contains exactly these elements. To solve this problem we simulate partial maps from annotated global maps from various sources. The annotations are integer labels for every point representing the ground truth room segmentation and a graph representing the ground truth topological map. 

We create the partials maps by manually defining a trajectory for each desired partial map and simulating an agent moving along them, scanning the global map at a set interval. We do this by using the DDA algorithm described in the methodology section. This allows us to objectively evaluate our approach for a large number of different scenarios at the cost of missing some of the subtleties inherent in non-simulated partial maps, such as changes in the environment and measurement error. To mitigate this, we apply pre-processing steps to the global map. These pre-processing steps are: random removal of points, adding noise to points and random rotation and translation of the point cloud. The latter's purpose is to simulate the unknown transformation between real-world partial maps. By comparing the results of our approach to the annotations in the ground truth global map we can objectively measure the performance of map extraction, matching and fusion. 

\subsubsection{Stanford 3D Indoor Scene Dataset}
The Stanford 3D Indoor Scene Dataset (S3DIS) consists of a collection of 3D scans of six different indoor environments (Area 1 through 6) and the raw measurements used to create them. The environments all span a single floor. Each environment scan in its original state consists of a number of point clouds, one for each room in the building. We merge these separate point clouds into a single cloud but retain the room labels as point attributes, as we use these as our ground truth room labels for the map extraction step. We manually created the topological graph of each area. The S3DIS was captured using high-end 3D scanners and thus has a high point density.

\subsubsection{Collaborative SLAM Dataset}
The Collaborative SLAM Dataset (CSLAMD) is a dataset meant specifically for collaborative SLAM. It consists of three environments (House, flat and lab), each consisting of multiple partial maps and their ground truth transformations. Two of the environments consist of multiple storeys. The partial maps were captured using a low-end 3D scanner and thus has a low point density. We merge the partial maps of each environment into a single global map to create the simulated partial maps described above. We then manually annotate each environment with our interpretation of an appropriate room segmentation and manually create the topological graph.

\subsubsection{Various Sources}
Finally, we also include datasets from various sources. These include a two-storey house with basement and an office conference room, both captured using high-end 3D scanners. We annotated these datasets by hand according to our interpretation of the spaces and their segmentation into rooms and manually create the topological graph. 

\subsubsection{Overview}

\begin{tabular}{ c c c c }
    Scan & Dataset & Number of rooms & Number of storeys \\ 
    Area 1 & S3DIS & 44 & 1 \\  
    Area 2 & S3DIS & 40 & 1 \\
    Area 3 & S3DIS & 23 & 1 \\
    Area 4 & S3DIS & 49 & 1 \\
    Area 5 & S3DIS & 68 & 1 \\
    Area 6 & S3DIS & 48 & 1 \\

    House & CSLAMD & cell9 & 2 \\
    Flat & CSLAMD & cell9 & 1 \\
    Lab & CSLAMD & cell9 & 4 \\

    House2 & Various & 23 & 2 \\

\end{tabular}

\subsection{Map Extraction}
In this section we show the results of our map extraction approach. To evaluate the results of our map extraction approach we compare the extracted topometric partial maps to the ground truth global map. For each node's geometry in a topometric map we find its Jaccard index with every node in the global map's geometry. This gives us a weighted bipartite graph, one part being the nodes in the partial map and the other being the nodes in the global map, with the weight representing the Jaccard index between nodes' geometry. We then assign each node in the partial map to a single node in the global map such that the sum of weights is maximized using linear assignment as described in the map matching section. This gives us the correspondence between nodes in the partial map and nodes in the global map. Afterwards, we compute the mean Jaccard index of the correspondences. This metric is called Mean Intersection over Union (MIoU). 

\begin{tabular}{ c c c c }
    Scan & Dataset & MIoU \\ 
    Area 1 & S3DIS & 0.732 \\  
    Area 2 & S3DIS & 0.828 \\
    Area 3 & S3DIS &  \\
    Area 4 & S3DIS &  \\
    Area 5 & S3DIS &  \\
    Area 6 & S3DIS &  \\

    House & CSLAMD &  &  \\
    Flat & CSLAMD &  &  \\
    Lab & CSLAMD &  &  \\

    House2 & Various &  &  \\

\end{tabular}

\subsection{Map Matching}
In this section we show the results of our map matching approach. Using to partial map to global map node correspondences described in the previous section we can distinguish incorrect matches from correct matches; a match between two partial maps is correct if both nodes in the match correspond to the same node in the global map. With this information we evaluate our map matching approach based on four metrics: precision, accuracy, recall and F1. 

\begin{tabular}{ c c c c c c}
    Scan & Dataset & Precision & Accuracy & Recall & F1 \\ 
    Area 1 & S3DIS &  &  & & \\  
    Area 2 & S3DIS &  &  & &\\
    Area 3 & S3DIS &  &  & &\\
    Area 4 & S3DIS &  &  & &\\
    Area 5 & S3DIS &  &  & &\\
    Area 6 & S3DIS &  &  & &\\

    House & CSLAMD &  &  & &\\
    Flat & CSLAMD &  &  & &\\
    Lab & CSLAMD &  &  & &\\

    House2 & Various &  &  \\

\end{tabular}


\subsection{Map Fusion}
In this section we show the results of our map merging approach. We evaluate the performance of our map merging approach by comparing the computed transformation to the transformations applied to the ground truth when simulating the partial maps. We also evaluate the extracted topological graph by computing its edit distance to the ground truth global map's topological graph.

\begin{tabular}{ c c c c }
    Scan & Dataset & Registration Error & Edit Distance \\ 
    Area 1 & S3DIS &  &  \\  
    Area 2 & S3DIS &  &  \\
    Area 3 & S3DIS &  &  \\
    Area 4 & S3DIS &  &  \\
    Area 5 & S3DIS &  &  \\
    Area 6 & S3DIS &  &  \\

    House & CSLAMD &  &  \\
    Flat & CSLAMD &  &  \\
    Lab & CSLAMD &  &  \\

    Elspeet & Various &  &  \\

\end{tabular}
\pagebreak
\section{Discussion}

\subsection{Map Extraction}
\label{section:map_extraction}
In the previous section we showed the results of our map extraction approach. In this section we will discuss these results. For the majority of tested environments the resultant room segmentation closely matches the ground truth room segmentation. This is especially the case in environments where rooms have clear delineations; environments where rooms have walls between them and are only connected by small openings. The results of our room segmentation approach match less closely in environments where this is not the case. Our approach often splits single rooms that are large in one or multiple dimensions, such as hallways or auditoriums, into multiple rooms. Additionally, rooms where there are obstructions to the view from inside the room are also split into multiple parts. However, these effects do not necessarily indicate a failure of our approach. The segmentation in the ground truth data is based on human intuition about what separates a room from its neighbours. Although room segmentation based on visibility clustering often closely matches this intuition it is inherently different as it does not take into account the intended use of rooms. Where humans might recognize that a long hallway or a large hall serves a single purpose, and should therefore be considered as the same room, visibility clustering fails to take this subjective interpretation of purpose into account. Nevertheless, the objective visibility clustering approach comes remarkably close to the subjective human approach. The opposite also occurs, where two or more rooms that are separate in the ground truth data are not separate in the room segmentation. This mostly occurs when there are no obstructions between two adjacent rooms. This can often be resolved by changing the clustering parameters. However, when the only separation between two rooms is based on purpose and not on visibility then our approach cannot separate them.

A common failure mode of our approach, which causes room segmentation to fail completely, is when the input point cloud data is of insufficient density to construct a connected navigation graph. In this case, the voxels belonging to the navigation graph are identified correctly but there are gaps between voxels. This can be solved by increasing the size of the kernel used for constructing the neighbourhood graph of the navigable voxels. However, this has the side effect that voxels that are not actually navigable are added to the navigation graph. The result is that low elevated surfaces with sloping sides, beds for example, are added to the navigation graph. While this usually does not have a large effect on map extraction in extreme cases it can also cause the ceiling to become part of the navigation graph. This will usually cause significant errors in room segmentation, as the view from above the ceiling towards the rest of the map is often completely unobstructed.

Another way that room segmentation may fail is when stairs have very shallow treads and steep rises (respectively the horizontal and vertical part of its steps). This causes the stick kernel approach to fail to label the stairs' voxels as navigable, which means it will not be included in the navigation graph. This is because the wide part of the stick kernel placed on one step may intersect with the next step. If there are no other connections between two storeys then this will cause one or multiple storeys to become disconnected from the navigation graph, excluding it from the extracted topometric map. This problem can be resolved by changing the dimensions of the stick kernel. However, this may in turn cause other problems. Increasing the height of the thin part of the stick kernel causes low elevated surfaces with sloping sides to be included in the navigation graph, as described in the previous paragraph. Decreasing the radius of the stick kernel's wide part will include parts of the map that are not actually navigable in the navigation graph. The problem can also be solved by increasing the size of the kernel used to construct the navigable voxels' neighbourhood graph to force the stairs'  voxels to become connected even though some are missing but this causes the same issues as described in the previous paragraph.

Differences between the ground truth topological graph and the extracted topological graph are caused by differences in room segmentation. One such case is when a hallway connected to a room is split into multiple rooms around the opening towards the connected room. This will result in a triangular subgraph between the two parts of the hallway and the connected room, which in reality should just be a single edge between the hallway and the room. 

% TODO: add figures illustrating each problem

\subsection{Map Matching}
In this section we will discuss the results of our map matching approach. We will discuss this in two parts. The first concerns the results for feature embedding, the second concerns the hypothesis growing step. 

\paragraph{Feature embedding}
As seen in the results section the performance of initial matching strongly depends on the feature embedding approach. We find that the deep learning approach gives the best results here as it is able to handle large differences in segmentation (cases where the voxels assigned to the same room between two maps have a large overlap but do not match exactly) and completeness (cases where a room in one partial map has not been captured completely). In contrast, the engineered feature and the spectral approaches are not able to deal with incompleteness and to a lesser degree differences in segmentation. However, even for the deep learning approach incompleteness and segmentation differences have a significant negative impact on performance. In some cases this problem is resolved by the graph convolution step due to the added information about the rooms' neighbours. However, the results of this are inconsistent and it is currently often better to not apply graph convolution at all. 

Another factor that a significant negative impact on feature embedding performance is the similarity of rooms. In cases where there are many near identical rooms, which is often the case in environments like offices and hospitals feature embedding fails to distinguish between them. This is especially the case when a large voxel size is used because it removes details like furnishing and clutter that may help distinguish between rooms. When this is the case only the shape of the room can be used to identify it which may match very closely to similar rooms. When similar rooms are not adjacent graph convolution can reduce this problem, especially when the adjacent rooms are very distinctive. However, the opposite is true when similar rooms are adjacent. In this case, graph convolution makes the rooms' already similar feature embedding even more similar, making it even harder to distinguish between them. This poses a problem because graph convolution can't be applied selectively and it is currently unknown how to predict when it will improve performance and when it won't. Based on our observations, all three embedding approaches suffer with distinguishing similar rooms, graph convolution or not, with deep learning performing slightly better than the other two.

\paragraph{Hypothesis growing}
Based on our results we find that hypothesis growing has the potential to significantly improve map matching performance. However, its performance greatly depends on the quality of the feature embedding. If the initial matching is completely incorrect then hypothesis growing also fails. Even if some initial matches are correct, the performance of hypothesis growing still depends on the quality of the feature embedding. An exception to this is when the initial match used to grow a hypothesis is at the end of a linear chain of rooms. In this case the growing step will succesfully match all rooms in the chain given that the feature embedding between two matches does not fall under the similarity threshold. 

We also find that the transformation estimation step is able to constrain region growing to give more reasonable results by preventing matches from being made that would bring the existing matching out of alignment. Adjusting the transformation difference threshold upwards allows the region growing to grow further while increasing the risk that an incorrect match is made. In reverse, adjusting it downwards makes region growing more restrained and decreases the risk of incorrect matches. In the ideal case with no differences between partial maps and their segmentation the threshold could be set to zero as any correct match would align perfectly with the existing matches. However, incompleteness of data and error between partial maps causes the centroids of two matching rooms to be in different positions, introducing error into the alignment. From this it follows that incompleteness, and to a lesser degree error, also has a significant impact on the hypothesis growing.

% TODO: expand on hypothesis clustering

\subsection{Map Fusion}


\subsection{Future Works}
In this section we discuss our recommendations for future developments of the three major components of this thesis: map extraction, map matching and map fusion. We make these recommendations based on the achieved results and our research during the creation of this thesis. 

\subsubsection{Map extraction}
% TODO: make sure bleeding is referenced in discussion section

% advanced room segmentation
\paragraph{Room segmentation}
As mentioned before, the current room segmentation approach differs from how humans identify rooms as it does not take into account the perceived purpose of the room. Taking both visibility and purpose into account could lead to a segmentation that more closely matches one a human would perform, but more importantly, one that is more consistent between partial maps and more robust to incompleteness and error. Assuming that a computer can succesfully infer a room's purpose based on its voxelized representation, large rooms such as hallways that are now arbitrarily divided based on clustering could be merged into one whole. Inferring a room's purpose is a subjective task that would be very hard to solve using traditional techniques. To achieve this, we suggest training a deep learning model that is suitable for segmentation of voxel or point cloud representations, such as PVCNN or DGCNN, on a manually labeled ground truth dataset. 

% robust navigation graph extraction
\paragraph{Robust topological graph extraction}
One of the major bottlenecks of our approach is the extraction of the topological graph. If this step fails then map extraction fails and map matching becomes impossible. Thus, in the future it would be important to identify an approach to topological graph extraction that is robust to the failure modes described in section \ref{section:map_extraction}. This would include a way to interpolate the navigation graph to fill in any missing holes that cause it to become disconnected. This could be done using interpolation techniques from image processing applied to 3-dimensional data or more advanced techniques such as the PCN network discussed in section \ref{section:map_match}. In both cases the main challenge is differentiating between voxels that should be present but are missing and voxels that should not be; making the wrong choice could lead to worse results than no interpolation at all. For example, a small gap in the floor can be either a piece of missing data or a gap between walls. Solving this challenge could drastically improve the robustness of our map extraction approach and should be considered in the future.

% hierarchical representation
\paragraph{Hierarchical topological representation}
Our current approach to map extraction results in a topometric map with a 'flat' graph representing the environment's topology. In reality, indoor environments can be considered as complex multi-level hierarchies. For example, a building can be divided into storeys which contain rooms which contain areas. As such, the structure of an indoor environment can also be represented by a hierarchical graph. The extra information contained in such a graph could be applied to improve feature embedding performance. For example, two rooms are more similar if their storeys are also similar than if they are not. Various techniques have been proposed in the literature surrounding this subject but none so far are based on visibility clustering. We hypothesize that by applying hierarchical clustering to visibility it is possible to extract a hierarchical structure of the environment. Whether this is true and what the characteristics of the resultant map are could be a valuable topic of research. A hierarchical topological representation could allow or require different methods for hypothesis growing and map fusion. For the former, work in the area of hierarchical graph matching could be applied. The latter is, to the knowledge of the author, still unresearched. 

\paragraph{Volumetric representation}
% volumetric representation
Our current approach only represents the surface of the geometry of the environment. This is because 3D scanners only capture that part of the environment. In reality, indoor environments are enclosed volumes. A possible improvement to our approach would be to describe the environment's geometry volumetrically, where each occupied voxel represents a volume within the building that is not obstructed. This would require a method to extract the volume from the surface geometry. Various research into this topic exists (REFERENCES HERE) but they fail when parts of the ceiling or floor are missing from the map, which is often the case. They also make assumptions such as constant storey height and only horizontal floors, which are often not the case in reality. Using a volumetric representation of the environment has a number of benefits. Navigation would no longer only be possible on the floor but throughout the entire volume. In reality most scanners use the floor to navigate but the advent of drones that operate indoors might change this. In addition, a volumetric representation might improve feature embedding performance as it describes the room more completely (DOES THIS MAKE SENSE?). Another avenue of research that moving to a volumetric approach would require would concern efficiently storing and processing the exponentially larger amount of data used in doing so. While this subject has been considered in this research by using sparse voxel octrees, variations on this or other data structures might be more effective. For example, implementations of sparse voxel octrees for GPUs exist.



\bibliographystyle{apacite}
\bibliography{bibliography}


\clearpage
\printglossaries



\end{document}
