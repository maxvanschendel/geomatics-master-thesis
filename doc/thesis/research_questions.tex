\section{Research Questions}

\subsection{Main question}
How can we apply topometric representations of indoor environments to solve the map merging problem?

\subsection{Subquestions}
\begin{enumerate}
    \item In what way can partial topometric maps be extracted from partial point cloud maps?
    \item What approach is best suited for identifying matches between partial topometric maps?
    \item How can the identified matches be used to fuse two or more partial topometric maps into a global topometric map?
\end{enumerate}

\subsection{Scope}

During this thesis the following will be created.

\begin{enumerate}
    \item A program that is capable of topometric map extraction from point clouds and topometric map merging.
    \item An analysis of the program's performance on publically available standard datasets.
    \item Reports containing documentation and background research.
\end{enumerate}

To better delineate the scope of the thesis we provide several aspects that will \textbf{not} be researched or discussed. 

\begin{enumerate}
    \item Map merging using known relative poses between agents or meeting strategies. Agent behaviour is assumed to be independent and agents are not able to sense eachother.
    \item Map merging using observations unrelated to the environment's geometric or topological characteristics. E.g. the environment's colour or actively transmitted beacon signals.
    \item Map merging assisted by a priori knowledge of the environment. E.g. building information models or floor plans.
    \item Map merging using the pose graphs of agents. Agent poses are assumed to be unknown.
    \item Achieving near real-time performance.
\end{enumerate}